%!TEX root = ../../main.tex

\renewcommand{\arraystretch}{0.8} % adjust to make lines closer together

\subsection{Fast Matrix Multiplication Algorithms} \label{sec:Fast Matrix Multiplication Algorithms}

    Recall from \Cref{sec:Matrix-Matrix Products} the way matrix multiplications are
    performed. In fact, \Cref{eq:two_by_two_mat_mul} demonstrates how to multiply
    two matrices $\mathbf{A, B} \in \mathbb{R}^{2\times2}$. Matrices $\mathbf{A}$
    and $\mathbf{B}$ don't necessarily need to be of size 2 by 2, as long as they
    are of a size that is a multiple of 2, we can perform such matrix multiplication
    using a recursive algorithm like the one on \Cref{alg:recursive_mat_mul}. This
    is the only time such algorithm is presented using the pseudocode format, for
    the sake of simplicity algorithms will be showcased in the format displayed in
    \ref{eq:classic_matmul}.

    \begin{algorithm}[!ht]
        \caption{MatMul}
        \label{alg:recursive_mat_mul}
        \begin{algorithmic}[0]
            \Function{$\mathbf{C} = $ MatMul}{$\mathbf{A, B}$}
                \If{dim($\mathbf{A}$) = dim($\mathbf{A}$) = 1}
                    \State{\textbf{return} $\mathbf{A\cdot B}$}
                \EndIf
                \State{Divide into quadrants:
                    \(
                    \mathbf{A} =
                    \left[
                    \begin{array}{cc}
                        \mathbf{A}_{11} & \mathbf{A}_{12} \\
                        \mathbf{A}_{21} & \mathbf{A}_{22}
                    \end{array}
                    \right]
                    \mathbf{B} =
                    \left[
                    \begin{array}{cc}
                        \mathbf{B}_{11} & \mathbf{B}_{12} \\
                        \mathbf{B}_{21} & \mathbf{B}_{22}
                    \end{array}
                    \right]
                    \)
                }
                \State{$\mathbf{M}_1$ = MatMul($\mathbf{A_{11}, B_{11}}$)}
                \State{$\mathbf{M}_2$ = MatMul($\mathbf{A_{12}, B_{21}}$)}
                \State{$\mathbf{M}_3$ = MatMul($\mathbf{A_{11}, B_{12}}$)}
                \State{$\mathbf{M}_4$ = MatMul($\mathbf{A_{12}, B_{22}}$)}
                \State{$\mathbf{M}_5$ = MatMul($\mathbf{A_{21}, B_{21}}$)}
                \State{$\mathbf{M}_6$ = MatMul($\mathbf{A_{22}, B_{12}}$)}
                \State{$\mathbf{M}_7$ = MatMul($\mathbf{A_{21}, B_{12}}$)}
                \State{$\mathbf{M}_8$ = MatMul($\mathbf{A_{22}, B_{22}}$)}
                \vspace{10pt}
                \State{\textbf{return} \(
                    \mathbf{C} =
                    \left[
                    \begin{array}{cc}
                        \mathbf{M}_1 + \mathbf{M}_2 & \mathbf{M}_3 + \mathbf{M}_4 \\
                        \mathbf{M}_5 + \mathbf{M}_6 & \mathbf{M}_7 + \mathbf{M}_8 \\
                    \end{array}
                    \right]
                    \)
                }
            \EndFunction
        \end{algorithmic}
    \end{algorithm}


    \begin{center}
        \textbf{Classic 2 by 2 Matrix Multiplication Algorithm}
        \vspace{-20pt}
    \end{center}
    \begin{equation}
        \begin{array}{rcl}
            \color{blue} \mathbf{M}_1 & = & \mathbf{A}_{11}\cdot \mathbf{B}_{11} \\
            \color{blue} \mathbf{M}_2 & = & \mathbf{A}_{12}\cdot \mathbf{B}_{21} \\
            \color{blue} \mathbf{M}_3 & = & \mathbf{A}_{11}\cdot \mathbf{B}_{12} \\
            \color{blue} \mathbf{M}_4 & = & \mathbf{A}_{12}\cdot \mathbf{B}_{22} \\
            \color{blue} \mathbf{M}_5 & = & \mathbf{A}_{21}\cdot \mathbf{B}_{11} \\
            \color{blue} \mathbf{M}_6 & = & \mathbf{A}_{22}\cdot \mathbf{B}_{21} \\
            \color{blue} \mathbf{M}_7 & = & \mathbf{A}_{21}\cdot \mathbf{B}_{12} \\
            \color{blue} \mathbf{M}_8 & = & \mathbf{A}_{22}\cdot \mathbf{B}_{22} \\
            \mathbf{C}_{11} & = & \color{blue} \mathbf{M}_1 \color{black} + \color{blue} \mathbf{M}_2 \\
            \mathbf{C}_{12} & = & \color{blue} \mathbf{M}_3 \color{black} + \color{blue} \mathbf{M}_4 \\
            \mathbf{C}_{21} & = & \color{blue} \mathbf{M}_5 \color{black} + \color{blue} \mathbf{M}_6 \\
            \mathbf{C}_{22} & = & \color{blue} \mathbf{M}_7 \color{black} + \color{blue} \mathbf{M}_8
        \end{array}
        \label{eq:classic_matmul}
    \end{equation}


    This algorithm involves 8 multiplications and 4 additions of matrices of size
    $\nicefrac n2$. Therefore, our computational complexity becomes $T(n) =
    8T(\nicefrac{n}{2}) + O(n^2) = O(n^{\log_2 8}) = O(n^3)$. This computational
    cost can be decresed by carefully manipulating our multiplications and additions
    in order to reduce the number of recursive calls. In 1969, Volker Strassen
    became the first to develop an algorithm with cost less than $O(n^3)$, which
    made way for \textbf{fast matrix multiplication algorithms}. His classic
    algorithm is showcased in \ref{eq:classic_strassen}. His algorithm involves 7
    multiplications and 18 additions, but recall that additions are computationally
    less expensive than multiplications. The computational complexity of Strassen's
    algorithm is $T(n) = 7T(\nicefrac n2) + O(n^2) = O(n^{\log_2 7}) \approx O(n^{2.81})$. 

    \begin{center}
        \textbf{Classic Strassen's Algorithm}
        \vspace{-20pt}
    \end{center}
    \begin{equation}
        \begin{array}{rcl}
            \color{blue} \mathbf{M}_1 & = & (\mathbf{A}_{11} + \mathbf{A}_{22})\cdot (\mathbf{B}_{11} + \mathbf{B}_{22}) \\
            \color{blue} \mathbf{M}_2 & = & (\mathbf{A}_{12} + \mathbf{A}_{22})\cdot \mathbf{B}_{11} \\
            \color{blue} \mathbf{M}_3 & = & \mathbf{A}_{11}\cdot (\mathbf{B}_{21} - \mathbf{B}_{22}) \\
            \color{blue} \mathbf{M}_4 & = & \mathbf{A}_{22}\cdot (\mathbf{B}_{12} - \mathbf{B}_{11}) \\
            \color{blue} \mathbf{M}_5 & = & (\mathbf{A}_{11} + \mathbf{A}_{21})\cdot \mathbf{B}_{22} \\
            \color{blue} \mathbf{M}_6 & = & (\mathbf{A}_{12} - \mathbf{A}_{11})\cdot (\mathbf{B}_{11} + \mathbf{B}_{21}) \\
            \color{blue} \mathbf{M}_7 & = & (\mathbf{A}_{21} - \mathbf{A}_{22})\cdot (\mathbf{B}_{12} + \mathbf{B}_{22}) \\
            \mathbf{C}_{11} & = & \color{blue} \mathbf{M}_1 \color{black} + \color{blue} \mathbf{M}_4 \color{black} - \color{blue} \mathbf{M}_5 \color{black} + \color{blue} \mathbf{M}_7 \\
            \mathbf{C}_{12} & = & \color{blue} \mathbf{M}_3 \color{black} + \color{blue} \mathbf{M}_5 \\
            \mathbf{C}_{21} & = & \color{blue} \mathbf{M}_2 \color{black} + \color{blue} \mathbf{M}_4 \\
            \mathbf{C}_{22} & = & \color{blue} \mathbf{M}_1 \color{black} - \color{blue} \mathbf{M}_2 \color{black} + \color{blue} \mathbf{M}_3 \color{black} + \color{blue} \mathbf{M}_6
        \end{array}
        \label{eq:classic_strassen}
    \end{equation}

    Strassen's algorithm can be rearranged; we can modify the additions and
    multiplies to get another a permuted version of his algorithm with the same
    number of multiplies and additions. An example of these variations of
    Strassen's algorithms can be seen in \ref{eq:per_strassen}. This permutted
    version will be relevant for \Cref{sec:Cyclic Invariance} as it contains a
    special structure that is hard to visualize in the
    \Cref{eq:classic_strassen}. Notice how the two algorithms are the same,
    except that $\mathbf M_3$ became $\mathbf M_6$, $\mathbf M_4$ became
    $\mathbf M_3$, and $\mathbf M_6$ became $\mathbf M_4$, and the additions in
    $\mathbf C$ changed respectively. It is explained in \Cref{sec:Cyclic
    Invariant Matrix Multiplication Algorithms} that this is due to a change in
    the columns of a KTensor decomposition. 

    \begin{center}
        \textbf{Permuted Strassen's Algorithm}
        \vspace{-20pt}
    \end{center}
    \begin{equation}
        \begin{array}{rcl}
            \color{blue} \mathbf{M}_1 & = & (\mathbf{A}_{11} + \mathbf{A}_{22})\cdot (\mathbf{B}_{11} + \mathbf{B}_{22}) \\
            \color{blue} \mathbf{M}_2 & = & (\mathbf{A}_{12} + \mathbf{A}_{22})\cdot \mathbf{B}_{11} \\
            \color{blue} \mathbf{M}_3 & = & \mathbf{A}_{22}\cdot (\mathbf{B}_{12} - \mathbf{B}_{11}) \\
            \color{blue} \mathbf{M}_4 & = & (\mathbf{A}_{21} - \mathbf{A}_{22})\cdot (\mathbf{B}_{12} + \mathbf{B}_{22}) \\
            \color{blue} \mathbf{M}_5 & = & (\mathbf{A}_{11} + \mathbf{A}_{21})\cdot \mathbf{B}_{22} \\
            \color{blue} \mathbf{M}_6 & = & \mathbf{A}_{11}\cdot (\mathbf{B}_{21} - \mathbf{B}_{22}) \\
            \color{blue} \mathbf{M}_7 & = & (\mathbf{A}_{12} - \mathbf{A}_{11})\cdot (\mathbf{B}_{11} + \mathbf{B}_{21}) \\
            \mathbf{C}_{11} & = & \color{blue} \mathbf{M}_1 \color{black} + \color{blue} \mathbf{M}_3 \color{black} - \color{blue} \mathbf{M}_4 \color{black} - \color{blue} \mathbf{M}_6 \\
            \mathbf{C}_{12} & = & \color{blue} \mathbf{M}_2 \color{black} + \color{blue} \mathbf{M}_3 \\
            \mathbf{C}_{21} & = & \color{blue} \mathbf{M}_5 \color{black} + \color{blue} \mathbf{M}_6 \\
            \mathbf{C}_{22} & = & \color{blue} \mathbf{M}_1 \color{black} - \color{blue} \mathbf{M}_2 \color{black} + \color{blue} \mathbf{M}_5 \color{black} + \color{blue} \mathbf{M}_7
        \end{array}
        \label{eq:per_strassen}
    \end{equation}
    
    It is at this point that an important distinction must be made; what is
    considered a new algorithm? Both Strassen's algorithm and its permutation
    are considered Strassen's algorithm because they both involve 7
    multiplications. This is because the term \textit{Strassen's algorithm} has
    become an all-encapsulating term for any 2 by 2 matrix multiplication
    algorithm that uses 7 multiplications instead of the classic 8. For example,
    take a look at \Cref{eq:var_strassen}. It is no longer just a permutted
    version of \Cref{eq:classic_strassen} as this onw has 24 matrix additions as
    opposed to the original's 18. Again, since matrix addition is not the
    bottleneck, both of these algorithms have the same computational complexity.
    The algorithm below is still called Strassen's algorithm for the reasons
    mentioned previously. Furthermore, this variant version will also be
    relevant in \Cref{sec:Cyclic Invariance} as it contains the same srtructure
    we will explore later on, but slighly different. It has been proven that
    there are no possible fast matrix multiplication algorithms with 6
    multiplications. 

    \begin{center}
        \textbf{Variant Strassen's Algorithm}
        \vspace{-20pt}
    \end{center}
    \begin{equation}
        \begin{array}{rcl}
            \color{blue} \mathbf{M}_1 & = & \mathbf{A}_{11}\cdot \mathbf{B}_{11} \\
            \color{blue} \mathbf{M}_2 & = & (\mathbf{A}_{12} + \mathbf{A}_{22})\cdot (\mathbf{B}_{12} + \mathbf{B}_{22}) \\
            \color{blue} \mathbf{M}_3 & = & (\mathbf{A}_{22} - \mathbf{A}_{21})\cdot (\mathbf{B}_{22} - \mathbf{B}_{21}) \\
            \color{blue} \mathbf{M}_4 & = & (\mathbf{A}_{21} - \mathbf{A}_{12} - \mathbf{A}_{22})\cdot (\mathbf{B}_{21} - \mathbf{B}_{12} - \mathbf{B}_{22}) \\
            \color{blue} \mathbf{M}_5 & = & (-\mathbf{A}_{12})\cdot (-\mathbf{B}_{21}) \\
            \color{blue} \mathbf{M}_6 & = & (\mathbf{A}_{11} - \mathbf{A}_{12} + \mathbf{A}_{21} - \mathbf{A}_{22})\cdot (- \mathbf{B}_{12}) \\
            \color{blue} \mathbf{M}_7 & = & (- \mathbf{A}_{21})\cdot (\mathbf{B}_{11} - \mathbf{B}_{12} + \mathbf{B}_{21} - \mathbf{B}_{22}) \\
            \mathbf{C}_{11} & = & \color{blue} \mathbf{M}_1 \color{black} + \color{blue} \mathbf{M}_5 \\
            \mathbf{C}_{22} & = & \color{blue} \mathbf{M}_4 \color{black} - \color{blue} \mathbf{M}_3 \color{black} + \color{blue} \mathbf{M}_5 \color{black} + \color{blue} \mathbf{M}_6 \\
            \mathbf{C}_{22} & = & \color{blue} \mathbf{M}_2 \color{black} - \color{blue} \mathbf{M}_4 \color{black} - \color{blue} \mathbf{M}_5 \color{black} + \color{blue} \mathbf{M}_7 \\
            \mathbf{C}_{22} & = & \color{blue} \mathbf{M}_2 \color{black} + \color{blue} \mathbf{M}_3 \color{black} - \color{blue} \mathbf{M}_4 \color{black} + \color{blue} \mathbf{M}_5
        \end{array}
        \label{eq:var_strassen}
    \end{equation}
    

    Thus, a question arises, how many possibilities are there? We could search
    for different solutions that all have the same number of multiplications by
    performing either an eshaustive search or an optimization search on each
    parameter involved in matrix multiplication as seen in
    \ref{eq:serching_fast_matmul}. If we are searching for a discrete solution
    with only -1, 0, 1 as coefficients then we have $3^{84}$ possibilities,
    though not all of these possibilities would be valid algotithms. We will see
    how we can decrease this number later on to something much more feasible
    using the structure hinted at in these algorithms. Before we can move on to
    how to search for these algorithms non-exhaustively, we turn to the
    protagonist of this project.



    \begin{center}
        \textbf{Exhaustive Search of Fast MatMul Algorithms}
        \vspace{-10pt}
    \end{center}
    \begin{equation}
        \scalebox{0.9}{
            $\displaystyle
            \begin{array}{rcl}
                \color{blue} \mathbf{M}_1 & = & (\textcolor{purple}{ u_{11}^{(1)}} \mathbf{A}_{11} + \textcolor{purple}{u_{12}^{(1)}} \mathbf{A}_{12} + \textcolor{purple}{ u_{21}^{(1)}} \mathbf{A}_{21} + \textcolor{purple}{ u_{22}^{(1)}} \mathbf{A}_{22}) \cdot (\textcolor{purple}{ v_{11}^{(1)}} \mathbf{B}_{11} + \textcolor{purple}{v_{12}^{(1)}} \mathbf{B}_{12} + \textcolor{purple}{v_{21}^{(1)}} \mathbf{B}_{21} \textcolor{purple}{ v_{22}^{(1)}} + \mathbf{B}_{22}) \\
                \color{blue} \mathbf{M}_2 & = & (\textcolor{purple}{ u_{11}^{(2)}} \mathbf{A}_{11} + \textcolor{purple}{u_{12}^{(2)}} \mathbf{A}_{12} + \textcolor{purple}{ u_{21}^{(2)}} \mathbf{A}_{21} + \textcolor{purple}{ u_{22}^{(2)}} \mathbf{A}_{22}) \cdot (\textcolor{purple}{ v_{11}^{(2)}} \mathbf{B}_{11} + \textcolor{purple}{v_{12}^{(2)}} \mathbf{B}_{12} + \textcolor{purple}{v_{21}^{(2)}} \mathbf{B}_{21} \textcolor{purple}{ v_{22}^{(2)}} + \mathbf{B}_{22}) \\
                \color{blue} \mathbf{M}_3 & = & (\textcolor{purple}{ u_{11}^{(3)}} \mathbf{A}_{11} + \textcolor{purple}{u_{12}^{(3)}} \mathbf{A}_{12} + \textcolor{purple}{ u_{21}^{(3)}} \mathbf{A}_{21} + \textcolor{purple}{ u_{22}^{(3)}} \mathbf{A}_{22}) \cdot (\textcolor{purple}{ v_{11}^{(3)}} \mathbf{B}_{11} + \textcolor{purple}{v_{12}^{(3)}} \mathbf{B}_{12} + \textcolor{purple}{v_{21}^{(3)}} \mathbf{B}_{21} \textcolor{purple}{ v_{22}^{(3)}} + \mathbf{B}_{22}) \\
                \color{blue} \mathbf{M}_4 & = & (\textcolor{purple}{ u_{11}^{(4)}} \mathbf{A}_{11} + \textcolor{purple}{u_{12}^{(4)}} \mathbf{A}_{12} + \textcolor{purple}{ u_{21}^{(4)}} \mathbf{A}_{21} + \textcolor{purple}{ u_{22}^{(4)}} \mathbf{A}_{22}) \cdot (\textcolor{purple}{ v_{11}^{(4)}} \mathbf{B}_{11} + \textcolor{purple}{v_{12}^{(4)}} \mathbf{B}_{12} + \textcolor{purple}{v_{21}^{(4)}} \mathbf{B}_{21} \textcolor{purple}{ v_{22}^{(4)}} + \mathbf{B}_{22}) \\
                \color{blue} \mathbf{M}_5 & = & (\textcolor{purple}{ u_{11}^{(5)}} \mathbf{A}_{11} + \textcolor{purple}{u_{12}^{(5)}} \mathbf{A}_{12} + \textcolor{purple}{ u_{21}^{(5)}} \mathbf{A}_{21} + \textcolor{purple}{ u_{22}^{(5)}} \mathbf{A}_{22}) \cdot (\textcolor{purple}{ v_{11}^{(5)}} \mathbf{B}_{11} + \textcolor{purple}{v_{12}^{(5)}} \mathbf{B}_{12} + \textcolor{purple}{v_{21}^{(5)}} \mathbf{B}_{21} \textcolor{purple}{ v_{22}^{(5)}} + \mathbf{B}_{22}) \\
                \color{blue} \mathbf{M}_6 & = & (\textcolor{purple}{ u_{11}^{(6)}} \mathbf{A}_{11} + \textcolor{purple}{u_{12}^{(6)}} \mathbf{A}_{12} + \textcolor{purple}{ u_{21}^{(6)}} \mathbf{A}_{21} + \textcolor{purple}{ u_{22}^{(6)}} \mathbf{A}_{22}) \cdot (\textcolor{purple}{ v_{11}^{(6)}} \mathbf{B}_{11} + \textcolor{purple}{v_{12}^{(6)}} \mathbf{B}_{12} + \textcolor{purple}{v_{21}^{(6)}} \mathbf{B}_{21} \textcolor{purple}{ v_{22}^{(6)}} + \mathbf{B}_{22}) \\
                \color{blue} \mathbf{M}_7 & = & (\textcolor{purple}{ u_{11}^{(7)}} \mathbf{A}_{11} + \textcolor{purple}{u_{12}^{(7)}} \mathbf{A}_{12} + \textcolor{purple}{ u_{21}^{(7)}} \mathbf{A}_{21} + \textcolor{purple}{ u_{22}^{(7)}} \mathbf{A}_{22}) \cdot (\textcolor{purple}{ v_{11}^{(7)}} \mathbf{B}_{11} + \textcolor{purple}{v_{12}^{(7)}} \mathbf{B}_{12} + \textcolor{purple}{v_{21}^{(7)}} \mathbf{B}_{21} \textcolor{purple}{ v_{22}^{(7)}} + \mathbf{B}_{22}) \\ \\
                \mathbf{C}_{11} & = & \textcolor{purple}{w_{11}^{(1)}} \color{blue} \mathbf{M}_1 + \textcolor{purple}{w_{11}^{(2)}} \color{blue} \mathbf{M}_2 + \textcolor{purple}{w_{11}^{(3)}} \color{blue} \mathbf{M}_3 + \textcolor{purple}{w_{11}^{(4)}} \color{blue} \mathbf{M}_4 + \textcolor{purple}{w_{11}^{(5)}} \color{blue} \mathbf{M}_5 + \textcolor{purple}{w_{11}^{(6)}} \color{blue} \mathbf{M}_6 + \textcolor{purple}{w_{11}^{(7)}} \color{blue} \mathbf{M}_7 \\
                \mathbf{C}_{12} & = & \textcolor{purple}{w_{12}^{(1)}} \color{blue} \mathbf{M}_1 + \textcolor{purple}{w_{12}^{(2)}} \color{blue} \mathbf{M}_2 + \textcolor{purple}{w_{12}^{(3)}} \color{blue} \mathbf{M}_3 + \textcolor{purple}{w_{12}^{(4)}} \color{blue} \mathbf{M}_4 + \textcolor{purple}{w_{12}^{(5)}} \color{blue} \mathbf{M}_5 + \textcolor{purple}{w_{12}^{(6)}} \color{blue} \mathbf{M}_6 + \textcolor{purple}{w_{12}^{(7)}} \color{blue} \mathbf{M}_7 \\
                \mathbf{C}_{21} & = & \textcolor{purple}{w_{21}^{(1)}} \color{blue} \mathbf{M}_1 + \textcolor{purple}{w_{21}^{(2)}} \color{blue} \mathbf{M}_2 + \textcolor{purple}{w_{21}^{(3)}} \color{blue} \mathbf{M}_3 + \textcolor{purple}{w_{21}^{(4)}} \color{blue} \mathbf{M}_4 + \textcolor{purple}{w_{21}^{(5)}} \color{blue} \mathbf{M}_5 + \textcolor{purple}{w_{21}^{(6)}} \color{blue} \mathbf{M}_6 + \textcolor{purple}{w_{21}^{(7)}} \color{blue} \mathbf{M}_7 \\
                \mathbf{C}_{22} & = & \textcolor{purple}{w_{22}^{(1)}} \color{blue} \mathbf{M}_1 + \textcolor{purple}{w_{22}^{(2)}} \color{blue} \mathbf{M}_2 + \textcolor{purple}{w_{22}^{(3)}} \color{blue} \mathbf{M}_3 + \textcolor{purple}{w_{22}^{(4)}} \color{blue} \mathbf{M}_4 + \textcolor{purple}{w_{22}^{(5)}} \color{blue} \mathbf{M}_5 + \textcolor{purple}{w_{22}^{(6)}} \color{blue} \mathbf{M}_6 + \textcolor{purple}{w_{22}^{(7)}} \color{blue} \mathbf{M}_7 \\
            \end{array}
            $
        }
        \label{eq:serching_fast_matmul}
    \end{equation}



\subsection{The Matrix Multiplication Tensor} \label{sec:The Matrix Multiplication Tensor}

    There is a specific type of tensor that is crucial to this project: the
    Matrix Multiplication Tensor. Matrix multiplication can be performed in
    tensor format using this tensor as visualized in \Cref{fig:matmul_tensor}.
    If we wish to multiply two matrices $\mathbf{A} \in \mathbb{R}^{m\times n}$
    and $\mathbf{B}\in \mathbb{R}^{n\times p}$, we can form the respective
    matrix multiplication tensor $\mathcal{M}\in \mathbb{R}^{mn \times np\times
    mp}$ through \Cref{alg:matmul_tensor}. In order to do so, we vectorize
    $\mathbf{A}$ and $\mathbf{B}$ and multiply them through TTMs in the first
    and second mode respectively. The output, in the third mode, is the
    vectorization of the output $\mathbf{C} = \mathbf{A}\cdot \mathbf{B}$
    transposed (i.e. $\mathbf{C^\intercal}$). This project is concerned only
    with square matrices, thus we always assume $\mathbf{A}, \mathbf{B},
    \mathbf{C}\in \mathbb{R}^{n\times n}$ and
    $\mathcal{M}\in\mathbb{R}^{n^2\times n^2\times n^2}$.

    \begin{algorithm}
        \caption{Forming the Matrix Multiplication Tensor}
        \label{alg:matmul_tensor}
        \begin{algorithmic}[0]
            \Function{$\mathcal{T}$ = MatMul-Tensor}{$m, n, p$}
                \State{$\mathcal{T}$ = \texttt{zeros}($mn, np, mp$)} \Comment{Initialize Tensor}
                \For{$i = 1:m$}
                    \For{$i = 1:m$}
                        \For{$i = 1:m$}
                            \State{$\mathcal{T}(mj + i, nk + j, pi + k) = 1$}
                        \EndFor
                    \EndFor
                \EndFor
            \EndFunction
        \end{algorithmic}
    \end{algorithm}

    \begin{figure}
        \centering
        \begin{tikzpicture}[scale=1,namenode/.style={scale=1}]
    % These distances can be meddled with to get you what you want
    % Within reason...
    \def\ix{2} %
    \def\iy{2} %
    \def\iz{2} %
    \def\r{\iz/4}
    \def\corescale{1.75}
    \def\rx{\ix/\corescale}
    \def\ry{\iy/\corescale}
    \def\rz{\iz/\corescale}
    \def\sp{\rx/6}
    
    % This line chooses the starting location of the object
    \coordinate (MFrontLowerLeft) at (0,0);

    % This draws the front og the cube
    \draw (MFrontLowerLeft) rectangle ++ (\ix,\iy);

    % This scope draws the top face of the cube
    \begin{scope}[shift={(MFrontLowerLeft)}, canvas is zx plane at y=\iy,rotate=90]
        \draw (0,0) rectangle ++ (\ix,\iz);
    \end{scope}

    % This scope draws the left side of the cube
    \begin{scope}[shift={(MFrontLowerLeft)},canvas is zy plane at x=\ix,rotate=90]
        \draw (0,0) rectangle ++ (\iy,\iz); %
    \end{scope}

    % This writes the M in the front face of the cube
    \node[namenode] at ($(MFrontLowerLeft) + (0.5*\ix, 0.5*\iy)$)  {$\mathcal{M}$};

    \coordinate (ALowerLeft) at ($(MFrontLowerLeft) + (-\ix - 0.5, \iy-\r/1.5)$);
    \draw (ALowerLeft) rectangle ++ (\iy, \r/1.25);
    \node[namenode] at ($(ALowerLeft) + (\iy/2, 0.2)$) {$A$};

    \coordinate (BLowerLeft) at ($(MFrontLowerLeft) + (\ix+1.15,0.775)$);
    \draw (BLowerLeft) rectangle ++ (\r/1.25, \iy);
    \node[namenode] at ($(BLowerLeft) + (0.19, \iy/2)$) {$B$};

    \coordinate (L1) at ($(MFrontLowerLeft) + (\ix + \sp, \sp)$);
    \coordinate (L1H) at ($(L1) + (0, \iy)$);
    \draw[thin, densely dotted] (L1) -- (L1H);
    \coordinate (L1) at ($(MFrontLowerLeft) + (\ix + 2*\sp, 2*\sp)$);
    \coordinate (L1H) at ($(L1) + (0, \iy)$);
    \draw[thin, densely dotted] (L1) -- (L1H);
    \coordinate (L1) at ($(MFrontLowerLeft) + (\ix + 3*\sp, 3*\sp)$);
    \coordinate (L1H) at ($(L1) + (0, \iy)$);
    \draw[thin, densely dotted] (L1) -- (L1H);
    
    \coordinate (L1) at ($(MFrontLowerLeft) + (\sp, \iy + \sp)$);
    \coordinate (L1H) at ($(L1) + (\ix, 0)$);
    \draw[thin, densely dotted] (L1) -- (L1H);
    \coordinate (L1) at ($(MFrontLowerLeft) + (2*\sp, \iy + 2*\sp)$);
    \coordinate (L1H) at ($(L1) + (\ix, 0)$);
    \draw[thin, densely dotted] (L1) -- (L1H);
    \coordinate (L1) at ($(MFrontLowerLeft) + (3*\sp, \iy + 3*\sp)$);
    \coordinate (L1H) at ($(L1) + (\ix, 0)$);
    \draw[thin, densely dotted] (L1) -- (L1H);

    \node[namenode] at ($(MFrontLowerLeft) + (3.25+0.5*\ix, 0.5*\iy)$)  {$ = $};

    % This scope draws the top face of the cube
    \coordinate (CLowerLeft) at ($(MFrontLowerLeft) + (4.85, -\iy/3)$);
    \begin{scope}[shift={(CLowerLeft)}, canvas is zx plane at y=\iy,rotate=90]
        \draw (0,0) rectangle ++ (\r/1.25, \ix);
    \end{scope}

    \node[namenode] at ($(CLowerLeft) + (1.25, \iy + 0.3)$) {$C^\text{T}$};
\end{tikzpicture}
        \caption{Matrix Multiplication in Tensor Format}
        \label{fig:matmul_tensor}
    \end{figure}

    It is at this point we return to tensor decompositions, specifically the CP
    decomposition. Recall from \ref{sec:Kruskal Tensors and The CP
    Decomposition} that the decomposition compresses an input tensor into $r$
    $d$-way outer product components. It turns out, that if we decompose the
    matrix multiplication tensor using the CP Decomposition, there is a hidden
    fast matrix multiplication algorithm embedded in the components of the CP
    decomposition. In fact, the number of components $r$ corresponds to the
    number of multiplications in the algorithm. This in return results in two
    different implications (could argue that they mean the same thing in two
    different directions but we seperate these implications for the sake of
    understanding). The first implication is that given an algorithm, say one of
    the $3^84$ for $2\times 2$ algorithms with rank 7, we can form the KTensor
    of the corresponding algorithm, namely $\mathcal{\hat{M}}$ and take the norm
    of the difference from the original matmul tensor. If $\|\mathcal{M -
    \hat{M}}\| = 0$, then we have a valid algorithm, there are easier ways to
    check for the validity of an algorithm, but this helps with forming the
    search equation. Before continuing with the second implication, we must
    understand how to visualize the hidden algorithm in the factor matrices of
    a KTensor. 
    
    Below is the original Strassen's algorithm with two representations, the one
    we have seen before and the KTensor representation. The way to interpret the
    algorithm on the right, is that each horizontal lines separate the factor
    matrices, therefore just like in \Cref{fig:KTensor} the factor matrices
    $\mathbf{A}, \mathbf{B}$ and $\mathbf{C}$ are separated by the horizontal
    lines, the columns of the factor matrices represent the chicken feet
    components. If you pay close attention you can see how the columns
    representing $\mathbf{M}_\ell$ correspond to the representation on the
    right. If a 1 or -1 appear in the right representation then they appear as
    $\mathbf{A}_{ij}$ or $\mathbf{-A}_{ij}$ respectively on the left.

    \vspace{-50pt}
    \begin{multicols}{2}
        \begin{equation*}
            \begin{array}{rcl}
                \color{blue} \mathbf{M}_1 & = & (\mathbf{A}_{11} + \mathbf{A}_{22})\cdot (\mathbf{B}_{11} + \mathbf{B}_{22}) \\
                \color{blue} \mathbf{M}_2 & = & (\mathbf{A}_{12} + \mathbf{A}_{22})\cdot \mathbf{B}_{11} \\
                \color{blue} \mathbf{M}_3 & = & \mathbf{A}_{11}\cdot (\mathbf{B}_{21} - \mathbf{B}_{22}) \\
                \color{blue} \mathbf{M}_4 & = & \mathbf{A}_{22}\cdot (\mathbf{B}_{12} - \mathbf{B}_{11}) \\
                \color{blue} \mathbf{M}_5 & = & (\mathbf{A}_{11} + \mathbf{A}_{21})\cdot \mathbf{B}_{22} \\
                \color{blue} \mathbf{M}_6 & = & (\mathbf{A}_{12} - \mathbf{A}_{11})\cdot (\mathbf{B}_{11} + \mathbf{B}_{21}) \\
                \color{blue} \mathbf{M}_7 & = & (\mathbf{A}_{21} - \mathbf{A}_{22})\cdot (\mathbf{B}_{12} + \mathbf{B}_{22}) \\
                \mathbf{C}_{11} & = & \color{blue} \mathbf{M}_1 \color{black} + \color{blue} \mathbf{M}_4 \color{black} - \color{blue} \mathbf{M}_5 \color{black} + \color{blue} \mathbf{M}_7 \\
                \mathbf{C}_{12} & = & \color{blue} \mathbf{M}_3 \color{black} + \color{blue} \mathbf{M}_5 \\
                \mathbf{C}_{21} & = & \color{blue} \mathbf{M}_2 \color{black} + \color{blue} \mathbf{M}_4 \\
                \mathbf{C}_{22} & = & \color{blue} \mathbf{M}_1 \color{black} - \color{blue} \mathbf{M}_2 \color{black} + \color{blue} \mathbf{M}_3 \color{black} + \color{blue} \mathbf{M}_6
            \end{array}
        \end{equation*}

        \columnbreak

        \setlength{\arraycolsep}{3pt}
        \[\begin{array}{c|ccccccc}
            & \color{blue} \mathbf{M}_1 & \color{blue} \mathbf{M}_2 & \color{blue} \mathbf{M}_3 & \color{blue} \mathbf{M}_4 & \color{blue} \mathbf{M}_5 & \color{blue} \mathbf{M}_6 & \color{blue} \mathbf{M}_7 \\
            \hline
            \mathbf{A}_{11} & 1 & 0 & 1 & 0 & 1 & -1 & 0 \\
            \mathbf{A}_{12} & 0 & 1 & 0 & 0 & 0 & 1 & 0 \\
            \mathbf{A}_{21} & 0 & 0 & 0 & 0 & 1 & 0 & 1 \\
            \mathbf{A}_{22} & 1 & 1 & 0 & 1 & 0 & 0 & -1 \\
            \hline
            \mathbf{B}_{11} & 1 & 1 & 0 & -1 & 0 & 1 & 0 \\
            \mathbf{B}_{12} & 0 & 0 & 0 & 1 & 0 & 0 & 1 \\
            \mathbf{B}_{21} & 0 & 0 & 1 & 0 & 0 & 1 & 0 \\
            \mathbf{B}_{22} & 1 & 0 & -1 & 0 & 1 & 0 & 1 \\
            \hline
            \mathbf{C}_{11} & 1 & 0 & 0 & 1 & -1 & 0 & 1 \\
            \mathbf{C}_{21} & 0 & 0 & 1 & 0 & 1 & 0 & 0 \\
            \mathbf{C}_{12} & 0 & 1 & 0 & 1 & 0 & 0 & 0 \\
            \mathbf{C}_{22} & 1 & -1 & 1 & 0 & 0 & 1 & 0 \\
    \end{array}\]
    \end{multicols}
    
    
    The other implication, is that we can decompose the matmul tensor with
    specified CP rank (as in \Cref{eq:cp_opt}) through an optimization algorithm
    to search for fast matrix multiplication algorithms.






\subsection{Damped Gauss Newton Optimization for CP Decompositions}\label{sec:Damped Gauss Newton Optimization for CP Decompositions}
    
    Recall from \Cref{eq:cp_least_squares} that we perform a CP decomposition
    with the least squares error of our approximation. We will do so using
    optimization methods for reasons that will become clear soon. Optimization
    methods work with vector input, so for our purposes let 

    \begin{equation*}
        \mathbf{v} = \text{vec}
        \left(
        \left[
            \begin{array}{c}
                \mathbf{A} \\
                \mathbf{B} \\
                \mathbf{C}
            \end{array}
        \right]
        \right) = 
        \left[
            \begin{array}{c}
                \text{vec}(\mathbf{A}) \\
                \text{vec}(\mathbf{B}) \\
                \text{vec}(\mathbf{C})
            \end{array}
        \right]
        \in \mathbb{R}^{3nr}
    \end{equation*}

    Furthermore, let $\phi: \mathbb{R}^n \to \mathbb{R}^m$ be the nonlinear
    function. 

    \begin{equation}
        \phi (\mathbf{v}) = \text{vec}(\mathcal{M} - \llbracket \mathbf{A, B, C} \rrbracket)
    \end{equation}

    Thus, \Cref{eq:cp_least_squares} becomes 

    \begin{equation}
        \min_{A, B, C} f(\mathbf{v}) = \frac{1}{2} \|\phi (\mathbf{v})\|^2
    \end{equation}

    With that out of the way we must choose an optimization method. Gradient is
    too simple, taking one step further we reach Newton's method. Newton's
    method gets the search direction by using Newton's equation
    \[\nabla^2 f(\mathbf{v})\mathbf{d}_k = -\nabla f(\mathbf{v})\] where
    $\mathbf{d}_k$ is our approximation. We dislike the Hessian here because it
    is computationally expensive and difficult to derive mathematicall (perhaps
    not now but for the cylic part). Thus, we prefer the Gauss-Newton method
    which approximates the Hessian with $\mathbf{J^\intercal J}$ where
    $\mathbf{J}: \mathbb{R}^n\to \mathbb{R}^{n\times m}$ is the jacobian of
    $\phi$: \[(\mathbf{J^\intercal J})\mathbf{d}_k = -\nabla f(\mathbf{v})\] But
    even then, the Gauss-Newton matrix $\mathbf{J^\intercal J}$ can be singular,
    we add a damping parameter, $\lambda \mathbf{I}$, to enforce positive
    definiteness. Thus, we reach the damped Gauss-Newton Method:
    \[(\mathbf{J^\intercal J} + \lambda\mathbf{I})\mathbf{d}_k = -\nabla
    f(\mathbf{v})\]. At this point, we have two tasks at hand. Given our
    $\mathbf{v}, f(\mathbf{v})$ and $\phi(\mathbf{v})$, we must derive the
    gradient of $f$ for the right-hand side and the jacobian of $\phi$ for the
    left-hand side (when really we just want to know how to apply it to a
    vector). The gradient is easy. 

    \begin{equation} \label{eq:cp_dgn_gradient}
        \nabla f = \text{vec}
        \left(
        \left[
        \begin{array}{c}
            \frac{\partial f}{\partial \mathbf{A}} \\
            \frac{\partial f}{\partial \mathbf{B}} \\
            \frac{\partial f}{\partial \mathbf{C}}
        \end{array}
        \right]
        \right) = 
        \left[
            \begin{array}{c}
                \frac{\partial f}{\partial \text{vec}(\mathbf{A})} \\
                \frac{\partial f}{\partial \text{vec}(\mathbf{B})} \\
                \frac{\partial f}{\partial \text{vec}(\mathbf{C})}
            \end{array}
        \right]
        \in \mathbb{R}^{3nr}
    \end{equation}

    Where each partial derivative is defined as:

    \begin{equation}
        \begin{array}{rcl}
            \frac{\partial f}{\partial \mathbf{A}} & = & -\mathbf{M}_{(1)}(\mathbf{C\odot B}) + \mathbf{A(C^\intercal C\ast B^\intercal B)} \\ 
            \frac{\partial f}{\partial \mathbf{B}} & = & -\mathbf{M}_{(1)}(\mathbf{C\odot A}) + \mathbf{B(C^\intercal C\ast A^\intercal A)} \\ 
            \frac{\partial f}{\partial \mathbf{C}} & = & -\mathbf{M}_{(1)}(\mathbf{B\odot A}) + \mathbf{C(B^\intercal B\ast A^\intercal A)} 
        \end{array}
        \label{eq:cp_dgn_partials}
    \end{equation}
    
    The Jacobian is not so easy. 
    
    \begin{equation} \label{eq:cp_dgn_jacobian}
        \mathbf{J} = [\mathbf{J}_\mathbf{A} + \mathbf{J}_\mathbf{B} + \mathbf{J}_\mathbf{C}] \in \mathbb{R}^{n^3\times 3nr}
    \end{equation}
    
    Where
    
    \begin{equation}
        \begin{array}{rcl}
            \mathbf{J}_\mathbf{A} \equiv \frac{\partial \phi}{\partial \text{vec}(\mathbf{A})} = (\mathbf{C} \odot \mathbf{B}) \otimes \mathbf{I} \in \mathbb{R}^{n^3\times nr} \\
            \mathbf{J}_\mathbf{B} \equiv \frac{\partial \phi}{\partial \text{vec}(\mathbf{B})} = \mathbf{\Pi}_2^\intercal (\mathbf{C} \odot \mathbf{A}) \otimes \mathbf{I} \in \mathbb{R}^{n^3\times nr} \\
            \mathbf{J}_\mathbf{C} \equiv \frac{\partial \phi}{\partial \text{vec}(\mathbf{C})} = \mathbf{\Pi}_3^\intercal(\mathbf{B} \odot \mathbf{A}) \otimes \mathbf{I} \in \mathbb{R}^{n^3\times nr}
        \end{array}
        \label{eq:cp_dgn_jacobians}
    \end{equation}

    such that that $\Pi_k$ is the tensor perfect shuffle matrix, such that
    vec($\mathcal{X}$) = $\Pi_k$ vec($\mathbf{X}_{(k)}$), and $\Pi_1$ is not
    written explicitly because it is the identity matrix. We consider fast
    computation of $\mathbf{J^\intercal J} + \lambda \mathbf{I}$. Rather than
    forming $\mathbf{J}$ or $\mathbf{J^\intercal J}$ explicitly, we use the
    structure of these matrices to compute the matrix-vector product without
    computing any explicit Kronecker or Khatri-Rao products. From
    \Cref{eq:cp_dgn_jacobian} we have that the block structure of
    $\mathbf{J^\intercal J}$ is 

    \begin{equation*}
        \mathbf{J^\intercal J} = 
        \left[
        \begin{array}{ccc}
            \mathbf{J_A^\intercal J_A} & \mathbf{J_A^\intercal J_B} & \mathbf{J_A^\intercal J_C} \\ 
            \mathbf{J_B^\intercal J_A} & \mathbf{J_B^\intercal J_B} & \mathbf{J_B^\intercal J_C} \\ 
            \mathbf{J_C^\intercal J_A} & \mathbf{J_C^\intercal J_B} & \mathbf{J_C^\intercal J_C} 
        \end{array}
        \right]
    \end{equation*}
    
    then
    
    \begin{equation*}
        \mathbf{J^\intercal Jv} = 
        \left[
        \begin{array}{c}
            \mathbf{J_A^\intercal J_A\text{vec}(A)} + \mathbf{J_A^\intercal J_B\text{vec}(B)} + \mathbf{J_A^\intercal J_C\text{vec}(C)} \\ 
            \mathbf{J_B^\intercal J_A\text{vec}(A)} + \mathbf{J_B^\intercal J_B\text{vec}(B)} + \mathbf{J_B^\intercal J_C\text{vec}(C)} \\ 
            \mathbf{J_C^\intercal J_A\text{vec}(A)} + \mathbf{J_C^\intercal J_B\text{vec}(B)} + \mathbf{J_C^\intercal J_C\text{vec}(C)} 
        \end{array}
        \right]
    \end{equation*}
    
    Which actually becomes
    
    \begin{equation*}
        \mathbf{J^\intercal Jv} = 
        \left[
        \begin{array}{c}
            \text{vec}(\mathbf{\bar{A}(B^\intercal B \ast C^\intercal C) + A(\bar{B}^\intercal B\ast C^\intercal C) + A(B^\intercal B \ast \bar{C}^\intercal C)}) \\
            \text{vec}(\mathbf{B(\bar{A}^\intercal A \ast C^\intercal C) + \bar{B}(A^\intercal A\ast C^\intercal C) + B(A^\intercal A \ast \bar{C}^\intercal C)}) \\
            \text{vec}(\mathbf{C(\bar{A}^\intercal A \ast B^\intercal B) + C(A^\intercal A\ast \bar{B}^\intercal B) + \bar{C}(A^\intercal A \ast B^\intercal B)}) \\

        \end{array}
        \right]
    \end{equation*}


    \begin{algorithm}
        \caption{Damped Gauss-Newton On The Matrix Multiplication Tensor}
        \label{alg:CP-DGN}
        \begin{algorithmic}
            \State{\textbf{Input:} Matrix Multiplication Tensor $\mathcal{M}$,}
            \State{\hspace{3.25em} CP Tensor Rank $r$,}
            \State{\hspace{3.25em} Damping Parameter $\lambda \in \mathbb{R}^+$,}
            \State{\hspace{3.25em} Convergence Tolerance $\epsilon > 0$}
            
            \State \textbf{Output:} CP Tensor $\mathcal{K}$
            \Function{DGN}{$\mathcal{M}, r, \lambda, \epsilon$}
                \State{Initialize K and K\_prev to be a cell of length 3 of $n^2 \times r$ matrices}

                \For{$i = 1:MaxIters$}
                    \State{f $ \longleftarrow \frac{1}{2} \|\mathcal{M - K}\| $} \Comment{Compute Function Value}
                    \State{$\nabla \mathbf{f} \longleftarrow [\text{vec} \left( \frac{\partial f}{\partial \mathbf A}\right) \text{vec} \left( \frac{\partial f}{\partial \mathbf B}\right) \text{vec} \left( \frac{\partial f}{\partial \mathbf C}\right)]^\intercal$} \Comment{Compute Gradient}
                    \State{$S \longleftarrow \text{Solution to} (\textbf{J}^T\textbf{J} + \lambda I)K = - \nabla \mathbf{f}$}
                    \While{\textit{Goldstein Conditions Are Satisfied}}
                        \State{K $\longleftarrow$ K\_prev + $\alpha S$}
                        \State{f$_{new} \longleftarrow \frac{1}{2} \|\mathcal{M - K}\|$}
                        \State{$\alpha \longleftarrow \alpha / 2$}
                    \EndWhile{}
                    \If{f - f$_{new} < \epsilon$}
                        \State{\textbf{break}}
                    \EndIf{}
                \EndFor{}
            \EndFunction
        \end{algorithmic}
    \end{algorithm}
