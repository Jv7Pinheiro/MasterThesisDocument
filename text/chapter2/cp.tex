%!TEX root = ../../main.tex

Despite the tile, the focus of this project are not CP Decompositions. Rather,
their active use in searching for fast matrix multiplication algorithms. There
was a glimpse of these types of algorithms in \Cref{sec:Matrix-Matrix Products},
but the rabbit hole goes deeper than that. Therefore, the use of CP
Decompositons of a special tensor to discover new fast matrix multiplication
algorithms is left for \Cref{sec:The Matrix Multiplication Tensor}, and
\Cref{sec:Matrix Multiplication Algorithms} starts this chapter with some
much-needed background about Matrix Multiplication Algorithms.

% The goal for the first project is difficult to describe. Preliminary work on
% attempting to search for fast matrix multiplication algorithm shows that it is a
% daunting task. Even though exhaustive search is not enough to prove that an
% algorithm doesn't exist for certain ranks, it is clear that there are many more
% places where an algorithm fails to exist than the ones where it actually exists.
% But have implemented a new way of searching these algorithms much more
% efficiently than what was done in the past, and it could prove itself worthy of
% future use. We have more recently turned to a more mathematical approach to the
% project with the aid of Dr. Frank Moore and Dr. Pratyush Mishra. But where we
% are heading is still unknown. We wish to share some both the new method of
% searching for algorithm and some of the interesting solutions we found.

% The work done in the first project is more extensive. Recall that we can
% actively search for fast matrix multiplication algorithms by
% optimization as in \Cref{Opt}. We can do so by performing an CP
% Decomposition on the matrix multiplication tensor by specifying the size
% of the matrix $n$ and the number of multiplications in the algorithm
% $R$. This is described in algorithm \ref{alg:CP-DGN} where the details
% of how to compute the Function value, the gradient and the Jacobian are
% left as an exercise to the reader. But now, how do we modify it so that
% we can decrease the number of possibilities from $3^{84}$ to $3^{28}$ by
% relying on the cyclic invariance structure described earlier? The
% details of so can be seen in Algorithm \ref{alg:Mod-CP-DGN}. I once
% again leave the now much, much more complicated details of how to
% compute the function value, the gradient and the Jacobian as another
% exercise to the reader. 

\section{Matrix Multiplication Algorithms} \label{sec:Matrix Multiplication Algorithms}
    %!TEX root = ../../main.tex

\renewcommand{\arraystretch}{0.8} % adjust to make lines closer together

\subsection{Fast Matrix Multiplication Algorithms} \label{sec:Fast Matrix Multiplication Algorithms}

    Recall from \Cref{sec:Matrix-Matrix Products} the way matrix multiplications
    are performed. In fact, \Cref{eq:two_by_two_mat_mul} demonstrates how to
    multiply two matrices $\mathbf{A, B} \in \mathbb{R}^{2\times2}$. Matrices
    $\mathbf{A}$ and $\mathbf{B}$ don't necessarily need to be of size 2 by 2,
    we can perform such matrix multiplication using a recursive algorithm like
    the one on \Cref{alg:recursive_mat_mul}. This is the only time such
    algorithm is presented using the pseudocode format, for the sake of
    simplicity algorithms will be showcased in the format displayed in
    \ref{eq:classic_matmul}.

    \begin{algorithm}[!ht]
        \caption{MatMul}
        \label{alg:recursive_mat_mul}
        \begin{algorithmic}[0]
            \Function{$\mathbf{C} = $ MatMul}{$\mathbf{A, B}$}
                \If{dim($\mathbf{A}$) = dim($\mathbf{A}$) = 1}
                    \State{\textbf{return} $\mathbf{A\cdot B}$}
                \EndIf
                \State{Divide into quadrants:
                    \(
                    \mathbf{A} =
                    \left[
                    \begin{array}{cc}
                        \mathbf{A}_{11} & \mathbf{A}_{12} \\
                        \mathbf{A}_{21} & \mathbf{A}_{22}
                    \end{array}
                    \right]
                    \mathbf{B} =
                    \left[
                    \begin{array}{cc}
                        \mathbf{B}_{11} & \mathbf{B}_{12} \\
                        \mathbf{B}_{21} & \mathbf{B}_{22}
                    \end{array}
                    \right]
                    \)
                }
                \State{$\mathbf{M}_1$ = MatMul($\mathbf{A_{11}, B_{11}}$)}
                \State{$\mathbf{M}_2$ = MatMul($\mathbf{A_{12}, B_{21}}$)}
                \State{$\mathbf{M}_3$ = MatMul($\mathbf{A_{11}, B_{12}}$)}
                \State{$\mathbf{M}_4$ = MatMul($\mathbf{A_{12}, B_{22}}$)}
                \State{$\mathbf{M}_5$ = MatMul($\mathbf{A_{21}, B_{21}}$)}
                \State{$\mathbf{M}_6$ = MatMul($\mathbf{A_{22}, B_{12}}$)}
                \State{$\mathbf{M}_7$ = MatMul($\mathbf{A_{21}, B_{12}}$)}
                \State{$\mathbf{M}_8$ = MatMul($\mathbf{A_{22}, B_{22}}$)}
                \vspace{10pt}
                \State{\textbf{return} \(
                    \mathbf{C} =
                    \left[
                    \begin{array}{cc}
                        \mathbf{M}_1 + \mathbf{M}_2 & \mathbf{M}_3 + \mathbf{M}_4 \\
                        \mathbf{M}_5 + \mathbf{M}_6 & \mathbf{M}_7 + \mathbf{M}_8 \\
                    \end{array}
                    \right]
                    \)
                }
            \EndFunction
        \end{algorithmic}
    \end{algorithm}


    \begin{center}
        \textbf{Classic 2 by 2 Matrix Multiplication Algorithm}
        \vspace{-20pt}
    \end{center}
    \begin{equation}
        \begin{array}{rcl}
            \color{blue} \mathbf{M}_1 & = & \mathbf{A}_{11}\cdot \mathbf{B}_{11} \\
            \color{blue} \mathbf{M}_2 & = & \mathbf{A}_{12}\cdot \mathbf{B}_{21} \\
            \color{blue} \mathbf{M}_3 & = & \mathbf{A}_{11}\cdot \mathbf{B}_{12} \\
            \color{blue} \mathbf{M}_4 & = & \mathbf{A}_{12}\cdot \mathbf{B}_{22} \\
            \color{blue} \mathbf{M}_5 & = & \mathbf{A}_{21}\cdot \mathbf{B}_{11} \\
            \color{blue} \mathbf{M}_6 & = & \mathbf{A}_{22}\cdot \mathbf{B}_{21} \\
            \color{blue} \mathbf{M}_7 & = & \mathbf{A}_{21}\cdot \mathbf{B}_{12} \\
            \color{blue} \mathbf{M}_8 & = & \mathbf{A}_{22}\cdot \mathbf{B}_{22} \\
            \mathbf{C}_{11} & = & \color{blue} \mathbf{M}_1 \color{black} + \color{blue} \mathbf{M}_2 \\
            \mathbf{C}_{12} & = & \color{blue} \mathbf{M}_3 \color{black} + \color{blue} \mathbf{M}_4 \\
            \mathbf{C}_{21} & = & \color{blue} \mathbf{M}_5 \color{black} + \color{blue} \mathbf{M}_6 \\
            \mathbf{C}_{22} & = & \color{blue} \mathbf{M}_7 \color{black} + \color{blue} \mathbf{M}_8
        \end{array}
        \label{eq:classic_matmul}
    \end{equation}


    This algorithm involves 8 multiplications and 4 additions of matrices of size
    $\nicefrac n2$. Therefore, our computational complexity becomes $T(n) =
    8T(\nicefrac{n}{2}) + O(n^2) = O(n^{\log_2 8}) = O(n^3)$. This computational
    cost can be decreased by carefully manipulating our multiplications and additions
    in order to reduce the number of recursive calls. In 1969, Volker Strassen
    became the first to develop an algorithm with cost less than $O(n^3)$, which
    made way for \textbf{fast matrix multiplication algorithms}. His classic
    algorithm is showcased in \ref{eq:classic_strassen}. His algorithm involves 7
    multiplications and 18 additions, but recall that additions are computationally
    less expensive than multiplications. The computational complexity of Strassen's
    algorithm is $T(n) = 7T(\nicefrac n2) + O(n^2) = O(n^{\log_2 7}) \approx O(n^{2.81})$. 

    \begin{center}
        \textbf{Classic Strassen's Algorithm}
        \vspace{-20pt}
    \end{center}
    \begin{equation}
        \begin{array}{rcl}
            \color{blue} \mathbf{M}_1 & = & (\mathbf{A}_{11} + \mathbf{A}_{22})\cdot (\mathbf{B}_{11} + \mathbf{B}_{22}) \\
            \color{blue} \mathbf{M}_2 & = & (\mathbf{A}_{12} + \mathbf{A}_{22})\cdot \mathbf{B}_{11} \\
            \color{blue} \mathbf{M}_3 & = & \mathbf{A}_{11}\cdot (\mathbf{B}_{21} - \mathbf{B}_{22}) \\
            \color{blue} \mathbf{M}_4 & = & \mathbf{A}_{22}\cdot (\mathbf{B}_{12} - \mathbf{B}_{11}) \\
            \color{blue} \mathbf{M}_5 & = & (\mathbf{A}_{11} + \mathbf{A}_{21})\cdot \mathbf{B}_{22} \\
            \color{blue} \mathbf{M}_6 & = & (\mathbf{A}_{12} - \mathbf{A}_{11})\cdot (\mathbf{B}_{11} + \mathbf{B}_{21}) \\
            \color{blue} \mathbf{M}_7 & = & (\mathbf{A}_{21} - \mathbf{A}_{22})\cdot (\mathbf{B}_{12} + \mathbf{B}_{22}) \\
            \mathbf{C}_{11} & = & \color{blue} \mathbf{M}_1 \color{black} + \color{blue} \mathbf{M}_4 \color{black} - \color{blue} \mathbf{M}_5 \color{black} + \color{blue} \mathbf{M}_7 \\
            \mathbf{C}_{12} & = & \color{blue} \mathbf{M}_3 \color{black} + \color{blue} \mathbf{M}_5 \\
            \mathbf{C}_{21} & = & \color{blue} \mathbf{M}_2 \color{black} + \color{blue} \mathbf{M}_4 \\
            \mathbf{C}_{22} & = & \color{blue} \mathbf{M}_1 \color{black} - \color{blue} \mathbf{M}_2 \color{black} + \color{blue} \mathbf{M}_3 \color{black} + \color{blue} \mathbf{M}_6
        \end{array}
        \label{eq:classic_strassen}
    \end{equation}

    Strassen's algorithm can be rearranged; we can modify the additions and
    multiplies to get another a permuted version of his algorithm with the same
    number of multiplies and additions. An example of these variations of
    Strassen's algorithms can be seen in \ref{eq:per_strassen}. This permuted
    version will be relevant for \Cref{sec:Cyclic Invariance} as it contains a
    special structure that is hard to visualize in the
    \Cref{eq:classic_strassen}. Notice how the two algorithms are the same,
    except that $\mathbf M_3$ became $\mathbf M_6$, $\mathbf M_4$ became
    $\mathbf M_3$, $\mathbf M_6$ became $\mathbf M_7$, and $\mathbf{M}_7$ became
    $\mathbf{M}_4$ and the additions in $\mathbf C$ changed respectively. It is
    explained in \Cref{sec:Cyclic Invariant Matrix Multiplication Algorithms}
    that this is due to a change in the columns of a KTensor decomposition. 

    \begin{center}
        \textbf{Permuted Strassen's Algorithm}
        \vspace{-20pt}
    \end{center}
    \begin{equation}
        \begin{array}{rcl}
            \color{blue} \mathbf{M}_1 & = & (\mathbf{A}_{11} + \mathbf{A}_{22})\cdot (\mathbf{B}_{11} + \mathbf{B}_{22}) \\
            \color{blue} \mathbf{M}_2 & = & (\mathbf{A}_{12} + \mathbf{A}_{22})\cdot \mathbf{B}_{11} \\
            \color{blue} \mathbf{M}_3 & = & \mathbf{A}_{22}\cdot (\mathbf{B}_{12} - \mathbf{B}_{11}) \\
            \color{blue} \mathbf{M}_4 & = & (\mathbf{A}_{21} - \mathbf{A}_{22})\cdot (\mathbf{B}_{12} + \mathbf{B}_{22}) \\
            \color{blue} \mathbf{M}_5 & = & (\mathbf{A}_{11} + \mathbf{A}_{21})\cdot \mathbf{B}_{22} \\
            \color{blue} \mathbf{M}_6 & = & \mathbf{A}_{11}\cdot (\mathbf{B}_{21} - \mathbf{B}_{22}) \\
            \color{blue} \mathbf{M}_7 & = & (\mathbf{A}_{12} - \mathbf{A}_{11})\cdot (\mathbf{B}_{11} + \mathbf{B}_{21}) \\
            \mathbf{C}_{11} & = & \color{blue} \mathbf{M}_1 \color{black} + \color{blue} \mathbf{M}_3 \color{black} - \color{blue} \mathbf{M}_4 \color{black} - \color{blue} \mathbf{M}_6 \\
            \mathbf{C}_{12} & = & \color{blue} \mathbf{M}_2 \color{black} + \color{blue} \mathbf{M}_3 \\
            \mathbf{C}_{21} & = & \color{blue} \mathbf{M}_5 \color{black} + \color{blue} \mathbf{M}_6 \\
            \mathbf{C}_{22} & = & \color{blue} \mathbf{M}_1 \color{black} - \color{blue} \mathbf{M}_2 \color{black} + \color{blue} \mathbf{M}_5 \color{black} + \color{blue} \mathbf{M}_7
        \end{array}
        \label{eq:per_strassen}
    \end{equation}
    
    It is at this point that an important distinction must be made; what is
    considered a new algorithm? Both Strassen's algorithm and its permutation
    are considered Strassen's algorithm because they both involve 7
    multiplications. This is because the term \textit{Strassen's algorithm} has
    become an all-encapsulating term for any 2 by 2 matrix multiplication
    algorithm that uses 7 multiplications instead of the classic 8 (since they
    are all equivalent up to $GL_3$, transpose-like, and cyclic invariant
    permutations). For example, take a look at \Cref{eq:var_strassen}. It is no
    longer just a permuted version of \Cref{eq:classic_strassen} as this onw has
    24 matrix additions as opposed to the original's 18. Again, since matrix
    addition is not the bottleneck, both of these algorithms have the same
    computational complexity. The algorithm below is still called Strassen's
    algorithm for the reasons mentioned previously. Furthermore, this variant
    version will also be relevant in \Cref{sec:Cyclic Invariance} as it contains
    the same srtructure we will explore later on, but slighly different. It has
    been proven that there are no possible fast matrix multiplication algorithms
    with 6 multiplications. 

    \begin{center}
        \textbf{Variant Strassen's Algorithm}
        \vspace{-20pt}
    \end{center}
    \begin{equation}
        \begin{array}{rcl}
            \color{blue} \mathbf{M}_1 & = & \mathbf{A}_{11}\cdot \mathbf{B}_{11} \\
            \color{blue} \mathbf{M}_2 & = & (\mathbf{A}_{12} + \mathbf{A}_{22})\cdot (\mathbf{B}_{12} + \mathbf{B}_{22}) \\
            \color{blue} \mathbf{M}_3 & = & (\mathbf{A}_{22} - \mathbf{A}_{21})\cdot (\mathbf{B}_{22} - \mathbf{B}_{21}) \\
            \color{blue} \mathbf{M}_4 & = & (\mathbf{A}_{21} - \mathbf{A}_{12} - \mathbf{A}_{22})\cdot (\mathbf{B}_{21} - \mathbf{B}_{12} - \mathbf{B}_{22}) \\
            \color{blue} \mathbf{M}_5 & = & (-\mathbf{A}_{12})\cdot (-\mathbf{B}_{21}) \\
            \color{blue} \mathbf{M}_6 & = & (\mathbf{A}_{11} - \mathbf{A}_{12} + \mathbf{A}_{21} - \mathbf{A}_{22})\cdot (- \mathbf{B}_{12}) \\
            \color{blue} \mathbf{M}_7 & = & (- \mathbf{A}_{21})\cdot (\mathbf{B}_{11} - \mathbf{B}_{12} + \mathbf{B}_{21} - \mathbf{B}_{22}) \\
            \mathbf{C}_{11} & = & \color{blue} \mathbf{M}_1 \color{black} + \color{blue} \mathbf{M}_5 \\
            \mathbf{C}_{22} & = & \color{blue} \mathbf{M}_4 \color{black} - \color{blue} \mathbf{M}_3 \color{black} + \color{blue} \mathbf{M}_5 \color{black} + \color{blue} \mathbf{M}_6 \\
            \mathbf{C}_{22} & = & \color{blue} \mathbf{M}_2 \color{black} - \color{blue} \mathbf{M}_4 \color{black} - \color{blue} \mathbf{M}_5 \color{black} + \color{blue} \mathbf{M}_7 \\
            \mathbf{C}_{22} & = & \color{blue} \mathbf{M}_2 \color{black} + \color{blue} \mathbf{M}_3 \color{black} - \color{blue} \mathbf{M}_4 \color{black} + \color{blue} \mathbf{M}_5
        \end{array}
        \label{eq:var_strassen}
    \end{equation}
    

    Thus, a question arises, how many possibilities are there? We could search
    for different solutions that all have the same number of multiplications by
    performing either an eshaustive search or an optimization search on each
    parameter involved in matrix multiplication as seen in
    \ref{eq:serching_fast_matmul}. If we are searching for a discrete solution
    with only -1, 0, 1 as coefficients then we have $3^{84}$ possibilities,
    though not all of these possibilities would be valid algotithms. We will see
    how we can decrease this number later on to something much more feasible
    using the structure hinted at in these algorithms. Before we can move on to
    how to search for these algorithms non-exhaustively, we turn to the
    protagonist of this project.



    \begin{center}
        \textbf{Exhaustive Search of Fast MatMul Algorithms}
        \vspace{-10pt}
    \end{center}
    \begin{equation}
        \scalebox{0.9}{
            $\displaystyle
            \begin{array}{rcl}
                \color{blue} \mathbf{M}_1 & = & (\textcolor{purple}{ u_{11}^{(1)}} \mathbf{A}_{11} + \textcolor{purple}{u_{12}^{(1)}} \mathbf{A}_{12} + \textcolor{purple}{ u_{21}^{(1)}} \mathbf{A}_{21} + \textcolor{purple}{ u_{22}^{(1)}} \mathbf{A}_{22}) \cdot (\textcolor{purple}{ v_{11}^{(1)}} \mathbf{B}_{11} + \textcolor{purple}{v_{12}^{(1)}} \mathbf{B}_{12} + \textcolor{purple}{v_{21}^{(1)}} \mathbf{B}_{21} \textcolor{purple}{ v_{22}^{(1)}} + \mathbf{B}_{22}) \\
                \color{blue} \mathbf{M}_2 & = & (\textcolor{purple}{ u_{11}^{(2)}} \mathbf{A}_{11} + \textcolor{purple}{u_{12}^{(2)}} \mathbf{A}_{12} + \textcolor{purple}{ u_{21}^{(2)}} \mathbf{A}_{21} + \textcolor{purple}{ u_{22}^{(2)}} \mathbf{A}_{22}) \cdot (\textcolor{purple}{ v_{11}^{(2)}} \mathbf{B}_{11} + \textcolor{purple}{v_{12}^{(2)}} \mathbf{B}_{12} + \textcolor{purple}{v_{21}^{(2)}} \mathbf{B}_{21} \textcolor{purple}{ v_{22}^{(2)}} + \mathbf{B}_{22}) \\
                \color{blue} \mathbf{M}_3 & = & (\textcolor{purple}{ u_{11}^{(3)}} \mathbf{A}_{11} + \textcolor{purple}{u_{12}^{(3)}} \mathbf{A}_{12} + \textcolor{purple}{ u_{21}^{(3)}} \mathbf{A}_{21} + \textcolor{purple}{ u_{22}^{(3)}} \mathbf{A}_{22}) \cdot (\textcolor{purple}{ v_{11}^{(3)}} \mathbf{B}_{11} + \textcolor{purple}{v_{12}^{(3)}} \mathbf{B}_{12} + \textcolor{purple}{v_{21}^{(3)}} \mathbf{B}_{21} \textcolor{purple}{ v_{22}^{(3)}} + \mathbf{B}_{22}) \\
                \color{blue} \mathbf{M}_4 & = & (\textcolor{purple}{ u_{11}^{(4)}} \mathbf{A}_{11} + \textcolor{purple}{u_{12}^{(4)}} \mathbf{A}_{12} + \textcolor{purple}{ u_{21}^{(4)}} \mathbf{A}_{21} + \textcolor{purple}{ u_{22}^{(4)}} \mathbf{A}_{22}) \cdot (\textcolor{purple}{ v_{11}^{(4)}} \mathbf{B}_{11} + \textcolor{purple}{v_{12}^{(4)}} \mathbf{B}_{12} + \textcolor{purple}{v_{21}^{(4)}} \mathbf{B}_{21} \textcolor{purple}{ v_{22}^{(4)}} + \mathbf{B}_{22}) \\
                \color{blue} \mathbf{M}_5 & = & (\textcolor{purple}{ u_{11}^{(5)}} \mathbf{A}_{11} + \textcolor{purple}{u_{12}^{(5)}} \mathbf{A}_{12} + \textcolor{purple}{ u_{21}^{(5)}} \mathbf{A}_{21} + \textcolor{purple}{ u_{22}^{(5)}} \mathbf{A}_{22}) \cdot (\textcolor{purple}{ v_{11}^{(5)}} \mathbf{B}_{11} + \textcolor{purple}{v_{12}^{(5)}} \mathbf{B}_{12} + \textcolor{purple}{v_{21}^{(5)}} \mathbf{B}_{21} \textcolor{purple}{ v_{22}^{(5)}} + \mathbf{B}_{22}) \\
                \color{blue} \mathbf{M}_6 & = & (\textcolor{purple}{ u_{11}^{(6)}} \mathbf{A}_{11} + \textcolor{purple}{u_{12}^{(6)}} \mathbf{A}_{12} + \textcolor{purple}{ u_{21}^{(6)}} \mathbf{A}_{21} + \textcolor{purple}{ u_{22}^{(6)}} \mathbf{A}_{22}) \cdot (\textcolor{purple}{ v_{11}^{(6)}} \mathbf{B}_{11} + \textcolor{purple}{v_{12}^{(6)}} \mathbf{B}_{12} + \textcolor{purple}{v_{21}^{(6)}} \mathbf{B}_{21} \textcolor{purple}{ v_{22}^{(6)}} + \mathbf{B}_{22}) \\
                \color{blue} \mathbf{M}_7 & = & (\textcolor{purple}{ u_{11}^{(7)}} \mathbf{A}_{11} + \textcolor{purple}{u_{12}^{(7)}} \mathbf{A}_{12} + \textcolor{purple}{ u_{21}^{(7)}} \mathbf{A}_{21} + \textcolor{purple}{ u_{22}^{(7)}} \mathbf{A}_{22}) \cdot (\textcolor{purple}{ v_{11}^{(7)}} \mathbf{B}_{11} + \textcolor{purple}{v_{12}^{(7)}} \mathbf{B}_{12} + \textcolor{purple}{v_{21}^{(7)}} \mathbf{B}_{21} \textcolor{purple}{ v_{22}^{(7)}} + \mathbf{B}_{22}) \\ \\
                \mathbf{C}_{11} & = & \textcolor{purple}{w_{11}^{(1)}} \color{blue} \mathbf{M}_1 + \textcolor{purple}{w_{11}^{(2)}} \color{blue} \mathbf{M}_2 + \textcolor{purple}{w_{11}^{(3)}} \color{blue} \mathbf{M}_3 + \textcolor{purple}{w_{11}^{(4)}} \color{blue} \mathbf{M}_4 + \textcolor{purple}{w_{11}^{(5)}} \color{blue} \mathbf{M}_5 + \textcolor{purple}{w_{11}^{(6)}} \color{blue} \mathbf{M}_6 + \textcolor{purple}{w_{11}^{(7)}} \color{blue} \mathbf{M}_7 \\
                \mathbf{C}_{12} & = & \textcolor{purple}{w_{12}^{(1)}} \color{blue} \mathbf{M}_1 + \textcolor{purple}{w_{12}^{(2)}} \color{blue} \mathbf{M}_2 + \textcolor{purple}{w_{12}^{(3)}} \color{blue} \mathbf{M}_3 + \textcolor{purple}{w_{12}^{(4)}} \color{blue} \mathbf{M}_4 + \textcolor{purple}{w_{12}^{(5)}} \color{blue} \mathbf{M}_5 + \textcolor{purple}{w_{12}^{(6)}} \color{blue} \mathbf{M}_6 + \textcolor{purple}{w_{12}^{(7)}} \color{blue} \mathbf{M}_7 \\
                \mathbf{C}_{21} & = & \textcolor{purple}{w_{21}^{(1)}} \color{blue} \mathbf{M}_1 + \textcolor{purple}{w_{21}^{(2)}} \color{blue} \mathbf{M}_2 + \textcolor{purple}{w_{21}^{(3)}} \color{blue} \mathbf{M}_3 + \textcolor{purple}{w_{21}^{(4)}} \color{blue} \mathbf{M}_4 + \textcolor{purple}{w_{21}^{(5)}} \color{blue} \mathbf{M}_5 + \textcolor{purple}{w_{21}^{(6)}} \color{blue} \mathbf{M}_6 + \textcolor{purple}{w_{21}^{(7)}} \color{blue} \mathbf{M}_7 \\
                \mathbf{C}_{22} & = & \textcolor{purple}{w_{22}^{(1)}} \color{blue} \mathbf{M}_1 + \textcolor{purple}{w_{22}^{(2)}} \color{blue} \mathbf{M}_2 + \textcolor{purple}{w_{22}^{(3)}} \color{blue} \mathbf{M}_3 + \textcolor{purple}{w_{22}^{(4)}} \color{blue} \mathbf{M}_4 + \textcolor{purple}{w_{22}^{(5)}} \color{blue} \mathbf{M}_5 + \textcolor{purple}{w_{22}^{(6)}} \color{blue} \mathbf{M}_6 + \textcolor{purple}{w_{22}^{(7)}} \color{blue} \mathbf{M}_7 \\
            \end{array}
            $
        }
        \label{eq:serching_fast_matmul}
    \end{equation}



\subsection{The Matrix Multiplication Tensor} \label{sec:The Matrix Multiplication Tensor}

    There is a specific type of tensor that is crucial to this project: the
    Matrix Multiplication Tensor. Matrix multiplication can be performed in
    tensor format using this tensor as visualized in \Cref{fig:matmul_tensor}.
    If we wish to multiply two matrices $\mathbf{A} \in \mathbb{R}^{m\times n}$
    and $\mathbf{B}\in \mathbb{R}^{n\times p}$, we can form the respective
    matrix multiplication tensor $\mathcal{M}\in \mathbb{R}^{mn \times np\times
    mp}$ through \Cref{alg:matmul_tensor}. In order to do so, we vectorize
    $\mathbf{A}$ and $\mathbf{B}$ and multiply them through TTMs in the first
    and second mode respectively. The output, in the third mode, is the
    vectorization of the output $\mathbf{C} = \mathbf{A}\cdot \mathbf{B}$
    transposed (i.e. $\mathbf{C^\intercal}$). This project is concerned only
    with square matrices, thus we always assume $\mathbf{A}, \mathbf{B},
    \mathbf{C}\in \mathbb{R}^{n\times n}$ and
    $\mathcal{M}\in\mathbb{R}^{n^2\times n^2\times n^2}$.

    \begin{algorithm}
        \caption{Forming the Matrix Multiplication Tensor}
        \label{alg:matmul_tensor}
        \begin{algorithmic}[0]
            \Function{$\mathcal{T}$ = MatMul-Tensor}{$m, n, p$}
                \State{$\mathcal{T}$ = \texttt{zeros}($mn, np, mp$)} \Comment{Initialize Tensor}
                \For{$i = 1:m$}
                    \For{$i = 1:m$}
                        \For{$i = 1:m$}
                            \State{$\mathcal{T}(mj + i, nk + j, pi + k) = 1$}
                        \EndFor
                    \EndFor
                \EndFor
            \EndFunction
        \end{algorithmic}
    \end{algorithm}

    \begin{figure}
        \centering
        \begin{tikzpicture}[scale=1,namenode/.style={scale=1}]
    % These distances can be meddled with to get you what you want
    % Within reason...
    \def\ix{2} %
    \def\iy{2} %
    \def\iz{2} %
    \def\r{\iz/4}
    \def\corescale{1.75}
    \def\rx{\ix/\corescale}
    \def\ry{\iy/\corescale}
    \def\rz{\iz/\corescale}
    \def\sp{\rx/6}
    
    % This line chooses the starting location of the object
    \coordinate (MFrontLowerLeft) at (0,0);

    % This draws the front og the cube
    \draw (MFrontLowerLeft) rectangle ++ (\ix,\iy);

    % This scope draws the top face of the cube
    \begin{scope}[shift={(MFrontLowerLeft)}, canvas is zx plane at y=\iy,rotate=90]
        \draw (0,0) rectangle ++ (\ix,\iz);
    \end{scope}

    % This scope draws the left side of the cube
    \begin{scope}[shift={(MFrontLowerLeft)},canvas is zy plane at x=\ix,rotate=90]
        \draw (0,0) rectangle ++ (\iy,\iz); %
    \end{scope}

    % This writes the M in the front face of the cube
    \node[namenode] at ($(MFrontLowerLeft) + (0.5*\ix, 0.5*\iy)$)  {$\mathcal{M}$};

    \coordinate (ALowerLeft) at ($(MFrontLowerLeft) + (-\ix - 0.5, \iy-\r/1.5)$);
    \draw (ALowerLeft) rectangle ++ (\iy, \r/1.25);
    \node[namenode] at ($(ALowerLeft) + (\iy/2, 0.2)$) {$A$};

    \coordinate (BLowerLeft) at ($(MFrontLowerLeft) + (\ix+1.15,0.775)$);
    \draw (BLowerLeft) rectangle ++ (\r/1.25, \iy);
    \node[namenode] at ($(BLowerLeft) + (0.19, \iy/2)$) {$B$};

    \coordinate (L1) at ($(MFrontLowerLeft) + (\ix + \sp, \sp)$);
    \coordinate (L1H) at ($(L1) + (0, \iy)$);
    \draw[thin, densely dotted] (L1) -- (L1H);
    \coordinate (L1) at ($(MFrontLowerLeft) + (\ix + 2*\sp, 2*\sp)$);
    \coordinate (L1H) at ($(L1) + (0, \iy)$);
    \draw[thin, densely dotted] (L1) -- (L1H);
    \coordinate (L1) at ($(MFrontLowerLeft) + (\ix + 3*\sp, 3*\sp)$);
    \coordinate (L1H) at ($(L1) + (0, \iy)$);
    \draw[thin, densely dotted] (L1) -- (L1H);
    
    \coordinate (L1) at ($(MFrontLowerLeft) + (\sp, \iy + \sp)$);
    \coordinate (L1H) at ($(L1) + (\ix, 0)$);
    \draw[thin, densely dotted] (L1) -- (L1H);
    \coordinate (L1) at ($(MFrontLowerLeft) + (2*\sp, \iy + 2*\sp)$);
    \coordinate (L1H) at ($(L1) + (\ix, 0)$);
    \draw[thin, densely dotted] (L1) -- (L1H);
    \coordinate (L1) at ($(MFrontLowerLeft) + (3*\sp, \iy + 3*\sp)$);
    \coordinate (L1H) at ($(L1) + (\ix, 0)$);
    \draw[thin, densely dotted] (L1) -- (L1H);

    \node[namenode] at ($(MFrontLowerLeft) + (3.25+0.5*\ix, 0.5*\iy)$)  {$ = $};

    % This scope draws the top face of the cube
    \coordinate (CLowerLeft) at ($(MFrontLowerLeft) + (4.85, -\iy/3)$);
    \begin{scope}[shift={(CLowerLeft)}, canvas is zx plane at y=\iy,rotate=90]
        \draw (0,0) rectangle ++ (\r/1.25, \ix);
    \end{scope}

    \node[namenode] at ($(CLowerLeft) + (1.25, \iy + 0.3)$) {$C^\text{T}$};
\end{tikzpicture}
        \caption[Matrix Multiplication in Tensor Format]{Matrix Multiplication
        in Tensor Format. The matrices should be interpreted as vectorizations
        of the matrices, for example the label $\mathbf{A}$ should really be
        vec($\mathbf{A}$)}
        \label{fig:matmul_tensor}
    \end{figure}

    It is at this point we return to tensor decompositions, specifically the CP
    decomposition. Recall from \ref{sec:Kruskal Tensors and The CP
    Decomposition} that the decomposition compresses an input tensor into $r$
    $d$-way outer product components. It turns out, that if we decompose the
    matrix multiplication tensor using the CP Decomposition, there is a hidden
    fast matrix multiplication algorithm embedded in the components of the CP
    decomposition. In fact, the number of components $r$ corresponds to the
    number of multiplications in the algorithm. This in return results in two
    different implications (could argue that they mean the same thing in two
    different directions but we seperate these implications for the sake of
    understanding). The first implication is that given an algorithm, say one of
    the $3^84$ for $2\times 2$ algorithms with rank 7, we can form the KTensor
    of the corresponding algorithm, namely $\mathcal{\hat{M}}$ and take the norm
    of the difference from the original matmul tensor. If $\|\mathcal{M -
    \hat{M}}\| = 0$, then we have a valid algorithm, there are easier ways to
    check for the validity of an algorithm, but this helps with forming the
    search equation. Before continuing with the second implication, we must
    understand how to visualize the hidden algorithm in the factor matrices of
    a KTensor. 
    
    Below is the original Strassen's algorithm with two representations, the one
    we have seen before and the KTensor representation. The way to interpret the
    algorithm on the right, is that each horizontal lines separate the factor
    matrices, therefore just like in \Cref{fig:KTensor} the factor matrices
    $\mathbf{A}, \mathbf{B}$ and $\mathbf{C}$ are separated by the horizontal
    lines, the columns of the factor matrices represent the chicken feet
    components. If you pay close attention you can see how the columns
    representing $\mathbf{M}_\ell$ correspond to the representation on the
    right. If a 1 or -1 appear in the right representation then they appear as
    $\mathbf{A}_{ij}$ or $\mathbf{-A}_{ij}$ respectively on the left.

    \vspace{-50pt}
    \begin{multicols}{2}
        \begin{equation*}
            \begin{array}{rcl}
                \color{blue} \mathbf{M}_1 & = & (\mathbf{A}_{11} + \mathbf{A}_{22})\cdot (\mathbf{B}_{11} + \mathbf{B}_{22}) \\
                \color{blue} \mathbf{M}_2 & = & (\mathbf{A}_{12} + \mathbf{A}_{22})\cdot \mathbf{B}_{11} \\
                \color{blue} \mathbf{M}_3 & = & \mathbf{A}_{11}\cdot (\mathbf{B}_{21} - \mathbf{B}_{22}) \\
                \color{blue} \mathbf{M}_4 & = & \mathbf{A}_{22}\cdot (\mathbf{B}_{12} - \mathbf{B}_{11}) \\
                \color{blue} \mathbf{M}_5 & = & (\mathbf{A}_{11} + \mathbf{A}_{21})\cdot \mathbf{B}_{22} \\
                \color{blue} \mathbf{M}_6 & = & (\mathbf{A}_{12} - \mathbf{A}_{11})\cdot (\mathbf{B}_{11} + \mathbf{B}_{21}) \\
                \color{blue} \mathbf{M}_7 & = & (\mathbf{A}_{21} - \mathbf{A}_{22})\cdot (\mathbf{B}_{12} + \mathbf{B}_{22}) \\
                \mathbf{C}_{11} & = & \color{blue} \mathbf{M}_1 \color{black} + \color{blue} \mathbf{M}_4 \color{black} - \color{blue} \mathbf{M}_5 \color{black} + \color{blue} \mathbf{M}_7 \\
                \mathbf{C}_{12} & = & \color{blue} \mathbf{M}_3 \color{black} + \color{blue} \mathbf{M}_5 \\
                \mathbf{C}_{21} & = & \color{blue} \mathbf{M}_2 \color{black} + \color{blue} \mathbf{M}_4 \\
                \mathbf{C}_{22} & = & \color{blue} \mathbf{M}_1 \color{black} - \color{blue} \mathbf{M}_2 \color{black} + \color{blue} \mathbf{M}_3 \color{black} + \color{blue} \mathbf{M}_6
            \end{array}
        \end{equation*}

        \columnbreak

        \setlength{\arraycolsep}{3pt}
        \[\begin{array}{c|ccccccc}
            & \color{blue} \mathbf{M}_1 & \color{blue} \mathbf{M}_2 & \color{blue} \mathbf{M}_3 & \color{blue} \mathbf{M}_4 & \color{blue} \mathbf{M}_5 & \color{blue} \mathbf{M}_6 & \color{blue} \mathbf{M}_7 \\
            \hline
            \mathbf{A}_{11} & 1 & 0 & 1 & 0 & 1 & -1 & 0 \\
            \mathbf{A}_{12} & 0 & 1 & 0 & 0 & 0 & 1 & 0 \\
            \mathbf{A}_{21} & 0 & 0 & 0 & 0 & 1 & 0 & 1 \\
            \mathbf{A}_{22} & 1 & 1 & 0 & 1 & 0 & 0 & -1 \\
            \hline
            \mathbf{B}_{11} & 1 & 1 & 0 & -1 & 0 & 1 & 0 \\
            \mathbf{B}_{12} & 0 & 0 & 0 & 1 & 0 & 0 & 1 \\
            \mathbf{B}_{21} & 0 & 0 & 1 & 0 & 0 & 1 & 0 \\
            \mathbf{B}_{22} & 1 & 0 & -1 & 0 & 1 & 0 & 1 \\
            \hline
            \mathbf{C}_{11} & 1 & 0 & 0 & 1 & -1 & 0 & 1 \\
            \mathbf{C}_{21} & 0 & 0 & 1 & 0 & 1 & 0 & 0 \\
            \mathbf{C}_{12} & 0 & 1 & 0 & 1 & 0 & 0 & 0 \\
            \mathbf{C}_{22} & 1 & -1 & 1 & 0 & 0 & 1 & 0 \\
    \end{array}\]
    \end{multicols}
    
    
    The other implication, is that we can decompose the matmul tensor with
    specified CP rank (as in \Cref{eq:cp_opt}) through an optimization algorithm
    to search for fast matrix multiplication algorithms.






\subsection{Damped Gauss Newton Optimization for CP Decompositions}\label{sec:Damped Gauss Newton Optimization for CP Decompositions}
    
    Recall from \Cref{eq:cp_least_squares} that we perform a CP decomposition
    with the least squares error of our approximation. We will do so using
    optimization methods for reasons that will become clear soon. Optimization
    methods work with vector input, so for our purposes let 

    \begin{equation*}
        \mathbf{v} = \text{vec}
        \left(
        \left[
            \begin{array}{c}
                \mathbf{A} \\
                \mathbf{B} \\
                \mathbf{C}
            \end{array}
        \right]
        \right) = 
        \left[
            \begin{array}{c}
                \text{vec}(\mathbf{A}) \\
                \text{vec}(\mathbf{B}) \\
                \text{vec}(\mathbf{C})
            \end{array}
        \right]
        \in \mathbb{R}^{3nr}
    \end{equation*}

    Furthermore, let $\phi: \mathbb{R}^n \to \mathbb{R}^m$ be the nonlinear
    function. 

    \begin{equation}
        \phi (\mathbf{v}) = \text{vec}(\mathcal{M} - \llbracket \mathbf{A, B, C} \rrbracket)
    \end{equation}

    Thus, \Cref{eq:cp_least_squares} becomes 

    \begin{equation}
        \min f(\mathbf{v}) = \frac{1}{2} \|\phi (\mathbf{v})\|^2
    \end{equation}

    With that out of the way we must choose an optimization method. Gradient is
    too simple, taking one step further we reach Newton's method. Newton's
    method gets the search direction by using Newton's equation
    \[\nabla^2 f(\mathbf{v})\mathbf{d}_k = -\nabla f(\mathbf{v})\] where
    $\mathbf{d}_k$ is our approximation. We dislike the Hessian here because it
    is computationally expensive and difficult to derive mathematicall (perhaps
    not now but for the cylic part). Thus, we prefer the Gauss-Newton method
    which approximates the Hessian with $\mathbf{J^\intercal J}$ where
    $\mathbf{J}: \mathbb{R}^n\to \mathbb{R}^{n\times m}$ is the jacobian of
    $\phi$: \[(\mathbf{J^\intercal J})\mathbf{d}_k = -\nabla f(\mathbf{v})\] But
    even then, the Gauss-Newton matrix $\mathbf{J^\intercal J}$ can be singular,
    we add a damping parameter, $\lambda \mathbf{I}$, to enforce positive
    definiteness. Thus, we reach the damped Gauss-Newton Method:
    \[(\mathbf{J^\intercal J} + \lambda\mathbf{I})\mathbf{d}_k = -\nabla
    f(\mathbf{v})\]. At this point, we have two tasks at hand. Given our
    $\mathbf{v}, f(\mathbf{v})$ and $\phi(\mathbf{v})$, we must derive the
    gradient of $f$ for the right-hand side and the jacobian of $\phi$ for the
    left-hand side (when really we just want to know how to apply it to a
    vector). The gradient is easy. 

    \begin{equation} \label{eq:cp_dgn_gradient}
        \nabla f = \text{vec}
        \left(
        \left[
        \begin{array}{c}
            \nicefrac{\partial f}{\partial \mathbf{A}} \\
            \nicefrac{\partial f}{\partial \mathbf{B}} \\
            \nicefrac{\partial f}{\partial \mathbf{C}}
        \end{array}
        \right]
        \right) = 
        \left[
            \begin{array}{c}
                \nicefrac{\partial f}{\partial \text{vec}(\mathbf{A})} \\
                \nicefrac{\partial f}{\partial \text{vec}(\mathbf{B})} \\
                \nicefrac{\partial f}{\partial \text{vec}(\mathbf{C})}
            \end{array}
        \right]
        \in \mathbb{R}^{3n^4}
    \end{equation}

    Where each partial derivative is defined as:

    \begin{equation}
        \begin{array}{rcl}
            \nicefrac{\partial f}{\partial \mathbf{A}} & = & -\mathbf{M}_{(1)}(\mathbf{C\odot B}) + \mathbf{A(C^\intercal C\ast B^\intercal B)}\in \mathbb{R}^{n^2\times n^2} \\ 
            \nicefrac{\partial f}{\partial \mathbf{B}} & = & -\mathbf{M}_{(2)}(\mathbf{C\odot A}) + \mathbf{B(C^\intercal C\ast A^\intercal A)}\in \mathbb{R}^{n^2\times n^2} \\ 
            \nicefrac{\partial f}{\partial \mathbf{C}} & = & -\mathbf{M}_{(3)}(\mathbf{B\odot A}) + \mathbf{C(B^\intercal B\ast A^\intercal A)}\in \mathbb{R}^{n^2\times n^2} 
        \end{array}
        \label{eq:cp_dgn_partials}
    \end{equation}
    
    The Jacobian is not so easy. 
    
    \begin{equation} \label{eq:cp_dgn_jacobian}
        \mathbf{J} = [\mathbf{J}_\mathbf{A} + \mathbf{J}_\mathbf{B} + \mathbf{J}_\mathbf{C}] \in \mathbb{R}^{n^3\times 3nr}
    \end{equation}
    
    Where
    
    \begin{equation}
        \begin{array}{rcl}
            \mathbf{J}_\mathbf{A} \equiv \frac{\partial \phi}{\partial \text{vec}(\mathbf{A})} = (\mathbf{C} \odot \mathbf{B}) \otimes \mathbf{I} \in \mathbb{R}^{n^3\times nr} \\
            \mathbf{J}_\mathbf{B} \equiv \frac{\partial \phi}{\partial \text{vec}(\mathbf{B})} = \mathbf{\Pi}_2^\intercal (\mathbf{C} \odot \mathbf{A}) \otimes \mathbf{I} \in \mathbb{R}^{n^3\times nr} \\
            \mathbf{J}_\mathbf{C} \equiv \frac{\partial \phi}{\partial \text{vec}(\mathbf{C})} = \mathbf{\Pi}_3^\intercal(\mathbf{B} \odot \mathbf{A}) \otimes \mathbf{I} \in \mathbb{R}^{n^3\times nr}
        \end{array}
        \label{eq:cp_dgn_jacobians}
    \end{equation}

    such that that $\Pi_k$ is the tensor perfect shuffle matrix, such that
    vec($\mathcal{X}$) = $\Pi_k$ vec($\mathbf{X}_{(k)}$), and $\Pi_1$ is not
    written explicitly because it is the identity matrix. We consider fast
    computation of $\mathbf{J^\intercal J} + \lambda \mathbf{I}$. Rather than
    forming $\mathbf{J}$ or $\mathbf{J^\intercal J}$ explicitly, we use the
    structure of these matrices to compute the matrix-vector product without
    computing any explicit Kronecker or Khatri-Rao products. From
    \Cref{eq:cp_dgn_jacobian} we have that the block structure of
    $\mathbf{J^\intercal J}$ is 

    \begin{equation*}
        \mathbf{J^\intercal J} = 
        \left[
        \begin{array}{ccc}
            \mathbf{J_A^\intercal J_A} & \mathbf{J_A^\intercal J_B} & \mathbf{J_A^\intercal J_C} \\ 
            \mathbf{J_B^\intercal J_A} & \mathbf{J_B^\intercal J_B} & \mathbf{J_B^\intercal J_C} \\ 
            \mathbf{J_C^\intercal J_A} & \mathbf{J_C^\intercal J_B} & \mathbf{J_C^\intercal J_C} 
        \end{array}
        \right]
    \end{equation*}
    
    then
    
    \begin{equation*}
        \mathbf{J^\intercal Jv} = 
        \left[
        \begin{array}{c}
            \mathbf{J_A^\intercal J_A\text{vec}(A)} + \mathbf{J_A^\intercal J_B\text{vec}(B)} + \mathbf{J_A^\intercal J_C\text{vec}(C)} \\ 
            \mathbf{J_B^\intercal J_A\text{vec}(A)} + \mathbf{J_B^\intercal J_B\text{vec}(B)} + \mathbf{J_B^\intercal J_C\text{vec}(C)} \\ 
            \mathbf{J_C^\intercal J_A\text{vec}(A)} + \mathbf{J_C^\intercal J_B\text{vec}(B)} + \mathbf{J_C^\intercal J_C\text{vec}(C)} 
        \end{array}
        \right]
    \end{equation*}
    
    Which actually becomes
    
    \begin{equation*}
        \mathbf{J^\intercal Jv} = 
        \left[
        \begin{array}{c}
            \text{vec}(\mathbf{\bar{A}(B^\intercal B \ast C^\intercal C) + A(\bar{B}^\intercal B\ast C^\intercal C) + A(B^\intercal B \ast \bar{C}^\intercal C)}) \\
            \text{vec}(\mathbf{B(\bar{A}^\intercal A \ast C^\intercal C) + \bar{B}(A^\intercal A\ast C^\intercal C) + B(A^\intercal A \ast \bar{C}^\intercal C)}) \\
            \text{vec}(\mathbf{C(\bar{A}^\intercal A \ast B^\intercal B) + C(A^\intercal A\ast \bar{B}^\intercal B) + \bar{C}(A^\intercal A \ast B^\intercal B)}) \\
        \end{array}
        \right]
    \end{equation*}


    \begin{algorithm}
        \caption{Damped Gauss-Newton On The Matrix Multiplication Tensor}
        \label{alg:CP-DGN}
        \begin{algorithmic}
            \State{\textbf{Input:} Matrix Multiplication Tensor $\mathcal{M}$,}
            \State{\hspace{3.25em} CP Tensor Rank $r$,}
            \State{\hspace{3.25em} Damping Parameter $\lambda \in \mathbb{R}^+$,}
            \State{\hspace{3.25em} Convergence Tolerance $\epsilon > 0$}
            
            \State \textbf{Output:} CP Tensor $\mathcal{K}$
            \Function{DGN}{$\mathcal{M}, r, \lambda, \epsilon$}
                \State{Initialize K and K\_prev to be a cell of length 3 of $n^2 \times r$ matrices}

                \For{$i = 1:MaxIters$}
                    \State{f $ \longleftarrow \frac{1}{2} \|\mathcal{M - K}\| $} \Comment{Compute Function Value}
                    \State{$\nabla \mathbf{f} \longleftarrow [\text{vec} \left( \frac{\partial f}{\partial \mathbf A}\right) \text{vec} \left( \frac{\partial f}{\partial \mathbf B}\right) \text{vec} \left( \frac{\partial f}{\partial \mathbf C}\right)]^\intercal$} \Comment{Compute Gradient}
                    \State{$S \longleftarrow \text{Solution to} (\textbf{J}^T\textbf{J} + \lambda I)K = - \nabla \mathbf{f}$}
                    \While{\textit{Goldstein Conditions Are Satisfied}}
                        \State{K $\longleftarrow$ K\_prev + $\alpha S$}
                        \State{f$_{new} \longleftarrow \frac{1}{2} \|\mathcal{M - K}\|$}
                        \State{$\alpha \longleftarrow \alpha / 2$}
                    \EndWhile{}
                    \If{f - f$_{new} < \epsilon$}
                        \State{\textbf{break}}
                    \EndIf{}
                \EndFor{}
            \EndFunction
        \end{algorithmic}
    \end{algorithm}



\section{Cyclic Invariance} \label{sec:Cyclic Invariance}
    %!TEX root = ../../main.tex

We can reduce our search space in our search for fast matrix multiplication
algorithms by levevering \textbf{cyclic invariance}. Cyclic invariance is an
added structure in matrix multiplication algorithms that reduce the number of
variables of the CP Decomposition optimization problem for the matrix
multiplication tensor. 

\subsection{Cyclic Invariant Matrix Multiplication Algorithms} \label{sec:Cyclic Invariant Matrix Multiplication Algorithms} 
    
    Recall that we can rearrange the multiplications and additions of Strassen's
    agorithms to obtain variantions. Some of these variantions can be cyclic
    invariant. Below is one of these Strassen's variant algorithm next to the
    original. Both of these algorithms have already been introduced back in
    \Cref{sec:Fast Matrix Multiplication Algorithms}. Notive how the variant
    Strassen's algorithm on the left is composed of smaller submatrices that
    appear throughout the main factor matrices of the KTensor. The $4\times 1$
    matrix in red is called the \textbf{symmetric component}, and it always
    appears at the beginning of all three factor matrices, we denote it as
    $\mathbf{S}$. The remaining three $4\times 2$ submatrices are called the
    \textbf{cyclic component}, we denote them as $\mathbf{U, V, W}$. 

    \pagebreak
    \vspace{-50pt}
    \begin{multicols}{2}
        \setlength{\arraycolsep}{3pt}
        \[\begin{array}{c|ccccccc}
                & \color{blue} \mathbf{M}_1 & \color{blue} \mathbf{M}_2 & \color{blue} \mathbf{M}_3 & \color{blue} \mathbf{M}_4 & \color{blue} \mathbf{M}_5 & \color{blue} \mathbf{M}_6 & \color{blue} \mathbf{M}_7 \\
                \hline
                \mathbf{A}_{11} & 1 & 0 & 1 & 0 & 1 & -1 & 0 \\
                \mathbf{A}_{12} & 0 & 1 & 0 & 0 & 0 & 1 & 0 \\
                \mathbf{A}_{21} & 0 & 0 & 0 & 0 & 1 & 0 & 1 \\
                \mathbf{A}_{22} & 1 & 1 & 0 & 1 & 0 & 0 & -1 \\
                \hline
                \mathbf{B}_{11} & 1 & 1 & 0 & -1 & 0 & 1 & 0 \\
                \mathbf{B}_{12} & 0 & 0 & 0 & 1 & 0 & 0 & 1 \\
                \mathbf{B}_{21} & 0 & 0 & 1 & 0 & 0 & 1 & 0 \\
                \mathbf{B}_{22} & 1 & 0 & -1 & 0 & 1 & 0 & 1 \\
                \hline
                \mathbf{C}_{11} & 1 & 0 & 0 & 1 & -1 & 0 & 1 \\
                \mathbf{C}_{21} & 0 & 0 & 1 & 0 & 1 & 0 & 0 \\
                \mathbf{C}_{12} & 0 & 1 & 0 & 1 & 0 & 0 & 0 \\
                \mathbf{C}_{22} & 1 & -1 & 1 & 0 & 0 & 1 & 0 \\
        \end{array}\]

        \columnbreak

        \setlength{\arraycolsep}{3pt}
        \[\begin{array}{c|ccccccc}
                & \color{blue} \mathbf{M}_1 & \color{blue} \mathbf{M}_2 & \color{blue} \mathbf{M}_3 & \color{blue} \mathbf{M}_4 & \color{blue} \mathbf{M}_5 & \color{blue} \mathbf{M}_6 & \color{blue} \mathbf{M}_7 \\
                \hline
                \mathbf{A}_{11} & \color{red} 1 & \color{mycustomgreen} 0 & \color{mycustomgreen} 0 & \color{violet} 0 & \color{violet} 1 & \color{orange} 1 & \color{orange} -1 \\
                \mathbf{A}_{12} & \color{red} 0 & \color{mycustomgreen} 1 & \color{mycustomgreen} 0 & \color{violet} 0 & \color{violet} 0 & \color{orange} 0 & \color{orange} 1 \\
                \mathbf{A}_{21} & \color{red} 0 & \color{mycustomgreen} 0 & \color{mycustomgreen} 0 & \color{violet} 1 & \color{violet} 1 & \color{orange} 0 & \color{orange} 0 \\
                \mathbf{A}_{22} & \color{red} 1 & \color{mycustomgreen} 1 & \color{mycustomgreen} 1 & \color{violet} -1 & \color{violet} 0 & \color{orange} 0 & \color{orange} 0 \\
                \hline
                \mathbf{B}_{11} & \color{red} 1 & \color{orange} 1 & \color{orange} -1 & \color{mycustomgreen} 0 & \color{mycustomgreen} 0 & \color{violet} 0 & \color{violet} 1 \\
                \mathbf{B}_{12} & \color{red} 0 & \color{orange} 0 & \color{orange} 1 & \color{mycustomgreen} 1 & \color{mycustomgreen} 0 & \color{violet} 0 & \color{violet} 0 \\
                \mathbf{B}_{21} & \color{red} 0 & \color{orange} 0 & \color{orange} 0 & \color{mycustomgreen} 0 & \color{mycustomgreen} 0 & \color{violet} 1 & \color{violet} 1 \\
                \mathbf{B}_{22} & \color{red} 1 & \color{orange} 0 & \color{orange} 0 & \color{mycustomgreen} 1 & \color{mycustomgreen} 1 & \color{violet} -1 & \color{violet} 0 \\
                \hline
                \mathbf{C}_{11} & \color{red} 1 & \color{violet} 0 & \color{violet} 1 & \color{orange} 1 & \color{orange} -1 & \color{mycustomgreen} 0 & \color{mycustomgreen} 0 \\
                \mathbf{C}_{21} & \color{red} 0 & \color{violet} 0 & \color{violet} 0 & \color{orange} 0 & \color{orange} 1 & \color{mycustomgreen} 1 & \color{mycustomgreen} 0 \\
                \mathbf{C}_{12} & \color{red} 0 & \color{violet} 1 & \color{violet} 1 & \color{orange} 0 & \color{orange} 0 & \color{mycustomgreen} 0 & \color{mycustomgreen} 0 \\
                \mathbf{C}_{22} & \color{red} 1 & \color{violet} -1 & \color{violet} 0 & \color{orange} 0 & \color{orange} 0 & \color{mycustomgreen} 1 & \color{mycustomgreen} 1 \\
        \end{array}\]
    \end{multicols}

    Because they are always submatrices of the factor matrices, both the
    symmetric and the cyclic components have the same number of rows as the
    factor matrices, namely $n$. However, they can have a different number of
    columns. We denote the number of columns of the symmetric component as $r_s$
    and the number of columns of the cyclic component as $r_c$. Since $r$ is the
    rank of the CP Decomposition, we have that $r_s + 3r_c = r$. Because of
    this, given a matrix multiplication tensor of $n\times n$ matrices, and a
    given rank $r$, there are multiple choices for $r_c$, which in turn define
    the value of $r_s$ since $r_s = r - 3r_c$. For rank 7 algorithms of $2\times
    2$ matrices, we have two options $[r_s = 1, r_c = 2]$ and  $[r_s = 4, r_c =
    1]$. Below are two different Strassen's algorithm, on the left is the same
    algorithm we saw above of $r_s = 1$, and on the right is an $r_s = 4$
    algorithm. Besides the difference in $r_s$, these algorithms share another
    distinction, which is the number of matrix additions. The algorithm on the
    left has 18 additions while the one on the left has 24 additions, thus
    making it slighly less optimal. Generally we disregard this distinction as
    matrix addition is not the bottleneck computation. 

%     \pagebreak
    \vspace{-50pt}
    \begin{multicols}{2}
        \setlength{\arraycolsep}{3pt}
        \[\begin{array}{c|ccccccc}
                & \color{blue} \mathbf{M}_1 & \color{blue} \mathbf{M}_2 & \color{blue} \mathbf{M}_3 & \color{blue} \mathbf{M}_4 & \color{blue} \mathbf{M}_5 & \color{blue} \mathbf{M}_6 & \color{blue} \mathbf{M}_7 \\
                \hline
                \mathbf{A}_{11} & \color{red} 1 & \color{mycustomgreen} 0 & \color{mycustomgreen} 0 & \color{violet} 0 & \color{violet} 1 & \color{orange} 1 & \color{orange} -1 \\
                \mathbf{A}_{12} & \color{red} 0 & \color{mycustomgreen} 1 & \color{mycustomgreen} 0 & \color{violet} 0 & \color{violet} 0 & \color{orange} 0 & \color{orange} 1 \\
                \mathbf{A}_{21} & \color{red} 0 & \color{mycustomgreen} 0 & \color{mycustomgreen} 0 & \color{violet} 1 & \color{violet} 1 & \color{orange} 0 & \color{orange} 0 \\
                \mathbf{A}_{22} & \color{red} 1 & \color{mycustomgreen} 1 & \color{mycustomgreen} 1 & \color{violet} -1 & \color{violet} 0 & \color{orange} 0 & \color{orange} 0 \\
                \hline
                \mathbf{B}_{11} & \color{red} 1 & \color{orange} 1 & \color{orange} -1 & \color{mycustomgreen} 0 & \color{mycustomgreen} 0 & \color{violet} 0 & \color{violet} 1 \\
                \mathbf{B}_{12} & \color{red} 0 & \color{orange} 0 & \color{orange} 1 & \color{mycustomgreen} 1 & \color{mycustomgreen} 0 & \color{violet} 0 & \color{violet} 0 \\
                \mathbf{B}_{21} & \color{red} 0 & \color{orange} 0 & \color{orange} 0 & \color{mycustomgreen} 0 & \color{mycustomgreen} 0 & \color{violet} 1 & \color{violet} 1 \\
                \mathbf{B}_{22} & \color{red} 1 & \color{orange} 0 & \color{orange} 0 & \color{mycustomgreen} 1 & \color{mycustomgreen} 1 & \color{violet} -1 & \color{violet} 0 \\
                \hline
                \mathbf{C}_{11} & \color{red} 1 & \color{violet} 0 & \color{violet} 1 & \color{orange} 1 & \color{orange} -1 & \color{mycustomgreen} 0 & \color{mycustomgreen} 0 \\
                \mathbf{C}_{21} & \color{red} 0 & \color{violet} 0 & \color{violet} 0 & \color{orange} 0 & \color{orange} 1 & \color{mycustomgreen} 1 & \color{mycustomgreen} 0 \\
                \mathbf{C}_{12} & \color{red} 0 & \color{violet} 1 & \color{violet} 1 & \color{orange} 0 & \color{orange} 0 & \color{mycustomgreen} 0 & \color{mycustomgreen} 0 \\
                \mathbf{C}_{22} & \color{red} 1 & \color{violet} -1 & \color{violet} 0 & \color{orange} 0 & \color{orange} 0 & \color{mycustomgreen} 1 & \color{mycustomgreen} 1 \\
        \end{array}\]
    
        \columnbreak
        
        \setlength{\arraycolsep}{3pt}
        \[\begin{array}{c|ccccccc}
                & \color{blue} \mathbf{M}_1 & \color{blue} \mathbf{M}_2 & \color{blue} \mathbf{M}_3 & \color{blue} \mathbf{M}_4 & \color{blue} \mathbf{M}_5 & \color{blue} \mathbf{M}_6 & \color{blue} \mathbf{M}_7 \\
                \hline
                \mathbf{A}_{11} & \color{red} 1 & \color{red} 0 & \color{red} 0 & \color{red} 0  & \color{mycustomgreen} 0 &  \color{violet} 1 &  \color{orange} 0 \\
                \mathbf{A}_{12} & \color{red} 0 & \color{red} 0 & \color{red} -1 & \color{red} 1 & \color{mycustomgreen} 0 &  \color{violet} 1 &  \color{orange} -1 \\
                \mathbf{A}_{21} & \color{red} 0 & \color{red} 1 & \color{red} 0 & \color{red} -1 & \color{mycustomgreen} -1 & \color{violet} -1 & \color{orange} 0 \\
                \mathbf{A}_{22} & \color{red} 0 & \color{red} 1 & \color{red} 1 & \color{red} -1 & \color{mycustomgreen} 0 &  \color{violet} -1 & \color{orange} 0 \\
                \hline
                \mathbf{B}_{11} & \color{red} 1 & \color{red} 0 & \color{red} 0 & \color{red} 0  & \color{orange} 0 &  \color{mycustomgreen} 0 &  \color{violet} 1 \\
                \mathbf{B}_{12} & \color{red} 0 & \color{red} 0 & \color{red} -1 & \color{red} 1 & \color{orange} -1 & \color{mycustomgreen} 0 &  \color{violet} 1 \\
                \mathbf{B}_{21} & \color{red} 0 & \color{red} 1 & \color{red} 0 & \color{red} -1 & \color{orange} 0 &  \color{mycustomgreen} -1 & \color{violet} -1 \\
                \mathbf{B}_{22} & \color{red} 0 & \color{red} 1 & \color{red} 1 & \color{red} -1 & \color{orange} 0 &  \color{mycustomgreen} 0 &  \color{violet} -1 \\
                \hline
                \mathbf{C}_{11} & \color{red} 1 & \color{red} 0 & \color{red} 0 & \color{red} 0  & \color{violet} 1 &  \color{orange} 0 &  \color{mycustomgreen} 0 \\
                \mathbf{C}_{21} & \color{red} 0 & \color{red} 0 & \color{red} -1 & \color{red} 1 & \color{violet} 1 &  \color{orange} -1 & \color{mycustomgreen} 0 \\
                \mathbf{C}_{12} & \color{red} 0 & \color{red} 1 & \color{red} 0 & \color{red} -1 & \color{violet} -1 & \color{orange} 0 &  \color{mycustomgreen} -1 \\
                \mathbf{C}_{22} & \color{red} 0 & \color{red} 1 & \color{red} 1 & \color{red} -1 & \color{violet} -1 & \color{orange} 0 &  \color{mycustomgreen} 0 \\
        \end{array}\]    
    \end{multicols}

% The following two are the previous two in format number two
% \begin{multicols}{2}
%         \textbf{Variant of Strassen's Algorithm (Rs=1 Rc=2)}
%         \begin{eqnarray*}
%                 \color{blue} M_1 & = & (A_{11} + A_{22})\cdot (B_{11} + B_{22}) \\
%                 \color{blue} M_2 & = & (A_{12} + A_{22})\cdot B_{11} \\
%                 \color{blue} M_3 & = & A_{22}\cdot (B_{12} - B_{11}) \\
%                 \color{blue} M_4 & = & (A_{21} - A_{22})\cdot (B_{12} + B_{22}) \\
%                 \color{blue} M_5 & = & (A_{11} + A_{21})\cdot B_{22} \\
%                 \color{blue} M_6 & = & A_{11}\cdot (B_{21} - B_{22}) \\
%                 \color{blue} M_7 & = & (A_{12} - A_{11})\cdot (B_{11} + B_{21}) \\
%                 C_{11} & = & \color{blue} M_1 \color{black} + \color{blue} M_3 \color{black} - \color{blue} M_4 \color{black} - \color{blue} M_6 \\
%                 C_{12} & = & \color{blue} M_2 \color{black} + \color{blue} M_3 \\
%                 C_{21} & = & \color{blue} M_5 \color{black} + \color{blue} M_6 \\
%                 C_{22} & = & \color{blue} M_1 \color{black} - \color{blue} M_2 \color{black} + \color{blue} M_5 \color{black} + \color{blue} M_7
%         \end{eqnarray*}

%         \columnbreak

%         \textbf{Variant of Strassen's Algorithm (Rs=4 Rc=1)}
%         \begin{eqnarray*}
%                 \color{blue} M_1 & = & A_{11}\cdot B_{11} \\
%                 \color{blue} M_2 & = & (A_{12} + A_{22})\cdot (B_{12} + B_{22}) \\
%                 \color{blue} M_3 & = & (A_{22} - A_{21})\cdot (B_{22} - B_{21}) \\
%                 \color{blue} M_4 & = & (A_{21} - A_{12} - A_{22})\cdot (B_{21} - B_{12} - B_{22}) \\
%                 \color{blue} M_5 & = & (-A_{12})\cdot (-B_{21}) \\
%                 \color{blue} M_6 & = & (A_{11} - A_{12} + A_{21} - A_{22})\cdot (- B_{12}) \\
%                 \color{blue} M_7 & = & (- A_{21})\cdot (B_{11} - B_{12} + B_{21} - B_{22}) \\
%                 C_{11} & = & \color{blue} M_1 \color{black} + \color{blue} M_5 \\
%                 C_{22} & = & \color{blue} M_4 \color{black} - \color{blue} M_3 \color{black} + \color{blue} M_5 \color{black} + \color{blue} M_6 \\
%                 C_{22} & = & \color{blue} M_2 \color{black} - \color{blue} M_4 \color{black} - \color{blue} M_5 \color{black} + \color{blue} M_7 \\
%                 C_{22} & = & \color{blue} M_2 \color{black} + \color{blue} M_3 \color{black} - \color{blue} M_4 \color{black} + \color{blue} M_5
%         \end{eqnarray*}
% \end{multicols}

Recall that our factor matrices correspond to the components of the KTensor as
seen in \Cref{fig:KTensor_factor_matrices}. The same factor matrices with ths
cyclic invariant structure imposed on them now look like
\Cref{fig:cyc_inv_matrices}. That means we can now adjust \Cref{fig:KTensor} to
look like \Cref{fig:cyc_inv_cp}.

\begin{figure}
    \centering
    \begin{tikzpicture}[scale=1.5,namenode/.style={scale=1}]
    \def\ix{2} % This modifies the width of the objects
    \def\iy{1.5} % This modifies the height of the objects
    \def\is{1} %
    \def\h{.75}
    \def\corescale{1.75}
    \def\rx{\ix/\corescale}
    \def\ry{\iy/\corescale}
    \def\rz{\is/\corescale}
    
    \coordinate (MiddleS) at (0, 0);
    \draw[fill=red] (MiddleS) rectangle ++ (\is,\iy);
    \node[namenode] at ($(MiddleS) + (0.5*\is, 0.5*\iy)$)  {$\mathbf{S}$};
    \node[namenode] at ($(MiddleS) + (-0.5*\is, 0.5*\iy)$)  {$\mathbf{B}$};
    
    \coordinate (MiddleW) at ($(MiddleS) + (\is,0)$);
    \draw[fill=mycustomorange] (MiddleW) rectangle ++ (\ix,\iy);
    \node[namenode] at ($(MiddleW) + (0.5*\ix, 0.5*\iy)$)  {$\mathbf{W}$};

    \coordinate (MiddleU) at ($(MiddleW) + (\ix,0)$);
    \draw[fill=mycustomgreen] (MiddleU) rectangle ++ (\ix,\iy);
    \node[namenode] at ($(MiddleU) + (0.5*\ix, 0.5*\iy)$)  {$\mathbf{U}$};
    
    \coordinate (MiddleV) at ($(MiddleU) + (\ix,0)$);
    \draw[fill=mycustompurple] (MiddleV) rectangle ++ (\ix,\iy);
    \node[namenode] at ($(MiddleV) + (0.5*\ix, 0.5*\iy)$)  {$\mathbf{V}$};
    \node[namenode] at ($(MiddleV) + (1.25*\ix, 0.5*\iy)$)  {$n^2$};


    \coordinate (TopS) at ($(MiddleS) + (0, \iy + \h)$);
    \draw[fill=red] (TopS) rectangle ++ (\is,\iy);
    \node[namenode] at ($(TopS) + (0.5*\is, 0.5*\iy)$)  {$\mathbf{S}$};
    \node[namenode] at ($(TopS) + (-0.5*\is, 0.5*\iy)$)  {$\mathbf{A}$};

    \coordinate (TopU) at ($(TopS) + (\is,0)$);
    \draw[fill=mycustomgreen] (TopU) rectangle ++ (\ix,\iy);
    \node[namenode] at ($(TopU) + (0.5*\ix, 0.5*\iy)$)  {$\mathbf{U}$};

    \coordinate (TopV) at ($(TopU) + (\ix,0)$);
    \draw[fill=mycustompurple] (TopV) rectangle ++ (\ix,\iy);
    \node[namenode] at ($(TopV) + (0.5*\ix, 0.5*\iy)$)  {$\mathbf{V}$};
    
    \coordinate (TopW) at ($(TopV) + (\ix,0)$);
    \draw[fill=mycustomorange] (TopW) rectangle ++ (\ix,\iy);
    \node[namenode] at ($(TopW) + (0.5*\ix, 0.5*\iy)$)  {$\mathbf{W}$};
    \node[namenode] at ($(TopW) + (1.25*\ix, 0.5*\iy)$)  {$n^2$};


    \coordinate (BottomS) at ($(MiddleS) - (0, \iy + \h)$);
    \draw[fill=red] (BottomS) rectangle ++ (\is,\iy);
    \node[namenode] at ($(BottomS) + (0.5*\is, 0.5*\iy)$)  {$\mathbf{S}$};
    \node[namenode] at ($(BottomS) + (-0.5*\is, 0.5*\iy)$)  {$\mathbf{C}$};
    \node[namenode] at ($(BottomS) + (0.5*\is, -0.25*\iy)$)  {$r_s$};

    \coordinate (BottomV) at ($(BottomS) + (\is,0)$);
    \draw[fill=mycustompurple] (BottomV) rectangle ++ (\ix,\iy);
    \node[namenode] at ($(BottomV) + (0.5*\ix, 0.5*\iy)$)  {$\mathbf{V}$};
    \node[namenode] at ($(BottomV) + (0.5*\ix, -0.25*\iy)$)  {$r_c$};

    \coordinate (BottomW) at ($(BottomV) + (\ix,0)$);
    \draw[fill=mycustomorange] (BottomW) rectangle ++ (\ix,\iy);
    \node[namenode] at ($(BottomW) + (0.5*\ix, 0.5*\iy)$)  {$\mathbf{W}$};
    \node[namenode] at ($(BottomW) + (0.5*\ix, -0.25*\iy)$)  {$r_c$};
    \node[namenode] at ($(\is/2 + 3*\ix/2, -2.15*\iy)$)  {$r = r_s + 3r_c$};
    
    \coordinate (BottomU) at ($(BottomW) + (\ix,0)$);
    \draw[fill=mycustomgreen] (BottomU) rectangle ++ (\ix,\iy);
    \node[namenode] at ($(BottomU) + (0.5*\ix, 0.5*\iy)$)  {$\mathbf{U}$};
    \node[namenode] at ($(BottomU) + (0.5*\ix, -0.25*\iy)$)  {$r_c$};
    \node[namenode] at ($(BottomU) + (1.25*\ix, 0.5*\iy)$)  {$n^2$};
\end{tikzpicture}
    \caption{Cyclic Invariance in a KTensor}
    \label{fig:cyc_inv_matrices}
\end{figure}

\begin{figure}
    \centering
    \begin{tikzpicture}[scale=0.75,namenode/.style={scale=0.75}]
    \def\ix{2} % This modifies the height of the objects
    \def\iy{2} % This modifies the width of the objects
    \def\iz{2.5} %
    \def\r{.25}

    %%%%%%%%%%%%%%%
    %%%    S    %%%
    %%%%%%%%%%%%%%%

    %%%%%%%%% 1st S Component %%%%%%%%%
    % This draws the vertical chicken feet
    \coordinate (S1VLowerLeft) at (0,0);
    \draw[fill=red] (S1VLowerLeft) rectangle ++ (\r,\iy);
    \node[namenode] at ($(S1VLowerLeft)-(0,2*\r)$) {$\mathbf{S}_1$};

    % This draws the horizontal chicken feet
    \coordinate (S1HLowerLeft) at ($(S1VLowerLeft) + (\r,\iy) + (\r,\r)$);
    \draw[fill=red] (S1HLowerLeft) rectangle ++ (\ix,\r);
    \node[namenode] at ($(S1HLowerLeft)+(.5*\iy,-2*\r)$) {$\mathbf{S}_1$};

    % This scopee draws the tp-screen chicken feet
    \begin{scope}[shift={(0,2*\r)},canvas is zx plane at y=\iy,rotate=90]
        \draw[fill=red] (0,\r) rectangle ++ (\r,\iz);
    \end{scope}
    \node[namenode] at ($(S1HLowerLeft)+(.6*\ix,1)$) {$\mathbf{S}_1$};
    
    % S ellipsis
    \coordinate (SEllipsis) at (2*\iy,.75*\ix);
    \node[namenode] at (SEllipsis) {$+ \quad \cdots \quad +$};
    
    %%%%%%%%% last S Component %%%%%%%%%
    % This draws the vertical chicken feet
    \coordinate (SRVLowerLeft) at ($(SEllipsis) + (\iy,-.75*\ix)$);
    \draw[fill=red] (SRVLowerLeft) rectangle ++ (\r,\iy);
    \node[namenode] at ($(SRVLowerLeft)-(0,2*\r)$) {$\mathbf{S}_{r_s}$};

    % This draws the horizontal chicken feet
    \coordinate (SRHLowerLeft) at ($(SRVLowerLeft) + (\r,\iy) + (\r,\r)$);
    \draw[fill=red] (SRHLowerLeft) rectangle ++ (\ix,\r);
    \node[namenode] at ($(SRHLowerLeft)+(.5*\iy,-2*\r)$) {$\mathbf{S}_{r_s}$};
    
        % This scopee draws the tp-screen chicken feet
    \begin{scope}[shift={(3*\iy,2*\r)},canvas is zx plane at y=\iy,rotate=90]
        \draw[fill=red] (0,\r) rectangle ++ (\r,\iz);
    \end{scope}
    \node[namenode] at ($(SRHLowerLeft)+(.6*\ix,1)$) {$\mathbf{S}_{r_s}$};

    %%%%%%%%%%%%%%%%
    %%% Cyclic 1 %%%
    %%%%%%%%%%%%%%%%
    
    % top sum
    \coordinate (TopSum) at (4.5*\iy,.75*\ix);
    \node[namenode] at (TopSum) {$+ $};

    %%%%%%%%% 1st Cyclic 1 Component %%%%%%%%%
    % This draws the vertical chicken feet
    \coordinate (U1VLowerLeft) at ($(TopSum) - (0,0.75*\iy) + (0.75,0)$);
    \draw[fill=mycustomgreen] (U1VLowerLeft) rectangle ++ (\r,\iy);
    \node[namenode] at ($(U1VLowerLeft)-(0,2*\r)$) {$\mathbf{U}_1$};

    % This draws the horizontal chicken feet
    \coordinate (W1HLowerLeft) at ($(U1VLowerLeft) + (\r,\iy) + (\r,\r)$);
    \draw[fill=mycustomorange] (W1HLowerLeft) rectangle ++ (\ix,\r);
    \node[namenode] at ($(W1HLowerLeft)+(.5*\iy,-2*\r)$) {$\mathbf{W}_1$};

    % This scopee draws the to-screen chicken feet
    \begin{scope}[shift={(4.875*\iy,2*\r)},canvas is zx plane at y=\iy,rotate=90]
        \draw[fill=mycustompurple] (0,\r) rectangle ++ (\r,\iz);
    \end{scope}
    \node[namenode] at ($(W1HLowerLeft)+(.6*\ix,1)$) {$\mathbf{V}_1$};
    
    % top ellipsis
    \coordinate (TopEllipsis) at (6.875*\iy,.75*\ix);
    \node[namenode] at (TopEllipsis) {$+ \quad \cdots \quad +$};
    
    %%%%%%%%% last Cyclic 1 Component %%%%%%%%%
    % This draws the vertical chicken feet
    \coordinate (URVLowerLeft) at ($(TopEllipsis) + (\iy,-.75*\ix)$);
    \draw[fill=mycustomgreen] (URVLowerLeft) rectangle ++ (\r,\iy);
    \node[namenode] at ($(URVLowerLeft)-(0,2*\r)$) {$\mathbf{U}_{r_c}$};

    % This draws the horizontal chicken feet
    \coordinate (WRHLowerLeft) at ($(URVLowerLeft) + (\r,\iy) + (\r,\r)$);
    \draw[fill=mycustomorange] (WRHLowerLeft) rectangle ++ (\ix,\r);
    \node[namenode] at ($(WRHLowerLeft)+(.5*\iy,-2*\r)$) {$\mathbf{W}_{r_c}$};

    % This scopee draws the to-screen chicken feet
    \begin{scope}[shift={(7.875*\iy,2*\r)},canvas is zx plane at y=\iy,rotate=90]
        \draw[fill=mycustompurple] (0,\r) rectangle ++ (\r,\iz);
    \end{scope}
    \node[namenode] at ($(WRHLowerLeft)+(.6*\ix,1)$) {$\mathbf{V}_{r_c}$};

    %%%%%%%%%%%%%%%%
    %%% Cyclic 2 %%%
    %%%%%%%%%%%%%%%%

    % middle sum
    \coordinate (MiddleSum) at ($(TopSum) + (0, -17*\r)$);
    \node[namenode] at (MiddleSum) {$+ $};

    %%%%%%%%% 1st Cyclic 2 Component %%%%%%%%%
    % This draws the vertical chicken feet
    \coordinate (V1VLowerLeft) at ($(MiddleSum) - (0,0.75*\iy) + (0.75,0)$);
    \draw[fill=mycustompurple] (V1VLowerLeft) rectangle ++ (\r,\iy);
    \node[namenode] at ($(V1VLowerLeft)-(0,2*\r)$) {$\mathbf{V}_1$};

    % This draws the horizontal chicken feet
    \coordinate (U1HLowerLeft) at ($(V1VLowerLeft) + (\r,\iy) + (\r,\r)$);
    \draw[fill=mycustomgreen] (U1HLowerLeft) rectangle ++ (\ix,\r);
    \node[namenode] at ($(U1HLowerLeft)+(.5*\iy,-2*\r)$) {$\mathbf{U}_1$};

    % This scopee draws the tp-screen chicken feet
    \begin{scope}[shift={(4.875*\iy,-15*\r)},canvas is zx plane at y=\iy,rotate=90]
        \draw[fill=mycustomorange] (0,\r) rectangle ++ (\r,\iz);
    \end{scope}
    \node[namenode] at ($(U1HLowerLeft)+(.6*\ix,1)$) {$\mathbf{W}_1$};
    
    % middle ellipsis
    \coordinate (MiddleEllipsis) at ($(MiddleSum) + (4.75, 0)$);
    \node[namenode] at (MiddleEllipsis) {$+ \quad \cdots \quad +$};
    
    %%%%%%%%% last Cyclic 2 Component %%%%%%%%%
    % This draws the vertical chicken feet
    \coordinate (VRVLowerLeft) at ($(MiddleEllipsis) + (\iy,-.75*\ix)$);
    \draw[fill=mycustompurple] (VRVLowerLeft) rectangle ++ (\r,\iy);
    \node[namenode] at ($(VRVLowerLeft)-(0,2*\r)$) {$\mathbf{V}_{r_c}$};

    % This draws the horizontal chicken feet
    \coordinate (URHLowerLeft) at ($(VRVLowerLeft) + (\r,\iy) + (\r,\r)$);
    \draw[fill=mycustomgreen] (URHLowerLeft) rectangle ++ (\ix,\r);
    \node[namenode] at ($(URHLowerLeft)+(.5*\iy,-2*\r)$) {$\mathbf{U}_{r_c}$};
    
    % This scopee draws the tp-screen chicken feet
    \begin{scope}[shift={(7.875*\iy,-15*\r)},canvas is zx plane at y=\iy,rotate=90]
        \draw[fill=mycustomorange] (0,\r) rectangle ++ (\r,\iz);
    \end{scope}
    \node[namenode] at ($(URHLowerLeft)+(.6*\ix,1)$) {$\mathbf{W}_{r_c}$};

    %%%%%%%%%%%%%%%%
    %%% Cyclic 3 %%%
    %%%%%%%%%%%%%%%%
    
    % bottom sum
    \coordinate (BottomSum) at ($(MiddleSum) + (0, -17*\r)$);
    \node[namenode] at (BottomSum) {$+ $};

    %%%%%%%%% 1st Cyclic 3 Component %%%%%%%%%
    % This draws the vertical chicken feet
    \coordinate (W1VLowerLeft) at ($(BottomSum) - (0,0.75*\iy) + (0.75,0)$);
    \draw[fill=mycustomorange] (W1VLowerLeft) rectangle ++ (\r,\iy);
    \node[namenode] at ($(W1VLowerLeft)-(0,2*\r)$) {$\mathbf{W}_1$};

    % This draws the horizontal chicken feet
    \coordinate (V1HLowerLeft) at ($(W1VLowerLeft) + (\r,\iy) + (\r,\r)$);
    \draw[fill=mycustompurple] (V1HLowerLeft) rectangle ++ (\ix,\r);
    \node[namenode] at ($(V1HLowerLeft)+(.5*\iy,-2*\r)$) {$\mathbf{V}_1$};

    % This scopee draws the tp-screen chicken feet
    \begin{scope}[shift={(4.875*\iy,-2.125*15*\r)},canvas is zx plane at y=\iy,rotate=90]
        \draw[fill=mycustomgreen] (0,\r) rectangle ++ (\r,\iz);
    \end{scope}
    \node[namenode] at ($(V1HLowerLeft)+(.6*\ix,1)$) {$\mathbf{U}_1$};
    
    % bottom ellipsis
    \coordinate (MiddleEllipsis) at ($(BottomSum) + (4.75, 0)$);
    \node[namenode] at (MiddleEllipsis) {$+ \quad \cdots \quad +$};
    
    %%%%%%%%% last Cyclic 3 Component %%%%%%%%%
    % This draws the vertical chicken feet
    \coordinate (WRVLowerLeft) at ($(MiddleEllipsis) + (\iy,-.75*\ix)$);
    \draw[fill=mycustomorange] (WRVLowerLeft) rectangle ++ (\r,\iy);
    \node[namenode] at ($(WRVLowerLeft)-(0,2*\r)$) {$\mathbf{W}_{r_c}$};

    % This draws the horizontal chicken feet
    \coordinate (VRHLowerLeft) at ($(WRVLowerLeft) + (\r,\iy) + (\r,\r)$);
    \draw[fill=mycustompurple] (VRHLowerLeft) rectangle ++ (\ix,\r);
    \node[namenode] at ($(VRHLowerLeft)+(.5*\iy,-2*\r)$) {$\mathbf{V}_{r_c}$};

    % This scopee draws the tp-screen chicken feet
    \begin{scope}[shift={(7.875*\iy,-2.125*15*\r)},canvas is zx plane at y=\iy,rotate=90]
        \draw[fill=mycustomgreen] (0,\r) rectangle ++ (\r,\iz);
    \end{scope}
    \node[namenode] at ($(VRHLowerLeft)+(.6*\ix,1)$) {$\mathbf{U}_{r_c}$};
\end{tikzpicture}

    \caption{CP Decomposition Diagram with Cyclic Invariant Structure}
    \label{fig:cyc_inv_cp}
\end{figure}


\newpage
\subsection{Adapting CP\_DGN to Cyclic Invariance} \label{Adapting CP-DGN to Cyclic Invariance}

    We must now adapt \Cref{sec:Cyclic Invariant Matrix Multiplication
    Algorithms} to our cyclic invariance structure. Now we have the imposed
    structure on any CP Decomposition of the matrix multiplication tensor. 

    \begin{equation*}
        \begin{array}{rcl}
            \mathbf{A} = [\begin{array}{cccc} \mathbf S & \mathbf U & \mathbf V & \mathbf W \end{array}]\\
            \mathbf{B} = [\begin{array}{cccc} \mathbf S & \mathbf W & \mathbf U & \mathbf V \end{array}]\\
            \mathbf{C} = [\begin{array}{cccc} \mathbf S & \mathbf V & \mathbf W & \mathbf U \end{array}]
        \end{array}
    \end{equation*}

    That means \Cref{eq:cp_least_squares} becomes \Cref{eq:ci_cp_least_squares}
    in accordance with \Cref{fig:cyc_inv_cp}. 

    \begin{equation} \label{eq:ci_cp_least_squares}
        f(v) = \frac{1}{2} \sum_{i}^m \sum_{j}^n \sum_{k}^p \left( x_{ijk} - \sum_{q}^{r_s} s_{iq}s_{jq}s_{kq} - \sum_{l}^{r_c} (u_{il}v_{jl}w_{kl} + w_{il}u_{jl}v_{kl} + v_{il}w_{jl}u_{kl}) \right)^2
    \end{equation}

    In order for that to be the case, we must change our $\mathbf{v}$ such that

    \begin{equation*}
        \mathbf{v} = \text{vec}
        \left(
        \left[
            \begin{array}{c}
                \mathbf{S} \\
                \mathbf{U} \\
                \mathbf{V} \\
                \mathbf{W}
            \end{array}
        \right]
        \right) = 
        \left[
            \begin{array}{c}
                \text{vec}(\mathbf{S}) \\
                \text{vec}(\mathbf{U}) \\
                \text{vec}(\mathbf{V}) \\
                \text{vec}(\mathbf{W})
            \end{array}
        \right]
        \in \mathbb{R}^{nr}
    \end{equation*}

    The Gradient becomes

    \begin{equation}
        \nabla f =
        \left[
        \begin{array}{c}
                \mathsf{vec}(\frac{\partial f}{\partial \mathbf{S}})\\
                \mathsf{vec}(\frac{\partial f}{\partial \mathbf{U}})\\
                \mathsf{vec}(\frac{\partial f}{\partial \mathbf{V}})\\
                \mathsf{vec}(\frac{\partial f}{\partial \mathbf{W}})
        \end{array}
        \right]
        \in \mathbb{R}^{(m+n+p)r}
    \end{equation}

    Where each partial derivative is defined as:

    \begin{equation}
        \begin{array}{rcl}
            \frac{\partial f}{\partial \mathbf{S}} & = & \scriptstyle 3 \mathbf{\cdot \bigl(S\left(S^\intercal S\ast S^\intercal S\right) + U\left(V^\intercal S\ast W^\intercal S\right) + V\left(U^\intercal S\ast W^\intercal S\right) + W\left(U^\intercal S\ast V^\intercal S\right)}\\ 
            & & \scriptstyle - \mathbf{(X_{(1)} + X_{(2)} + X_{(3)})(S\odot S)}\bigl) \\

            \frac{\partial f}{\partial \mathbf{U}} & = & \scriptstyle 3\mathbf{\cdot \bigl(S\left(S^\intercal V\ast S^\intercal W\right) + U\left(V^\intercal V\ast W^\intercal W\right) + V\left(W^\intercal V\ast U^\intercal W\right) + W\left(U^\intercal V\ast V^\intercal W\right)\bigl)}\\ 
            & & \scriptstyle - \mathbf{X_{(1)}(V\odot W) - X_{(2)}(W\odot V) - X_{(3)}(V\odot W)} \\

            \frac{\partial f}{\partial \mathbf{V}} & = & \scriptstyle 3\mathbf{\cdot \bigl(S\left(S^\intercal U\ast S^\intercal W\right) + U\left(W^\intercal U\ast V^\intercal W\right) + V\left(U^\intercal U\ast W^\intercal W\right) + W\left(V^\intercal U\ast U^\intercal W\right)\bigl)}\\ 
            & & \scriptstyle - \mathbf{X_{(1)}(W\odot U) - X_{(2)}(U\odot W) - X_{(3)}(W\odot U)} \\

            \frac{\partial f}{\partial \mathbf{W}} & = & \scriptstyle 3\mathbf{\cdot \bigl(S\left(S^\intercal U\ast S^\intercal V\right) + U\left(V^\intercal U\ast W^\intercal V\right) + V\left(W^\intercal U\ast U^\intercal V\right) + W\left(U^\intercal U\ast V^\intercal V\right)\bigl)}\\ 
            & & \scriptstyle - \mathbf{X_{(1)}(U\odot V) - X_{(2)}(V\odot U) - X_{(3)}(U\odot V)}
        \end{array}
        \label{eq:ci_cp_dgn_partials}
    \end{equation}

    Similarly, our jacobian becomes

    \begin{equation} \label{eq:ci_cp_dgn_jacobian}
        \mathbf{J} = [\mathbf{J}_\mathbf{S} + \mathbf{J}_\mathbf{U} + \mathbf{J}_\mathbf{V} + \mathbf{J}_\mathbf{W}] \in \mathbb{R}^{n^3\times nr}
    \end{equation}
    
    Where
    
    \begin{equation}
        \begin{array}{rcl}
                \mathbf{J}_\mathbf{S} & = \mathbf{(S\odot S)} \otimes \mathbf{I} + \mathbf{\Pi}_2^\intercal\cdot \mathbf{(S\odot S)} \otimes \mathbf{I} + \mathbf{\Pi}_3^\intercal\cdot \mathbf{(S\odot S)} \otimes \mathbf{I}\\
                \mathbf{J}_\mathbf{U} & = \mathbf{(V\odot W)} \otimes \mathbf{I} + \mathbf{\Pi}_2^\intercal\cdot \mathbf{(W\odot V)} \otimes \mathbf{I} + \mathbf{\Pi}_3^\intercal\cdot \mathbf{(V\odot W)} \otimes \mathbf{I}\\
                \mathbf{J}_\mathbf{V} & = \mathbf{(W\odot U)} \otimes \mathbf{I} + \mathbf{\Pi}_2^\intercal\cdot \mathbf{(U\odot W)} \otimes \mathbf{I} + \mathbf{\Pi}_3^\intercal\cdot \mathbf{(W\odot U)} \otimes \mathbf{I}\\
                \mathbf{J}_\mathbf{V} & = \mathbf{(U\odot V)} \otimes \mathbf{I} + \mathbf{\Pi}_2^\intercal\cdot \mathbf{(V\odot U)} \otimes \mathbf{I} + \mathbf{\Pi}_3^\intercal\cdot \mathbf{(U\odot V)} \otimes \mathbf{I}
        \end{array}
        \label{eq:ci_cp_dgn_jacobians}
    \end{equation}

    However, just like in \Cref{sec:Damped Gauss Newton Optimization for CP
    Decompositions}, we don't care as much about the explicit jacobian as we do
    about how to apply $\mathbf{J^\intercal J}$ to a vector. 

    \begin{equation}
        \mathbf{J^\intercal J} \cdot \text{vec}(\mathbf{K}) = 
        \left[
        \begin{array}{cccc}
            \mathbf{J_S^\intercal J_S} & \mathbf{J_S^\intercal J_U} & \mathbf{J_S^\intercal J_V} & \mathbf{J_S^\intercal J_W} \\
            \mathbf{J_U^\intercal J_S} & \mathbf{J_U^\intercal J_U} & \mathbf{J_U^\intercal J_V} & \mathbf{J_U^\intercal J_W} \\
            \mathbf{J_V^\intercal J_S} & \mathbf{J_V^\intercal J_U} & \mathbf{J_V^\intercal J_V} & \mathbf{J_V^\intercal J_W} \\
            \mathbf{J_W^\intercal J_S} & \mathbf{J_W^\intercal J_U} & \mathbf{J_W^\intercal J_V} & \mathbf{J_W^\intercal J_W}
        \end{array}
        \right]
        \left[
        \begin{array}{c}
            \text{vec}(\mathbf{K_S}) \\
            \text{vec}(\mathbf{K_U}) \\
            \text{vec}(\mathbf{K_V}) \\
            \text{vec}(\mathbf{K_W})
        \end{array}
        \right]
    \end{equation}

    The equations for each entry of the above matrix-vector product can be found below:

    \newpage
    \begin{equation*}
        \begin{array}{ccc}
                \mathbf{J_S^\intercal J_S \text{vec}(K_s)} & = & \scriptstyle 3\cdot \text{vec}\bigl(\mathbf{K_S(S^\intercal S)\ast (S^\intercal S) + 2\cdot S(K_S^\intercal S)\ast (S^\intercal S)}\bigl) \\
                \mathbf{J_S^\intercal J_U \text{vec}(K_u)} & = & \scriptstyle 3\cdot \text{vec}\bigl(\mathbf{K_U(V^\intercal S)\ast (W^\intercal S) + V(K_U^\intercal S)\ast (W^\intercal S) + W(K_U^\intercal S)\ast (V^\intercal S)}\bigl) \\
                \mathbf{J_S^\intercal J_V \text{vec}(K_v)} & = & \scriptstyle 3\cdot \text{vec}\bigl(\mathbf{K_V(U^\intercal S)\ast (W^\intercal S) + U(K_V^\intercal S)\ast (W^\intercal S) + W(K_V^\intercal S)\ast (U^\intercal S)}\bigl) \\
                \mathbf{J_S^\intercal J_W \text{vec}(K_w)} & = & \scriptstyle 3\cdot \text{vec}\bigl(\mathbf{K_W(U^\intercal S)\ast (V^\intercal S) + U(K_W^\intercal S)\ast (V^\intercal S) + V(K_W^\intercal S)\ast (U^\intercal S)}\bigl) \\ \\
                \mathbf{J_U^\intercal J_S \text{vec}(K_s)} & = & \scriptstyle 3\cdot \text{vec}\bigl(\mathbf{K_S(S^\intercal V)\ast (S^\intercal W) + S\bigl((K_S^\intercal W)\ast (S^\intercal V) + (K_S^\intercal V)\ast (S^\intercal W)\bigl)}\bigl) \\
                \mathbf{J_U^\intercal J_U \text{vec}(K_u)} & = & \scriptstyle 3\cdot \text{vec}\bigl(\mathbf{K_U(V^\intercal V)\ast (W^\intercal W) + V(K_U^\intercal W)\ast (W^\intercal V) + W(K_U^\intercal V)\ast (V^\intercal W)}\bigl) \\
                \mathbf{J_U^\intercal J_V \text{vec}(K_v)} & = & \scriptstyle 3\cdot \text{vec}\bigl(\mathbf{K_V(W^\intercal V)\ast (U^\intercal W) + W(K_V^\intercal W)\ast (U^\intercal V) + U(K_V^\intercal V)\ast (W^\intercal W)}\bigl) \\
                \mathbf{J_U^\intercal J_W \text{vec}(K_w)} & = & \scriptstyle 3\cdot \text{vec}\bigl(\mathbf{K_W(U^\intercal V)\ast (V^\intercal W) + U(K_W^\intercal W)\ast (V^\intercal V) + V(K_W^\intercal V)\ast (U^\intercal W)}\bigl) \\ \\
                \mathbf{J_V^\intercal J_S \text{vec}(K_s)} & = & \scriptstyle 3\cdot \text{vec}\bigl(\mathbf{K_S(S^\intercal U)\ast (S^\intercal W) + S\bigl((K_S^\intercal U)\ast (S^\intercal W) + (K_S^\intercal W)\ast (S^\intercal U)\bigl)}\bigl) \\
                \mathbf{J_V^\intercal J_U \text{vec}(K_u)} & = & \scriptstyle 3\cdot \text{vec}\bigl(\mathbf{K_U(V^\intercal W)\ast (W^\intercal U) + V(K_U^\intercal U)\ast (W^\intercal W) + W(K_U^\intercal W)\ast (V^\intercal U)}\bigl) \\
                \mathbf{J_V^\intercal J_V \text{vec}(K_v)} & = & \scriptstyle 3\cdot \text{vec}\bigl(\mathbf{K_V(W^\intercal W)\ast (U^\intercal U) + W(K_V^\intercal U)\ast (U^\intercal W) + U(K_V^\intercal W)\ast (W^\intercal U)}\bigl) \\
                \mathbf{J_V^\intercal J_W \text{vec}(K_w)} & = & \scriptstyle 3\cdot \text{vec}\bigl(\mathbf{K_W(U^\intercal W)\ast (V^\intercal U) + U(K_W^\intercal U)\ast (V^\intercal W) + V(K_W^\intercal W)\ast (U^\intercal U)}\bigl) \\ \\
                \mathbf{J_W^\intercal J_S \text{vec}(K_s)} & = & \scriptstyle 3\cdot \text{vec}\bigl(\mathbf{K_S(S^\intercal U)\ast (S^\intercal V) + S\bigl((K_S^\intercal U)\ast (S^\intercal V) + (K_S^\intercal V)\ast (S^\intercal U)\bigl)}\bigl) \\
                \mathbf{J_W^\intercal J_U \text{vec}(K_u)} & = & \scriptstyle 3\cdot \text{vec}\bigl(\mathbf{K_U(V^\intercal U)\ast (W^\intercal V) + V(K_U^\intercal V)\ast (W^\intercal U) + W(K_U^\intercal U)\ast (V^\intercal V)}\bigl) \\
                \mathbf{J_W^\intercal J_V \text{vec}(K_v)} & = & \scriptstyle 3\cdot \text{vec}\bigl(\mathbf{K_V(W^\intercal W)\ast (U^\intercal V) + W(K_V^\intercal V)\ast (U^\intercal U) + U(K_V^\intercal U)\ast (W^\intercal V)}\bigl) \\
                \mathbf{J_W^\intercal J_W \text{vec}(K_w)} & = & \scriptstyle 3\cdot \text{vec}\bigl(\mathbf{K_W(U^\intercal U)\ast (V^\intercal V) + U(K_W^\intercal V)\ast (V^\intercal U) + V(K_W^\intercal U)\ast (U^\intercal V)}\bigl) \\
        \end{array}
    \end{equation*}

    Then, we reach our algorithm
    \begin{algorithm}
        \caption{Cyclic Invariant CP Damped Gauss-Newton}
        \label{alg:Mod-CP-DGN}
        \begin{algorithmic}
            \State{\textbf{Input:} Matrix Multiplication Tensor $\mathcal{M}$,}
            \State{\hspace{3.25em} CP Tensor Rank $r$,}
            \State{\hspace{3.25em} Damping Parameter $\lambda \in \mathbb{R}$,}
            \State{\hspace{3.25em} Convergence Tolerance $\epsilon > 0$}
            
            \State \textbf{Output:} CP Tensor $\mathcal{K}$
            \Function{DGN}{$\mathcal{M}, r, \lambda, \epsilon$}
                \State{Initialize K and K\_prev to be a cell of length 4 with the first entry being of $n^2 \times r_s$ and the remaining three $n^2 \times r_c$ matrices}

                \For{$i = 1:MaxIters$}
                    \State{f $\longleftarrow \frac{1}{2} \|\mathcal{M} - \mathbf{\llbracket S, S, S\rrbracket - \llbracket U, V, W\rrbracket - \llbracket W, U, V\rrbracket - \llbracket V, W, U\rrbracket} \| $}
                    \State{$\nabla \mathbf{f} \longleftarrow [\text{vec} \left( \frac{\partial f}{\partial \mathbf S}\right) \text{vec} \left( \frac{\partial f}{\partial \mathbf U}\right) \text{vec} \left( \frac{\partial f}{\partial \mathbf V}\right) \text{vec} \left( \frac{\partial f}{\partial \mathbf W}\right)]^T$} 
                    \State{$S \longleftarrow \text{Solution to } (\textbf{J}^T\textbf{J} + \lambda I)\mathbf{K} = -\nabla \mathbf{f}$}
                    \While{\textit{Goldstein Conditions Are Satisfied}}
                        \State{K $\longleftarrow$ K\_prev + $\alpha S$}
                        \State{f $_{new}\longleftarrow$ Compute Function Value}
                        \State{$\alpha \longleftarrow \alpha / 2$}
                    \EndWhile{}
                    \If{f - f $_{new} < \epsilon$}
                        \State{\textbf{break}}
                    \EndIf{}
                \EndFor{}
            \EndFunction
        \end{algorithmic}
    \end{algorithm}

\newpage
\section{Further Structure in Matrix Multiplication Algorithms} \label{sec:Further Structure in Matrix Multiplication Algorithms}

\JP{This is ongoing work. Essentially, we are searching for additional
structure in the solutions we have found. I will try to summarize the main ideas
in the talk as well as in this section before I graduate, but realistic the
precise writing of this section won't be in its full version before I graduate}




