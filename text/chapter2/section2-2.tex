%!TEX root = ../../main.tex

We can reduce our search space in our search for fast matrix multiplication
algorithms by levevering \textbf{cyclic invariance}. Cyclic invariance is an
added structure in matrix multiplication algorithms that reduce the number of
variables of the CP Decomposition optimization problem for the matrix
multiplication tensor. 

\subsection{Cyclic Invariant Matrix Multiplication Algorithms} \label{sec:Cyclic Invariant Matrix Multiplication Algorithms} 
    
    Recall that we can rearrange the multiplications and additions of Strassen's
    agorithms to obtain variantions. Some of these variantions can be cyclic
    invariant. Below is one of these Strassen's variant algorithm next to the
    original. Both of these algorithms have already been introduced back in
    \Cref{sec:Fast Matrix Multiplication Algorithms}. Notive how the variant
    Strassen's algorithm on the left is composed of smaller submatrices that
    appear throughout the main factor matrices of the KTensor. The $4\times 1$
    matrix in red is called the \textbf{symmetric component}, and it always
    appears at the beginning of all three factor matrices, we denote it as
    $\mathbf{S}$. The remaining three $4\times 2$ submatrices are called the
    \textbf{cyclic component}, we denote them as $\mathbf{U, V, W}$. 

    \pagebreak
    \vspace{-50pt}
    \begin{multicols}{2}
        \setlength{\arraycolsep}{3pt}
        \[\begin{array}{c|ccccccc}
                & \color{blue} \mathbf{M}_1 & \color{blue} \mathbf{M}_2 & \color{blue} \mathbf{M}_3 & \color{blue} \mathbf{M}_4 & \color{blue} \mathbf{M}_5 & \color{blue} \mathbf{M}_6 & \color{blue} \mathbf{M}_7 \\
                \hline
                \mathbf{A}_{11} & 1 & 0 & 1 & 0 & 1 & -1 & 0 \\
                \mathbf{A}_{12} & 0 & 1 & 0 & 0 & 0 & 1 & 0 \\
                \mathbf{A}_{21} & 0 & 0 & 0 & 0 & 1 & 0 & 1 \\
                \mathbf{A}_{22} & 1 & 1 & 0 & 1 & 0 & 0 & -1 \\
                \hline
                \mathbf{B}_{11} & 1 & 1 & 0 & -1 & 0 & 1 & 0 \\
                \mathbf{B}_{12} & 0 & 0 & 0 & 1 & 0 & 0 & 1 \\
                \mathbf{B}_{21} & 0 & 0 & 1 & 0 & 0 & 1 & 0 \\
                \mathbf{B}_{22} & 1 & 0 & -1 & 0 & 1 & 0 & 1 \\
                \hline
                \mathbf{C}_{11} & 1 & 0 & 0 & 1 & -1 & 0 & 1 \\
                \mathbf{C}_{21} & 0 & 0 & 1 & 0 & 1 & 0 & 0 \\
                \mathbf{C}_{12} & 0 & 1 & 0 & 1 & 0 & 0 & 0 \\
                \mathbf{C}_{22} & 1 & -1 & 1 & 0 & 0 & 1 & 0 \\
        \end{array}\]

        \columnbreak

        \setlength{\arraycolsep}{3pt}
        \[\begin{array}{c|ccccccc}
                & \color{blue} \mathbf{M}_1 & \color{blue} \mathbf{M}_2 & \color{blue} \mathbf{M}_3 & \color{blue} \mathbf{M}_4 & \color{blue} \mathbf{M}_5 & \color{blue} \mathbf{M}_6 & \color{blue} \mathbf{M}_7 \\
                \hline
                \mathbf{A}_{11} & \color{red} 1 & \color{mycustomgreen} 0 & \color{mycustomgreen} 0 & \color{violet} 0 & \color{violet} 1 & \color{orange} 1 & \color{orange} -1 \\
                \mathbf{A}_{12} & \color{red} 0 & \color{mycustomgreen} 1 & \color{mycustomgreen} 0 & \color{violet} 0 & \color{violet} 0 & \color{orange} 0 & \color{orange} 1 \\
                \mathbf{A}_{21} & \color{red} 0 & \color{mycustomgreen} 0 & \color{mycustomgreen} 0 & \color{violet} 1 & \color{violet} 1 & \color{orange} 0 & \color{orange} 0 \\
                \mathbf{A}_{22} & \color{red} 1 & \color{mycustomgreen} 1 & \color{mycustomgreen} 1 & \color{violet} -1 & \color{violet} 0 & \color{orange} 0 & \color{orange} 0 \\
                \hline
                \mathbf{B}_{11} & \color{red} 1 & \color{orange} 1 & \color{orange} -1 & \color{mycustomgreen} 0 & \color{mycustomgreen} 0 & \color{violet} 0 & \color{violet} 1 \\
                \mathbf{B}_{12} & \color{red} 0 & \color{orange} 0 & \color{orange} 1 & \color{mycustomgreen} 1 & \color{mycustomgreen} 0 & \color{violet} 0 & \color{violet} 0 \\
                \mathbf{B}_{21} & \color{red} 0 & \color{orange} 0 & \color{orange} 0 & \color{mycustomgreen} 0 & \color{mycustomgreen} 0 & \color{violet} 1 & \color{violet} 1 \\
                \mathbf{B}_{22} & \color{red} 1 & \color{orange} 0 & \color{orange} 0 & \color{mycustomgreen} 1 & \color{mycustomgreen} 1 & \color{violet} -1 & \color{violet} 0 \\
                \hline
                \mathbf{C}_{11} & \color{red} 1 & \color{violet} 0 & \color{violet} 1 & \color{orange} 1 & \color{orange} -1 & \color{mycustomgreen} 0 & \color{mycustomgreen} 0 \\
                \mathbf{C}_{21} & \color{red} 0 & \color{violet} 0 & \color{violet} 0 & \color{orange} 0 & \color{orange} 1 & \color{mycustomgreen} 1 & \color{mycustomgreen} 0 \\
                \mathbf{C}_{12} & \color{red} 0 & \color{violet} 1 & \color{violet} 1 & \color{orange} 0 & \color{orange} 0 & \color{mycustomgreen} 0 & \color{mycustomgreen} 0 \\
                \mathbf{C}_{22} & \color{red} 1 & \color{violet} -1 & \color{violet} 0 & \color{orange} 0 & \color{orange} 0 & \color{mycustomgreen} 1 & \color{mycustomgreen} 1 \\
        \end{array}\]
    \end{multicols}

    Because they are always submatrices of the factor matrices, both the
    symmetric and the cyclic components have the same number of rows as the
    factor matrices, namely $n$. However, they can have a different number of
    columns. We denote the number of columns of the symmetric component as $r_s$
    and the number of columns of the cyclic component as $r_c$. Since $r$ is the
    rank of the CP Decomposition, we have that $r_s + 3r_c = r$. Because of
    this, given a matrix multiplication tensor of $n\times n$ matrices, and a
    given rank $r$, there are multiple choices for $r_c$, which in turn define
    the value of $r_s$ since $r_s = r - 3r_c$. For rank 7 algorithms of $2\times
    2$ matrices, we have two options $[r_s = 1, r_c = 2]$ and  $[r_s = 4, r_c =
    1]$. Below are two different Strassen's algorithm, on the left is the same
    algorithm we saw above of $r_s = 1$, and on the right is an $r_s = 4$
    algorithm. Besides the difference in $r_s$, these algorithms share another
    distinction, which is the number of matrix additions. The algorithm on the
    left has 18 additions while the one on the left has 24 additions, thus
    making it slighly less optimal. Generally we disregard this distinction as
    matrix addition is not the bottleneck computation. 

%     \pagebreak
    \vspace{-50pt}
    \begin{multicols}{2}
        \setlength{\arraycolsep}{3pt}
        \[\begin{array}{c|ccccccc}
                & \color{blue} \mathbf{M}_1 & \color{blue} \mathbf{M}_2 & \color{blue} \mathbf{M}_3 & \color{blue} \mathbf{M}_4 & \color{blue} \mathbf{M}_5 & \color{blue} \mathbf{M}_6 & \color{blue} \mathbf{M}_7 \\
                \hline
                \mathbf{A}_{11} & \color{red} 1 & \color{mycustomgreen} 0 & \color{mycustomgreen} 0 & \color{violet} 0 & \color{violet} 1 & \color{orange} 1 & \color{orange} -1 \\
                \mathbf{A}_{12} & \color{red} 0 & \color{mycustomgreen} 1 & \color{mycustomgreen} 0 & \color{violet} 0 & \color{violet} 0 & \color{orange} 0 & \color{orange} 1 \\
                \mathbf{A}_{21} & \color{red} 0 & \color{mycustomgreen} 0 & \color{mycustomgreen} 0 & \color{violet} 1 & \color{violet} 1 & \color{orange} 0 & \color{orange} 0 \\
                \mathbf{A}_{22} & \color{red} 1 & \color{mycustomgreen} 1 & \color{mycustomgreen} 1 & \color{violet} -1 & \color{violet} 0 & \color{orange} 0 & \color{orange} 0 \\
                \hline
                \mathbf{B}_{11} & \color{red} 1 & \color{orange} 1 & \color{orange} -1 & \color{mycustomgreen} 0 & \color{mycustomgreen} 0 & \color{violet} 0 & \color{violet} 1 \\
                \mathbf{B}_{12} & \color{red} 0 & \color{orange} 0 & \color{orange} 1 & \color{mycustomgreen} 1 & \color{mycustomgreen} 0 & \color{violet} 0 & \color{violet} 0 \\
                \mathbf{B}_{21} & \color{red} 0 & \color{orange} 0 & \color{orange} 0 & \color{mycustomgreen} 0 & \color{mycustomgreen} 0 & \color{violet} 1 & \color{violet} 1 \\
                \mathbf{B}_{22} & \color{red} 1 & \color{orange} 0 & \color{orange} 0 & \color{mycustomgreen} 1 & \color{mycustomgreen} 1 & \color{violet} -1 & \color{violet} 0 \\
                \hline
                \mathbf{C}_{11} & \color{red} 1 & \color{violet} 0 & \color{violet} 1 & \color{orange} 1 & \color{orange} -1 & \color{mycustomgreen} 0 & \color{mycustomgreen} 0 \\
                \mathbf{C}_{21} & \color{red} 0 & \color{violet} 0 & \color{violet} 0 & \color{orange} 0 & \color{orange} 1 & \color{mycustomgreen} 1 & \color{mycustomgreen} 0 \\
                \mathbf{C}_{12} & \color{red} 0 & \color{violet} 1 & \color{violet} 1 & \color{orange} 0 & \color{orange} 0 & \color{mycustomgreen} 0 & \color{mycustomgreen} 0 \\
                \mathbf{C}_{22} & \color{red} 1 & \color{violet} -1 & \color{violet} 0 & \color{orange} 0 & \color{orange} 0 & \color{mycustomgreen} 1 & \color{mycustomgreen} 1 \\
        \end{array}\]
    
        \columnbreak
        
        \setlength{\arraycolsep}{3pt}
        \[\begin{array}{c|ccccccc}
                & \color{blue} \mathbf{M}_1 & \color{blue} \mathbf{M}_2 & \color{blue} \mathbf{M}_3 & \color{blue} \mathbf{M}_4 & \color{blue} \mathbf{M}_5 & \color{blue} \mathbf{M}_6 & \color{blue} \mathbf{M}_7 \\
                \hline
                \mathbf{A}_{11} & \color{red} 1 & \color{red} 0 & \color{red} 0 & \color{red} 0  & \color{mycustomgreen} 0 &  \color{violet} 1 &  \color{orange} 0 \\
                \mathbf{A}_{12} & \color{red} 0 & \color{red} 0 & \color{red} -1 & \color{red} 1 & \color{mycustomgreen} 0 &  \color{violet} 1 &  \color{orange} -1 \\
                \mathbf{A}_{21} & \color{red} 0 & \color{red} 1 & \color{red} 0 & \color{red} -1 & \color{mycustomgreen} -1 & \color{violet} -1 & \color{orange} 0 \\
                \mathbf{A}_{22} & \color{red} 0 & \color{red} 1 & \color{red} 1 & \color{red} -1 & \color{mycustomgreen} 0 &  \color{violet} -1 & \color{orange} 0 \\
                \hline
                \mathbf{B}_{11} & \color{red} 1 & \color{red} 0 & \color{red} 0 & \color{red} 0  & \color{orange} 0 &  \color{mycustomgreen} 0 &  \color{violet} 1 \\
                \mathbf{B}_{12} & \color{red} 0 & \color{red} 0 & \color{red} -1 & \color{red} 1 & \color{orange} -1 & \color{mycustomgreen} 0 &  \color{violet} 1 \\
                \mathbf{B}_{21} & \color{red} 0 & \color{red} 1 & \color{red} 0 & \color{red} -1 & \color{orange} 0 &  \color{mycustomgreen} -1 & \color{violet} -1 \\
                \mathbf{B}_{22} & \color{red} 0 & \color{red} 1 & \color{red} 1 & \color{red} -1 & \color{orange} 0 &  \color{mycustomgreen} 0 &  \color{violet} -1 \\
                \hline
                \mathbf{C}_{11} & \color{red} 1 & \color{red} 0 & \color{red} 0 & \color{red} 0  & \color{violet} 1 &  \color{orange} 0 &  \color{mycustomgreen} 0 \\
                \mathbf{C}_{21} & \color{red} 0 & \color{red} 0 & \color{red} -1 & \color{red} 1 & \color{violet} 1 &  \color{orange} -1 & \color{mycustomgreen} 0 \\
                \mathbf{C}_{12} & \color{red} 0 & \color{red} 1 & \color{red} 0 & \color{red} -1 & \color{violet} -1 & \color{orange} 0 &  \color{mycustomgreen} -1 \\
                \mathbf{C}_{22} & \color{red} 0 & \color{red} 1 & \color{red} 1 & \color{red} -1 & \color{violet} -1 & \color{orange} 0 &  \color{mycustomgreen} 0 \\
        \end{array}\]    
    \end{multicols}

% The following two are the previous two in format number two
% \begin{multicols}{2}
%         \textbf{Variant of Strassen's Algorithm (Rs=1 Rc=2)}
%         \begin{eqnarray*}
%                 \color{blue} M_1 & = & (A_{11} + A_{22})\cdot (B_{11} + B_{22}) \\
%                 \color{blue} M_2 & = & (A_{12} + A_{22})\cdot B_{11} \\
%                 \color{blue} M_3 & = & A_{22}\cdot (B_{12} - B_{11}) \\
%                 \color{blue} M_4 & = & (A_{21} - A_{22})\cdot (B_{12} + B_{22}) \\
%                 \color{blue} M_5 & = & (A_{11} + A_{21})\cdot B_{22} \\
%                 \color{blue} M_6 & = & A_{11}\cdot (B_{21} - B_{22}) \\
%                 \color{blue} M_7 & = & (A_{12} - A_{11})\cdot (B_{11} + B_{21}) \\
%                 C_{11} & = & \color{blue} M_1 \color{black} + \color{blue} M_3 \color{black} - \color{blue} M_4 \color{black} - \color{blue} M_6 \\
%                 C_{12} & = & \color{blue} M_2 \color{black} + \color{blue} M_3 \\
%                 C_{21} & = & \color{blue} M_5 \color{black} + \color{blue} M_6 \\
%                 C_{22} & = & \color{blue} M_1 \color{black} - \color{blue} M_2 \color{black} + \color{blue} M_5 \color{black} + \color{blue} M_7
%         \end{eqnarray*}

%         \columnbreak

%         \textbf{Variant of Strassen's Algorithm (Rs=4 Rc=1)}
%         \begin{eqnarray*}
%                 \color{blue} M_1 & = & A_{11}\cdot B_{11} \\
%                 \color{blue} M_2 & = & (A_{12} + A_{22})\cdot (B_{12} + B_{22}) \\
%                 \color{blue} M_3 & = & (A_{22} - A_{21})\cdot (B_{22} - B_{21}) \\
%                 \color{blue} M_4 & = & (A_{21} - A_{12} - A_{22})\cdot (B_{21} - B_{12} - B_{22}) \\
%                 \color{blue} M_5 & = & (-A_{12})\cdot (-B_{21}) \\
%                 \color{blue} M_6 & = & (A_{11} - A_{12} + A_{21} - A_{22})\cdot (- B_{12}) \\
%                 \color{blue} M_7 & = & (- A_{21})\cdot (B_{11} - B_{12} + B_{21} - B_{22}) \\
%                 C_{11} & = & \color{blue} M_1 \color{black} + \color{blue} M_5 \\
%                 C_{22} & = & \color{blue} M_4 \color{black} - \color{blue} M_3 \color{black} + \color{blue} M_5 \color{black} + \color{blue} M_6 \\
%                 C_{22} & = & \color{blue} M_2 \color{black} - \color{blue} M_4 \color{black} - \color{blue} M_5 \color{black} + \color{blue} M_7 \\
%                 C_{22} & = & \color{blue} M_2 \color{black} + \color{blue} M_3 \color{black} - \color{blue} M_4 \color{black} + \color{blue} M_5
%         \end{eqnarray*}
% \end{multicols}

Recall that our factor matrices correspond to the components of the KTensor as
seen in \Cref{fig:KTensor_factor_matrices}. The same factor matrices with ths
cyclic invariant structure imposed on them now look like
\Cref{fig:cyc_inv_matrices}. That means we can now adjust \Cref{fig:KTensor} to
look like \Cref{fig:cyc_inv_cp}.

\begin{figure}
    \centering
    \begin{tikzpicture}[scale=1.5,namenode/.style={scale=1}]
    \def\ix{2} % This modifies the width of the objects
    \def\iy{1.5} % This modifies the height of the objects
    \def\is{1} %
    \def\h{.75}
    \def\corescale{1.75}
    \def\rx{\ix/\corescale}
    \def\ry{\iy/\corescale}
    \def\rz{\is/\corescale}
    
    \coordinate (MiddleS) at (0, 0);
    \draw[fill=red] (MiddleS) rectangle ++ (\is,\iy);
    \node[namenode] at ($(MiddleS) + (0.5*\is, 0.5*\iy)$)  {$\mathbf{S}$};
    \node[namenode] at ($(MiddleS) + (-0.5*\is, 0.5*\iy)$)  {$\mathbf{B}$};
    
    \coordinate (MiddleW) at ($(MiddleS) + (\is,0)$);
    \draw[fill=mycustomorange] (MiddleW) rectangle ++ (\ix,\iy);
    \node[namenode] at ($(MiddleW) + (0.5*\ix, 0.5*\iy)$)  {$\mathbf{W}$};

    \coordinate (MiddleU) at ($(MiddleW) + (\ix,0)$);
    \draw[fill=mycustomgreen] (MiddleU) rectangle ++ (\ix,\iy);
    \node[namenode] at ($(MiddleU) + (0.5*\ix, 0.5*\iy)$)  {$\mathbf{U}$};
    
    \coordinate (MiddleV) at ($(MiddleU) + (\ix,0)$);
    \draw[fill=mycustompurple] (MiddleV) rectangle ++ (\ix,\iy);
    \node[namenode] at ($(MiddleV) + (0.5*\ix, 0.5*\iy)$)  {$\mathbf{V}$};
    \node[namenode] at ($(MiddleV) + (1.25*\ix, 0.5*\iy)$)  {$n^2$};


    \coordinate (TopS) at ($(MiddleS) + (0, \iy + \h)$);
    \draw[fill=red] (TopS) rectangle ++ (\is,\iy);
    \node[namenode] at ($(TopS) + (0.5*\is, 0.5*\iy)$)  {$\mathbf{S}$};
    \node[namenode] at ($(TopS) + (-0.5*\is, 0.5*\iy)$)  {$\mathbf{A}$};

    \coordinate (TopU) at ($(TopS) + (\is,0)$);
    \draw[fill=mycustomgreen] (TopU) rectangle ++ (\ix,\iy);
    \node[namenode] at ($(TopU) + (0.5*\ix, 0.5*\iy)$)  {$\mathbf{U}$};

    \coordinate (TopV) at ($(TopU) + (\ix,0)$);
    \draw[fill=mycustompurple] (TopV) rectangle ++ (\ix,\iy);
    \node[namenode] at ($(TopV) + (0.5*\ix, 0.5*\iy)$)  {$\mathbf{V}$};
    
    \coordinate (TopW) at ($(TopV) + (\ix,0)$);
    \draw[fill=mycustomorange] (TopW) rectangle ++ (\ix,\iy);
    \node[namenode] at ($(TopW) + (0.5*\ix, 0.5*\iy)$)  {$\mathbf{W}$};
    \node[namenode] at ($(TopW) + (1.25*\ix, 0.5*\iy)$)  {$n^2$};


    \coordinate (BottomS) at ($(MiddleS) - (0, \iy + \h)$);
    \draw[fill=red] (BottomS) rectangle ++ (\is,\iy);
    \node[namenode] at ($(BottomS) + (0.5*\is, 0.5*\iy)$)  {$\mathbf{S}$};
    \node[namenode] at ($(BottomS) + (-0.5*\is, 0.5*\iy)$)  {$\mathbf{C}$};
    \node[namenode] at ($(BottomS) + (0.5*\is, -0.25*\iy)$)  {$r_s$};

    \coordinate (BottomV) at ($(BottomS) + (\is,0)$);
    \draw[fill=mycustompurple] (BottomV) rectangle ++ (\ix,\iy);
    \node[namenode] at ($(BottomV) + (0.5*\ix, 0.5*\iy)$)  {$\mathbf{V}$};
    \node[namenode] at ($(BottomV) + (0.5*\ix, -0.25*\iy)$)  {$r_c$};

    \coordinate (BottomW) at ($(BottomV) + (\ix,0)$);
    \draw[fill=mycustomorange] (BottomW) rectangle ++ (\ix,\iy);
    \node[namenode] at ($(BottomW) + (0.5*\ix, 0.5*\iy)$)  {$\mathbf{W}$};
    \node[namenode] at ($(BottomW) + (0.5*\ix, -0.25*\iy)$)  {$r_c$};
    \node[namenode] at ($(\is/2 + 3*\ix/2, -2.15*\iy)$)  {$r = r_s + 3r_c$};
    
    \coordinate (BottomU) at ($(BottomW) + (\ix,0)$);
    \draw[fill=mycustomgreen] (BottomU) rectangle ++ (\ix,\iy);
    \node[namenode] at ($(BottomU) + (0.5*\ix, 0.5*\iy)$)  {$\mathbf{U}$};
    \node[namenode] at ($(BottomU) + (0.5*\ix, -0.25*\iy)$)  {$r_c$};
    \node[namenode] at ($(BottomU) + (1.25*\ix, 0.5*\iy)$)  {$n^2$};
\end{tikzpicture}
    \caption{Cyclic Invariance in a KTensor}
    \label{fig:cyc_inv_matrices}
\end{figure}

\begin{figure}
    \centering
    \begin{tikzpicture}[scale=0.75,namenode/.style={scale=0.75}]
    \def\ix{2} % This modifies the height of the objects
    \def\iy{2} % This modifies the width of the objects
    \def\iz{2.5} %
    \def\r{.25}

    %%%%%%%%%%%%%%%
    %%%    S    %%%
    %%%%%%%%%%%%%%%

    %%%%%%%%% 1st S Component %%%%%%%%%
    % This draws the vertical chicken feet
    \coordinate (S1VLowerLeft) at (0,0);
    \draw[fill=red] (S1VLowerLeft) rectangle ++ (\r,\iy);
    \node[namenode] at ($(S1VLowerLeft)-(0,2*\r)$) {$\mathbf{S}_1$};

    % This draws the horizontal chicken feet
    \coordinate (S1HLowerLeft) at ($(S1VLowerLeft) + (\r,\iy) + (\r,\r)$);
    \draw[fill=red] (S1HLowerLeft) rectangle ++ (\ix,\r);
    \node[namenode] at ($(S1HLowerLeft)+(.5*\iy,-2*\r)$) {$\mathbf{S}_1$};

    % This scopee draws the tp-screen chicken feet
    \begin{scope}[shift={(0,2*\r)},canvas is zx plane at y=\iy,rotate=90]
        \draw[fill=red] (0,\r) rectangle ++ (\r,\iz);
    \end{scope}
    \node[namenode] at ($(S1HLowerLeft)+(.6*\ix,1)$) {$\mathbf{S}_1$};
    
    % S ellipsis
    \coordinate (SEllipsis) at (2*\iy,.75*\ix);
    \node[namenode] at (SEllipsis) {$+ \quad \cdots \quad +$};
    
    %%%%%%%%% last S Component %%%%%%%%%
    % This draws the vertical chicken feet
    \coordinate (SRVLowerLeft) at ($(SEllipsis) + (\iy,-.75*\ix)$);
    \draw[fill=red] (SRVLowerLeft) rectangle ++ (\r,\iy);
    \node[namenode] at ($(SRVLowerLeft)-(0,2*\r)$) {$\mathbf{S}_{r_s}$};

    % This draws the horizontal chicken feet
    \coordinate (SRHLowerLeft) at ($(SRVLowerLeft) + (\r,\iy) + (\r,\r)$);
    \draw[fill=red] (SRHLowerLeft) rectangle ++ (\ix,\r);
    \node[namenode] at ($(SRHLowerLeft)+(.5*\iy,-2*\r)$) {$\mathbf{S}_{r_s}$};
    
        % This scopee draws the tp-screen chicken feet
    \begin{scope}[shift={(3*\iy,2*\r)},canvas is zx plane at y=\iy,rotate=90]
        \draw[fill=red] (0,\r) rectangle ++ (\r,\iz);
    \end{scope}
    \node[namenode] at ($(SRHLowerLeft)+(.6*\ix,1)$) {$\mathbf{S}_{r_s}$};

    %%%%%%%%%%%%%%%%
    %%% Cyclic 1 %%%
    %%%%%%%%%%%%%%%%
    
    % top sum
    \coordinate (TopSum) at (4.5*\iy,.75*\ix);
    \node[namenode] at (TopSum) {$+ $};

    %%%%%%%%% 1st Cyclic 1 Component %%%%%%%%%
    % This draws the vertical chicken feet
    \coordinate (U1VLowerLeft) at ($(TopSum) - (0,0.75*\iy) + (0.75,0)$);
    \draw[fill=mycustomgreen] (U1VLowerLeft) rectangle ++ (\r,\iy);
    \node[namenode] at ($(U1VLowerLeft)-(0,2*\r)$) {$\mathbf{U}_1$};

    % This draws the horizontal chicken feet
    \coordinate (W1HLowerLeft) at ($(U1VLowerLeft) + (\r,\iy) + (\r,\r)$);
    \draw[fill=mycustomorange] (W1HLowerLeft) rectangle ++ (\ix,\r);
    \node[namenode] at ($(W1HLowerLeft)+(.5*\iy,-2*\r)$) {$\mathbf{W}_1$};

    % This scopee draws the to-screen chicken feet
    \begin{scope}[shift={(4.875*\iy,2*\r)},canvas is zx plane at y=\iy,rotate=90]
        \draw[fill=mycustompurple] (0,\r) rectangle ++ (\r,\iz);
    \end{scope}
    \node[namenode] at ($(W1HLowerLeft)+(.6*\ix,1)$) {$\mathbf{V}_1$};
    
    % top ellipsis
    \coordinate (TopEllipsis) at (6.875*\iy,.75*\ix);
    \node[namenode] at (TopEllipsis) {$+ \quad \cdots \quad +$};
    
    %%%%%%%%% last Cyclic 1 Component %%%%%%%%%
    % This draws the vertical chicken feet
    \coordinate (URVLowerLeft) at ($(TopEllipsis) + (\iy,-.75*\ix)$);
    \draw[fill=mycustomgreen] (URVLowerLeft) rectangle ++ (\r,\iy);
    \node[namenode] at ($(URVLowerLeft)-(0,2*\r)$) {$\mathbf{U}_{r_c}$};

    % This draws the horizontal chicken feet
    \coordinate (WRHLowerLeft) at ($(URVLowerLeft) + (\r,\iy) + (\r,\r)$);
    \draw[fill=mycustomorange] (WRHLowerLeft) rectangle ++ (\ix,\r);
    \node[namenode] at ($(WRHLowerLeft)+(.5*\iy,-2*\r)$) {$\mathbf{W}_{r_c}$};

    % This scopee draws the to-screen chicken feet
    \begin{scope}[shift={(7.875*\iy,2*\r)},canvas is zx plane at y=\iy,rotate=90]
        \draw[fill=mycustompurple] (0,\r) rectangle ++ (\r,\iz);
    \end{scope}
    \node[namenode] at ($(WRHLowerLeft)+(.6*\ix,1)$) {$\mathbf{V}_{r_c}$};

    %%%%%%%%%%%%%%%%
    %%% Cyclic 2 %%%
    %%%%%%%%%%%%%%%%

    % middle sum
    \coordinate (MiddleSum) at ($(TopSum) + (0, -17*\r)$);
    \node[namenode] at (MiddleSum) {$+ $};

    %%%%%%%%% 1st Cyclic 2 Component %%%%%%%%%
    % This draws the vertical chicken feet
    \coordinate (V1VLowerLeft) at ($(MiddleSum) - (0,0.75*\iy) + (0.75,0)$);
    \draw[fill=mycustompurple] (V1VLowerLeft) rectangle ++ (\r,\iy);
    \node[namenode] at ($(V1VLowerLeft)-(0,2*\r)$) {$\mathbf{V}_1$};

    % This draws the horizontal chicken feet
    \coordinate (U1HLowerLeft) at ($(V1VLowerLeft) + (\r,\iy) + (\r,\r)$);
    \draw[fill=mycustomgreen] (U1HLowerLeft) rectangle ++ (\ix,\r);
    \node[namenode] at ($(U1HLowerLeft)+(.5*\iy,-2*\r)$) {$\mathbf{U}_1$};

    % This scopee draws the tp-screen chicken feet
    \begin{scope}[shift={(4.875*\iy,-15*\r)},canvas is zx plane at y=\iy,rotate=90]
        \draw[fill=mycustomorange] (0,\r) rectangle ++ (\r,\iz);
    \end{scope}
    \node[namenode] at ($(U1HLowerLeft)+(.6*\ix,1)$) {$\mathbf{W}_1$};
    
    % middle ellipsis
    \coordinate (MiddleEllipsis) at ($(MiddleSum) + (4.75, 0)$);
    \node[namenode] at (MiddleEllipsis) {$+ \quad \cdots \quad +$};
    
    %%%%%%%%% last Cyclic 2 Component %%%%%%%%%
    % This draws the vertical chicken feet
    \coordinate (VRVLowerLeft) at ($(MiddleEllipsis) + (\iy,-.75*\ix)$);
    \draw[fill=mycustompurple] (VRVLowerLeft) rectangle ++ (\r,\iy);
    \node[namenode] at ($(VRVLowerLeft)-(0,2*\r)$) {$\mathbf{V}_{r_c}$};

    % This draws the horizontal chicken feet
    \coordinate (URHLowerLeft) at ($(VRVLowerLeft) + (\r,\iy) + (\r,\r)$);
    \draw[fill=mycustomgreen] (URHLowerLeft) rectangle ++ (\ix,\r);
    \node[namenode] at ($(URHLowerLeft)+(.5*\iy,-2*\r)$) {$\mathbf{U}_{r_c}$};
    
    % This scopee draws the tp-screen chicken feet
    \begin{scope}[shift={(7.875*\iy,-15*\r)},canvas is zx plane at y=\iy,rotate=90]
        \draw[fill=mycustomorange] (0,\r) rectangle ++ (\r,\iz);
    \end{scope}
    \node[namenode] at ($(URHLowerLeft)+(.6*\ix,1)$) {$\mathbf{W}_{r_c}$};

    %%%%%%%%%%%%%%%%
    %%% Cyclic 3 %%%
    %%%%%%%%%%%%%%%%
    
    % bottom sum
    \coordinate (BottomSum) at ($(MiddleSum) + (0, -17*\r)$);
    \node[namenode] at (BottomSum) {$+ $};

    %%%%%%%%% 1st Cyclic 3 Component %%%%%%%%%
    % This draws the vertical chicken feet
    \coordinate (W1VLowerLeft) at ($(BottomSum) - (0,0.75*\iy) + (0.75,0)$);
    \draw[fill=mycustomorange] (W1VLowerLeft) rectangle ++ (\r,\iy);
    \node[namenode] at ($(W1VLowerLeft)-(0,2*\r)$) {$\mathbf{W}_1$};

    % This draws the horizontal chicken feet
    \coordinate (V1HLowerLeft) at ($(W1VLowerLeft) + (\r,\iy) + (\r,\r)$);
    \draw[fill=mycustompurple] (V1HLowerLeft) rectangle ++ (\ix,\r);
    \node[namenode] at ($(V1HLowerLeft)+(.5*\iy,-2*\r)$) {$\mathbf{V}_1$};

    % This scopee draws the tp-screen chicken feet
    \begin{scope}[shift={(4.875*\iy,-2.125*15*\r)},canvas is zx plane at y=\iy,rotate=90]
        \draw[fill=mycustomgreen] (0,\r) rectangle ++ (\r,\iz);
    \end{scope}
    \node[namenode] at ($(V1HLowerLeft)+(.6*\ix,1)$) {$\mathbf{U}_1$};
    
    % bottom ellipsis
    \coordinate (MiddleEllipsis) at ($(BottomSum) + (4.75, 0)$);
    \node[namenode] at (MiddleEllipsis) {$+ \quad \cdots \quad +$};
    
    %%%%%%%%% last Cyclic 3 Component %%%%%%%%%
    % This draws the vertical chicken feet
    \coordinate (WRVLowerLeft) at ($(MiddleEllipsis) + (\iy,-.75*\ix)$);
    \draw[fill=mycustomorange] (WRVLowerLeft) rectangle ++ (\r,\iy);
    \node[namenode] at ($(WRVLowerLeft)-(0,2*\r)$) {$\mathbf{W}_{r_c}$};

    % This draws the horizontal chicken feet
    \coordinate (VRHLowerLeft) at ($(WRVLowerLeft) + (\r,\iy) + (\r,\r)$);
    \draw[fill=mycustompurple] (VRHLowerLeft) rectangle ++ (\ix,\r);
    \node[namenode] at ($(VRHLowerLeft)+(.5*\iy,-2*\r)$) {$\mathbf{V}_{r_c}$};

    % This scopee draws the tp-screen chicken feet
    \begin{scope}[shift={(7.875*\iy,-2.125*15*\r)},canvas is zx plane at y=\iy,rotate=90]
        \draw[fill=mycustomgreen] (0,\r) rectangle ++ (\r,\iz);
    \end{scope}
    \node[namenode] at ($(VRHLowerLeft)+(.6*\ix,1)$) {$\mathbf{U}_{r_c}$};
\end{tikzpicture}

    \caption{CP Decomposition Diagram with Cyclic Invariant Structure}
    \label{fig:cyc_inv_cp}
\end{figure}


\newpage
\subsection{Adapting CP\_DGN to Cyclic Invariance} \label{Adapting CP-DGN to Cyclic Invariance}

    We must now adapt \Cref{sec:Cyclic Invariant Matrix Multiplication
    Algorithms} to our cyclic invariance structure. Now we have the imposed
    structure on any CP Decomposition of the matrix multiplication tensor. 

    \begin{equation*}
        \begin{array}{rcl}
            \mathbf{A} = [\begin{array}{cccc} \mathbf S & \mathbf U & \mathbf V & \mathbf W \end{array}]\\
            \mathbf{B} = [\begin{array}{cccc} \mathbf S & \mathbf W & \mathbf U & \mathbf V \end{array}]\\
            \mathbf{C} = [\begin{array}{cccc} \mathbf S & \mathbf V & \mathbf W & \mathbf U \end{array}]
        \end{array}
    \end{equation*}

    That means \Cref{eq:cp_least_squares} becomes \Cref{eq:ci_cp_least_squares}
    in accordance with \Cref{fig:cyc_inv_cp}. 

    \begin{equation} \label{eq:ci_cp_least_squares}
        f(v) = \frac{1}{2} \sum_{i}^m \sum_{j}^n \sum_{k}^p \left( x_{ijk} - \sum_{q}^{r_s} s_{iq}s_{jq}s_{kq} - \sum_{l}^{r_c} (u_{il}v_{jl}w_{kl} + w_{il}u_{jl}v_{kl} + v_{il}w_{jl}u_{kl}) \right)^2
    \end{equation}

    In order for that to be the case, we must change our $\mathbf{v}$ such that

    \begin{equation*}
        \mathbf{v} = \text{vec}
        \left(
        \left[
            \begin{array}{c}
                \mathbf{S} \\
                \mathbf{U} \\
                \mathbf{V} \\
                \mathbf{W}
            \end{array}
        \right]
        \right) = 
        \left[
            \begin{array}{c}
                \text{vec}(\mathbf{S}) \\
                \text{vec}(\mathbf{U}) \\
                \text{vec}(\mathbf{V}) \\
                \text{vec}(\mathbf{W})
            \end{array}
        \right]
        \in \mathbb{R}^{nr}
    \end{equation*}

    The Gradient becomes

    \begin{equation}
        \nabla f =
        \left[
        \begin{array}{c}
                \mathsf{vec}(\frac{\partial f}{\partial \mathbf{S}})\\
                \mathsf{vec}(\frac{\partial f}{\partial \mathbf{U}})\\
                \mathsf{vec}(\frac{\partial f}{\partial \mathbf{V}})\\
                \mathsf{vec}(\frac{\partial f}{\partial \mathbf{W}})
        \end{array}
        \right]
        \in \mathbb{R}^{(m+n+p)r}
    \end{equation}

    Where each partial derivative is defined as:

    \begin{equation}
        \begin{array}{rcl}
            \frac{\partial f}{\partial \mathbf{S}} & = & \scriptstyle 3 \mathbf{\cdot \bigl(S\left(S^\intercal S\ast S^\intercal S\right) + U\left(V^\intercal S\ast W^\intercal S\right) + V\left(U^\intercal S\ast W^\intercal S\right) + W\left(U^\intercal S\ast V^\intercal S\right)}\\ 
            & & \scriptstyle - \mathbf{(X_{(1)} + X_{(2)} + X_{(3)})(S\odot S)}\bigl) \\

            \frac{\partial f}{\partial \mathbf{U}} & = & \scriptstyle 3\mathbf{\cdot \bigl(S\left(S^\intercal V\ast S^\intercal W\right) + U\left(V^\intercal V\ast W^\intercal W\right) + V\left(W^\intercal V\ast U^\intercal W\right) + W\left(U^\intercal V\ast V^\intercal W\right)\bigl)}\\ 
            & & \scriptstyle - \mathbf{X_{(1)}(V\odot W) - X_{(2)}(W\odot V) - X_{(3)}(V\odot W)} \\

            \frac{\partial f}{\partial \mathbf{V}} & = & \scriptstyle 3\mathbf{\cdot \bigl(S\left(S^\intercal U\ast S^\intercal W\right) + U\left(W^\intercal U\ast V^\intercal W\right) + V\left(U^\intercal U\ast W^\intercal W\right) + W\left(V^\intercal U\ast U^\intercal W\right)\bigl)}\\ 
            & & \scriptstyle - \mathbf{X_{(1)}(W\odot U) - X_{(2)}(U\odot W) - X_{(3)}(W\odot U)} \\

            \frac{\partial f}{\partial \mathbf{W}} & = & \scriptstyle 3\mathbf{\cdot \bigl(S\left(S^\intercal U\ast S^\intercal V\right) + U\left(V^\intercal U\ast W^\intercal V\right) + V\left(W^\intercal U\ast U^\intercal V\right) + W\left(U^\intercal U\ast V^\intercal V\right)\bigl)}\\ 
            & & \scriptstyle - \mathbf{X_{(1)}(U\odot V) - X_{(2)}(V\odot U) - X_{(3)}(U\odot V)}
        \end{array}
        \label{eq:ci_cp_dgn_partials}
    \end{equation}

    Similarly, our jacobian becomes

    \begin{equation} \label{eq:ci_cp_dgn_jacobian}
        \mathbf{J} = [\mathbf{J}_\mathbf{S} + \mathbf{J}_\mathbf{U} + \mathbf{J}_\mathbf{V} + \mathbf{J}_\mathbf{W}] \in \mathbb{R}^{n^3\times nr}
    \end{equation}
    
    Where
    
    \begin{equation}
        \begin{array}{rcl}
                \mathbf{J}_\mathbf{S} & = \mathbf{(S\odot S)} \otimes \mathbf{I} + \mathbf{\Pi}_2^\intercal\cdot \mathbf{(S\odot S)} \otimes \mathbf{I} + \mathbf{\Pi}_3^\intercal\cdot \mathbf{(S\odot S)} \otimes \mathbf{I}\\
                \mathbf{J}_\mathbf{U} & = \mathbf{(V\odot W)} \otimes \mathbf{I} + \mathbf{\Pi}_2^\intercal\cdot \mathbf{(W\odot V)} \otimes \mathbf{I} + \mathbf{\Pi}_3^\intercal\cdot \mathbf{(V\odot W)} \otimes \mathbf{I}\\
                \mathbf{J}_\mathbf{V} & = \mathbf{(W\odot U)} \otimes \mathbf{I} + \mathbf{\Pi}_2^\intercal\cdot \mathbf{(U\odot W)} \otimes \mathbf{I} + \mathbf{\Pi}_3^\intercal\cdot \mathbf{(W\odot U)} \otimes \mathbf{I}\\
                \mathbf{J}_\mathbf{V} & = \mathbf{(U\odot V)} \otimes \mathbf{I} + \mathbf{\Pi}_2^\intercal\cdot \mathbf{(V\odot U)} \otimes \mathbf{I} + \mathbf{\Pi}_3^\intercal\cdot \mathbf{(U\odot V)} \otimes \mathbf{I}
        \end{array}
        \label{eq:ci_cp_dgn_jacobians}
    \end{equation}

    However, just like in \Cref{sec:Damped Gauss Newton Optimization for CP
    Decompositions}, we don't care as much about the explicit jacobian as we do
    about how to apply $\mathbf{J^\intercal J}$ to a vector. 

    \begin{equation}
        \mathbf{J^\intercal J} \cdot \text{vec}(\mathbf{K}) = 
        \left[
        \begin{array}{cccc}
            \mathbf{J_S^\intercal J_S} & \mathbf{J_S^\intercal J_U} & \mathbf{J_S^\intercal J_V} & \mathbf{J_S^\intercal J_W} \\
            \mathbf{J_U^\intercal J_S} & \mathbf{J_U^\intercal J_U} & \mathbf{J_U^\intercal J_V} & \mathbf{J_U^\intercal J_W} \\
            \mathbf{J_V^\intercal J_S} & \mathbf{J_V^\intercal J_U} & \mathbf{J_V^\intercal J_V} & \mathbf{J_V^\intercal J_W} \\
            \mathbf{J_W^\intercal J_S} & \mathbf{J_W^\intercal J_U} & \mathbf{J_W^\intercal J_V} & \mathbf{J_W^\intercal J_W}
        \end{array}
        \right]
        \left[
        \begin{array}{c}
            \text{vec}(\mathbf{K_S}) \\
            \text{vec}(\mathbf{K_U}) \\
            \text{vec}(\mathbf{K_V}) \\
            \text{vec}(\mathbf{K_W})
        \end{array}
        \right]
    \end{equation}

    The equations for each entry of the above matrix-vector product can be found below:

    \newpage
    \begin{equation*}
        \begin{array}{ccc}
                \mathbf{J_S^\intercal J_S \text{vec}(K_s)} & = & \scriptstyle 3\cdot \text{vec}\bigl(\mathbf{K_S(S^\intercal S)\ast (S^\intercal S) + 2\cdot S(K_S^\intercal S)\ast (S^\intercal S)}\bigl) \\
                \mathbf{J_S^\intercal J_U \text{vec}(K_u)} & = & \scriptstyle 3\cdot \text{vec}\bigl(\mathbf{K_U(V^\intercal S)\ast (W^\intercal S) + V(K_U^\intercal S)\ast (W^\intercal S) + W(K_U^\intercal S)\ast (V^\intercal S)}\bigl) \\
                \mathbf{J_S^\intercal J_V \text{vec}(K_v)} & = & \scriptstyle 3\cdot \text{vec}\bigl(\mathbf{K_V(U^\intercal S)\ast (W^\intercal S) + U(K_V^\intercal S)\ast (W^\intercal S) + W(K_V^\intercal S)\ast (U^\intercal S)}\bigl) \\
                \mathbf{J_S^\intercal J_W \text{vec}(K_w)} & = & \scriptstyle 3\cdot \text{vec}\bigl(\mathbf{K_W(U^\intercal S)\ast (V^\intercal S) + U(K_W^\intercal S)\ast (V^\intercal S) + V(K_W^\intercal S)\ast (U^\intercal S)}\bigl) \\ \\
                \mathbf{J_U^\intercal J_S \text{vec}(K_s)} & = & \scriptstyle 3\cdot \text{vec}\bigl(\mathbf{K_S(S^\intercal V)\ast (S^\intercal W) + S\bigl((K_S^\intercal W)\ast (S^\intercal V) + (K_S^\intercal V)\ast (S^\intercal W)\bigl)}\bigl) \\
                \mathbf{J_U^\intercal J_U \text{vec}(K_u)} & = & \scriptstyle 3\cdot \text{vec}\bigl(\mathbf{K_U(V^\intercal V)\ast (W^\intercal W) + V(K_U^\intercal W)\ast (W^\intercal V) + W(K_U^\intercal V)\ast (V^\intercal W)}\bigl) \\
                \mathbf{J_U^\intercal J_V \text{vec}(K_v)} & = & \scriptstyle 3\cdot \text{vec}\bigl(\mathbf{K_V(W^\intercal V)\ast (U^\intercal W) + W(K_V^\intercal W)\ast (U^\intercal V) + U(K_V^\intercal V)\ast (W^\intercal W)}\bigl) \\
                \mathbf{J_U^\intercal J_W \text{vec}(K_w)} & = & \scriptstyle 3\cdot \text{vec}\bigl(\mathbf{K_W(U^\intercal V)\ast (V^\intercal W) + U(K_W^\intercal W)\ast (V^\intercal V) + V(K_W^\intercal V)\ast (U^\intercal W)}\bigl) \\ \\
                \mathbf{J_V^\intercal J_S \text{vec}(K_s)} & = & \scriptstyle 3\cdot \text{vec}\bigl(\mathbf{K_S(S^\intercal U)\ast (S^\intercal W) + S\bigl((K_S^\intercal U)\ast (S^\intercal W) + (K_S^\intercal W)\ast (S^\intercal U)\bigl)}\bigl) \\
                \mathbf{J_V^\intercal J_U \text{vec}(K_u)} & = & \scriptstyle 3\cdot \text{vec}\bigl(\mathbf{K_U(V^\intercal W)\ast (W^\intercal U) + V(K_U^\intercal U)\ast (W^\intercal W) + W(K_U^\intercal W)\ast (V^\intercal U)}\bigl) \\
                \mathbf{J_V^\intercal J_V \text{vec}(K_v)} & = & \scriptstyle 3\cdot \text{vec}\bigl(\mathbf{K_V(W^\intercal W)\ast (U^\intercal U) + W(K_V^\intercal U)\ast (U^\intercal W) + U(K_V^\intercal W)\ast (W^\intercal U)}\bigl) \\
                \mathbf{J_V^\intercal J_W \text{vec}(K_w)} & = & \scriptstyle 3\cdot \text{vec}\bigl(\mathbf{K_W(U^\intercal W)\ast (V^\intercal U) + U(K_W^\intercal U)\ast (V^\intercal W) + V(K_W^\intercal W)\ast (U^\intercal U)}\bigl) \\ \\
                \mathbf{J_W^\intercal J_S \text{vec}(K_s)} & = & \scriptstyle 3\cdot \text{vec}\bigl(\mathbf{K_S(S^\intercal U)\ast (S^\intercal V) + S\bigl((K_S^\intercal U)\ast (S^\intercal V) + (K_S^\intercal V)\ast (S^\intercal U)\bigl)}\bigl) \\
                \mathbf{J_W^\intercal J_U \text{vec}(K_u)} & = & \scriptstyle 3\cdot \text{vec}\bigl(\mathbf{K_U(V^\intercal U)\ast (W^\intercal V) + V(K_U^\intercal V)\ast (W^\intercal U) + W(K_U^\intercal U)\ast (V^\intercal V)}\bigl) \\
                \mathbf{J_W^\intercal J_V \text{vec}(K_v)} & = & \scriptstyle 3\cdot \text{vec}\bigl(\mathbf{K_V(W^\intercal W)\ast (U^\intercal V) + W(K_V^\intercal V)\ast (U^\intercal U) + U(K_V^\intercal U)\ast (W^\intercal V)}\bigl) \\
                \mathbf{J_W^\intercal J_W \text{vec}(K_w)} & = & \scriptstyle 3\cdot \text{vec}\bigl(\mathbf{K_W(U^\intercal U)\ast (V^\intercal V) + U(K_W^\intercal V)\ast (V^\intercal U) + V(K_W^\intercal U)\ast (U^\intercal V)}\bigl) \\
        \end{array}
    \end{equation*}

    Then, we reach our algorithm
    \begin{algorithm}
        \caption{Cyclic Invariant CP Damped Gauss-Newton}
        \label{alg:Mod-CP-DGN}
        \begin{algorithmic}
            \State{\textbf{Input:} Matrix Multiplication Tensor $\mathcal{M}$,}
            \State{\hspace{3.25em} CP Tensor Rank $r$,}
            \State{\hspace{3.25em} Damping Parameter $\lambda \in \mathbb{R}$,}
            \State{\hspace{3.25em} Convergence Tolerance $\epsilon > 0$}
            
            \State \textbf{Output:} CP Tensor $\mathcal{K}$
            \Function{DGN}{$\mathcal{M}, r, \lambda, \epsilon$}
                \State{Initialize K and K\_prev to be a cell of length 4 with the first entry being of $n^2 \times r_s$ and the remaining three $n^2 \times r_c$ matrices}

                \For{$i = 1:MaxIters$}
                    \State{f $\longleftarrow \frac{1}{2} \|\mathcal{M} - \mathbf{\llbracket S, S, S\rrbracket - \llbracket U, V, W\rrbracket - \llbracket W, U, V\rrbracket - \llbracket V, W, U\rrbracket} \| $}
                    \State{$\nabla \mathbf{f} \longleftarrow [\text{vec} \left( \frac{\partial f}{\partial \mathbf S}\right) \text{vec} \left( \frac{\partial f}{\partial \mathbf U}\right) \text{vec} \left( \frac{\partial f}{\partial \mathbf V}\right) \text{vec} \left( \frac{\partial f}{\partial \mathbf W}\right)]^T$} 
                    \State{$S \longleftarrow \text{Solution to } (\textbf{J}^T\textbf{J} + \lambda I)\mathbf{K} = -\nabla \mathbf{f}$}
                    \While{\textit{Goldstein Conditions Are Satisfied}}
                        \State{K $\longleftarrow$ K\_prev + $\alpha S$}
                        \State{f $_{new}\longleftarrow$ Compute Function Value}
                        \State{$\alpha \longleftarrow \alpha / 2$}
                    \EndWhile{}
                    \If{f - f $_{new} < \epsilon$}
                        \State{\textbf{break}}
                    \EndIf{}
                \EndFor{}
            \EndFunction
        \end{algorithmic}
    \end{algorithm}