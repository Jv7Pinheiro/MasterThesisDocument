%!TEX root = ../../main.tex

\newcommand{\datapath}{}
\newcommand{\errone}{}
\newcommand{\errtwo}{}
\newcommand{\errthree}{}
\newcommand{\error}{}
\newcommand{\test}{}
\newcommand{\thresh}{}

This section presents a comparison of the
running time (strong scaling and running time breakdown) and compression (error
vs. time and error vs. compression ratio) performance of the various Tucker
algorithms presented in this work. All algorithms were implemented using the
TuckerMPI (C++/OpenMPI) library \cite{BKK20}.

\paragraph{Computing platform.} Our experiments were conducted on
NERSC Perlmutter (CPU partition). The system consists of 3072 compute nodes with
dual-socket AMD EPYC 7763 64-core CPUs. Each socket has 4 Non-Uniform Memory
Access (NUMA) regions for a total of 8 NUMA regions per node. Each NUMA region
has 64 GB of DRAM memory, therefore each CPU socket has 256 GB of DRAM for, a
total of 512 GB of memory per node.
% Each NUMA region can communicate with all the cores in its node, but it
% communicates the fastest with the 16 cores in its socket.

\paragraph{Experiments.} We perform experiments on synthetic tensors that are
randomly generated and tensors obtained from real applications. We use 3-way and
4-way tensors for the synthetic experiments, and three real datasets: Miranda
\cite{KD+20} (3-way), HCCI \cite{BCL14} (4-way), and SP \cite{KZCS16} (5-way).
The real datasets are described in more detail in \cref{sec:high_nr,sec:low_nr}.
Experiments performed on synthetic tensors are performed in single precision,
while experiments on real datasets are performed in single or double precision
depending on their storage precision on disk. Strong scaling experiments are
performed on the synthetic tensors. We show running time breakdown of both real
and synthetic experiments. For synthetic tensors we show the running time
breakdown at small and large scale to highlight how each step in a given
algorithm scales. For real tensors we vary the error tolerance and starting
ranks to show how performance breakdowns vary. Compression performance
experiments are performed only on the real datasets.
% two sets of experiments: strong scaling on
% synthetic tensors using single precision, and a comparative performance against
% the state-of-the art using the three data sets.

Even for a fixed number of processors $P$, the $d$-way processor grid has a
significant effect on all algorithms. As described in \cref{sec:TuckerMPI's STHOSVD},
STHOSVD benefits from processor grids with $P_1=1$, and HOOI variants using
dimension trees are theoretically more efficient when $P_1=P_d=1$. In addition,
for modes with small tensor dimension, a large processor dimension in that mode
may cause load imbalance due to uneven division. In all experiments, we test all
algorithms on a variety of grids, including those we expect to benefit
individual algorithms, and we report the fastest observed running times.
%For the most part,
%HOOI and HOSI prefer grids that have 1 in the last mode, HOOI-DT and HOSI-DT
%prefer algorithms that have 1 in both the first and last mode, and STHOSVD
%prefers grids that have 1 in the first mode. 
%The occasions where such pattern
%not result in their best performance for a certain experiment is when a
%processor grid mode greatly exceeds the input tensor or core tensor size for
%that mode since that creates work imbalance. 
%For example, in
%\cref{sec:synthetic_strong_scaling} for both the 3-way and 4-way experiments,
%HOSI-DT's best times came from grids that had a 1 in the first and last mode up
%to 512 cores, afterwards, it preferred grids that more evenly split the
%workload. From 1024 to 4096 cores HOSI-DT prefers grids with a 2 in the first
%and last mode. At 8192, HOSI-DT also preferred the grid that had a 2 in the
%first and last modes, but for the 3-way, it preferred grids that more evenly
%split the work. 

\subsection{Strong Scaling on Synthetic Tensors} \label{sec:synthetic_strong_scaling} 

    First, we present strong scaling
    experiments on the 3-way and 4-way synthetic tensors to demonstrate the parallel
    scaling of HOOI, HOOI-DT, HOSI, HOSI-DT, and STHOSVD. We choose tensor
    dimensions to maximize the size of the tensor that can fit on a single node (in
    single precision).
    % We present an experiment on synthetic tensors that demonstrates the parallel
    % scaling capabilities of each algorithm. Specifically, we wish to demonstrate
    % scaling of HOOI, HOOI-DT, HOSI, HOSI-DT, and STHOSVD on two synthetic
    % tensors, a 3-way and a 4-way tensor. The dimensions of each tensor were
    % chosen so that maximize the size of the tensor on single precision such that
    % the input tensor and its tucker decomposition could all fit on a single
    % node. The HOOI algorithms are run on two iterations. Since STHOSVD is not
    % an iterative algorithm, we informally refer to its runtime as a `single
    % iteration'.

    For synthetic input, we generate tensors by forming a Tucker-format tensor of specified rank and adding a specified level of noise.
    Thus, these experiments are performed for the rank-specified formulation of the Tucker approximation problem to recover the input. 
    We run for two iterations for each variant of HOOI even though we often have a sufficiently accurate approximation after a single iteration.
    We include overhead due to core analysis for the error-specified formulation in the experiments on the real datasets.
    % the rank-specified variations of all five algorithms, the input ranks are the
    % same as the ranks used to generate the tensor. In other words, we are still
    % not testing core analysis on the HOOI algorithms, and we are using the
    % rank-specified version of STHOSVD.
    The largest 3-way tensor that fits into single-node memory is a
    tensor of size $3750 \times 3750 \times 3750$. 
    We generate this tensor to
    have a rank of $30$ in all modes.
    % by multiplying factor matrices $\Mx{A}{i} \in \Rmsiz{3750, 30}$ into the core
    % tensor $\Tn{G}\in \Rmsiz{30, 30, 30}$.
    %Here, $\frac{n}{r} = 125$, and the compression ratio is $1,953,125$. 
    Similarly, we construct the 4-way tensor of size $560 \times 560 \times 560\times 
    560$ with Tucker ranks $(10,10,10,10)$.

    \begin{figure}
        \centering
        \renewcommand{\datapath}{../../data}

\begin{tikzpicture}
    \begin{axis}[
        width =0.65\textwidth,
        height=0.65\textwidth,
        xlabel={Number of Cores},
        ylabel={Time},
        ylabel near ticks,
        xlabel near ticks,
        ymode=log,
        log basis y={2},
        xmode=log,
        log basis x={2},
        xtick={1,2,4,8,16,32,64,128,256,512,1024,2048,4096,8192},
        xticklabels={1,2,4,8,16,32,64,128,256,512,1024,2048,4096,8192},
        ytick={2^(-1),1,2,4,8,16,32,64,128,256,512,1024,2048},
        x tick label style={rotate=45, anchor=east, font=\small},
        grid=both,
        legend pos=outer north east,
        % legend style={draw=none, cells={anchor=west}, font=\small},
        title={3 way},
    ]

        \addplot[color=red,mark=square,mark size=1.5pt]
        	table[x=nprocs, y=HOOI_totals, col sep=comma] {data/Cores_3way_3750_30_Single/Total_Times.csv};
         \addlegendentry{HOOI}
    
        \addplot[color=blue,mark=square,mark size=1.5pt]
        	table[x=nprocs, y=HOOI_DT_totals, col sep=comma] {/Users/joaodeoliveria/Documents/Thesis/MasterThesisDocument/data/Cores_3way_3750_30_Single/Total_Times.csv};
         \addlegendentry{HOOI-DT}
    
        \addplot[color=orange,mark=triangle,mark size=1.5pt]
        	table[x=nprocs, y=HOSI_totals, col sep=comma] {/Users/joaodeoliveria/Documents/Thesis/MasterThesisDocument/data/Cores_3way_3750_30_Single/Total_Times.csv};
         \addlegendentry{HOSI}
    
        \addplot[color=purple,mark=triangle,mark size=1.5pt]
        	table[x=nprocs, y=HOSI_DT_totals, col sep=comma] {/Users/joaodeoliveria/Documents/Thesis/MasterThesisDocument/data/Cores_3way_3750_30_Single/Total_Times.csv};
         \addlegendentry{HOSI-DT}
    
        \addplot[color=green,mark=o,mark size=1.5pt]
        	table[x=nprocs, y=STHOSVD_totals, col sep=comma] {/Users/joaodeoliveria/Documents/Thesis/MasterThesisDocument/data/Cores_3way_3750_30_Single/Total_Times.csv};
         \addlegendentry{STHOSVD}

    \end{axis}
\end{tikzpicture}
        \caption[3-way Strong Scaling]{Strong scaling comparison of Tucker
        algorithms in single precision using a 3-way $\mathbf{3750 \times 3750
        \times 3750}$ input tensor }
        \label{fig:3way_scaling}
    \end{figure}

    \begin{figure}
        \centering
        \renewcommand{\datapath}{../../data}

\begin{tikzpicture}
    \begin{axis}[
        width =0.65\textwidth,
        height=0.65\textwidth,
        xlabel={Number of Cores},
        ylabel={Time},
        ylabel near ticks,
        xlabel near ticks,
        ymode=log,
        log basis y={2},
        xmode=log,
        log basis x={2},
        xtick={1,2,4,8,16,32,64,128,256,512,1024,2048,4096,8192},
        xticklabels={1,2,4,8,16,32,64,128,256,512,1024,2048,4096,8192},
        ytick={2^(-1),1,2,4,8,16,32,64,128,256,512,1024,2048},
        x tick label style={rotate=45, anchor=east, font=\small},
        grid=both,
        legend pos=outer north east,
        % legend style={draw=none, cells={anchor=west}, font=\small},
        title={4 way},
    ]

        \addplot[color=red,mark=square,mark size=1.5pt]
        	table[x=nprocs, y=HOOI_totals, col sep=comma] {data/Cores_4way_560_10_Single/Total_Times.csv};
         \addlegendentry{HOOI}
    
        \addplot[color=blue,mark=square,mark size=1.5pt]
        	table[x=nprocs, y=HOOI_DT_totals, col sep=comma] {data/Cores_4way_560_10_Single/Total_Times.csv};
         \addlegendentry{HOOI-DT}
    
        \addplot[color=orange,mark=triangle,mark size=1.5pt]
        	table[x=nprocs, y=HOSI_totals, col sep=comma] {data/Cores_4way_560_10_Single/Total_Times.csv};
         \addlegendentry{HOSI}
    
        \addplot[color=purple,mark=triangle,mark size=1.5pt]
        	table[x=nprocs, y=HOSI_DT_totals, col sep=comma] {data/Cores_4way_560_10_Single/Total_Times.csv};
         \addlegendentry{HOSI-DT}
    
        \addplot[color=green,mark=o,mark size=1.5pt]
        	table[x=nprocs, y=STHOSVD_totals, col sep=comma] {data/Cores_4way_560_10_Single/Total_Times.csv};
         \addlegendentry{STHOSVD}

    \end{axis}
\end{tikzpicture}
        \caption[4-way Strong Scaling]{Strong scaling comparison of Tucker algorithms in single
        precision using a 4-way $\mathbf{560 \times 560 \times 560 \times 560}$
        input tensor}
        \label{fig:4way_scaling}
    \end{figure}

    Figures \ref{fig:3way_scaling} and \ref{fig:4way_scaling} shows the strong
    scaling results of the HOOI variants and STHOSVD on up to $4096$ cores for
    the 3-way and 4-way synthetic datasets. We observe that STHOSVD scales well
    to $64$ cores, attaining a speedup of $15.2\times$ over the single core
    STHOSVD run. STHOSVD continues to scale up to $2048$ cores, but achieves
    only a modest speedup of $1.3\times$ over the $64$ core run. This is due to
    TuckerMPI's limitation of having a sequential EVD implementation. In
    contrast, the 4-way STHOSVD strong scaling experiment shows good scaling up
    to $8192$ cores, achieving a speedup of $937\times$ over the single core
    run. This difference in STHOSVD performance is explained by the tensor
    dimension: a sequential EVD of a matrix of dimension 560 does not become the
    bottleneck until $P$ is large.
    %The high $\frac{n}{r}$ of the 3-way experiment makes it so that LLSV's
    %eigenvalue decomposition subroutine is the limiting factor, which is made
    %worse due to the sequential implementation. This is not the same as on the
    %4-way case as the lower $\frac{n}{r}$ makes the TTMs be the bottleneck.

    When comparing the two HOOI variants (which use Gram SVD), we observe that HOOI-DT yields a
    sequential speedup of $1.4\times$ over HOOI's direct TTM implementation for
    the 3-way tensor. For the 4-way tensor, HOOI-DT achieves a sequential
    speedup of $5.4\times$ faster than HOOI. When comparing parallel scaling in
    the 3-way case, we see that HOOI and HOOI-DT scale to $16$ cores with a
    speedup of $3.5\times$ and $2.8\times$, respectively, over their single core
    runs. However, neither variant scales beyond $16$ cores for the 3-way
    tensor because of the sequential EVD bottlenecks. 
    For the 4-way tensor, HOOI and HOOI-DT scale to $8192$ cores with a
    speedup of $629\times$ and $346\times$, respectively, over their single core
    runs.
    % However, neither HOOI variant achieves a speedup over STHOSVD at $8192$ cores.
    The performance of HOOI and HOOI-DT degrades at $128$ cores (single node)
    because both variants are memory-bandwidth bound, and we saturate bandwidth
    at $64$ cores. HOOI and HOOI-DT continue scaling beyond $128$ cores
    (multi-node scaling) because memory bandwidth increases. 
    As can be seen in the 4096 core plots of \cref{fig:scaling_breakdown},
    HOOI and HOOI-DT suffer from the problem of the sequential EVD, and they are approximately twice as
    slow as STHOSVD because they do twice as many EVDs over two iterations.


    HOSI and HOSI-DT show significantly better scaling on the 3-way tensor when
    compared to STHOSVD and the HOOI variants because of the difference in LLSV
    subroutines. HOSI-DT achieves sequential speedups of $6.5\times$ and
    $1.7\times$ over STHOSVD and HOOI-DT, respectively. The HOSI variants scale
    to 4096 cores with HOSI-DT achieving significant parallel speedups of
    $259\times$ and $515\times$ over STHOSVD and HOOI-DT, respectively. HOSI-DT
    is also the fastest Tucker variant for the 4-way experiment attaining
    speedups of $1.5\times$ and $2.9\times$ over STHOSVD and HOOI-DT,
    respectively when comparing the best running times of each algorithm. HOSI
    and HOSI-DT exhibit similar memory bandwidth scaling behavior as the HOOI
    variants where performance degrades at $128$ cores (single node) and
    continues to scale beyond $128$ cores (multi-node scaling).
    These can be seen on \Cref{fig:scaling_breakdown}. We chose to showcase the
    breakdown using 1 core and using 4096 cores.

    \begin{figure}
        \centering
        \newcommand{\itermax}{2}
\pgfmathsetmacro{\n}{\itermax - 1}
\renewcommand{\datapath}{data}

\newcommand{\ngp}[3]{
\nextgroupplot[title={#1: #3 Core(s)}]
    \foreach \i in {0, ..., \n} {
        \addplot[
            ybar,
            color=blue,
            fill=blue!25
            ] 
            table[x=Algorithms, y=TTM_Computation_Iter\i, col sep=comma] {\datapath/Cores_#1_#2/N#3_summary.csv};
        \addplot[
            ybar,
            color=blue,
            fill=blue!50
            ] 
            table[x=Algorithms, y=TTM_Communication_Iter\i, col sep=comma] {\datapath/Cores_#1_#2/N#3_summary.csv};
        \addplot[
            ybar,
            color=red,
            fill=red!25
            ] 
            table[x=Algorithms, y=GramLQ_Computation_Iter\i, col sep=comma] {\datapath/Cores_#1_#2/N#3_summary.csv};
        \addplot[
            ybar,
            color=red,
            fill=red!50
            ] 
            table[x=Algorithms, y=GramLQ_Communication_Iter\i, col sep=comma] {\datapath/Cores_#1_#2/N#3_summary.csv};
        \addplot[
            ybar,
            color=yellow,
            fill=yellow!50
            ] 
            table[x=Algorithms, y=EigSVD_Iter\i, col sep=comma] {\datapath/Cores_#1_#2/N#3_summary.csv};
    }
}

\begin{tikzpicture}
    \begin{groupplot}[
        group style={
            group size=2 by 2,
            horizontal sep=2cm,
            vertical sep=2cm,
        },
        width=0.4\textwidth,
        height=0.4\textwidth,
        ylabel near ticks,
        xlabel near ticks,
        ybar stacked,
        xtick=data,
        xticklabels={HOOI, HOOI-DT, HOSI, HOSI-DT, STHOSVD},
        symbolic x coords={HOOI, HOOI-DT, HOSI, HOSI-DT, STHOSVD},
        ylabel={Time},
        y label style = {yshift=-.1cm},
        %xlabel={Algorithms},
        ymin=0,
        grid=both,
        x tick label style={rotate=30, anchor=east, yshift=-.1cm},
        title style={yshift=-1.5ex},
    ]
    
    % First plot
    % \nextgroupplot[title=Plot 1: Hello]
    % \addplot[blue, thick] {x^2};
    \ngp{3way}{3750_30_Single}{1}
    
    % % % Second plot
    % % \nextgroupplot[title=Plot 2: World, ylabel={}]
    % % \addplot[red, thick] {sin(deg(x))};
    \ngp{3way}{3750_30_Single}{4096}

    % % % First plot
    % % \nextgroupplot[title=Plot 1: Hello]
    \ngp{4way}{560_10_Single}{1}
    
    % % % Second plot
    % % \nextgroupplot[title=Plot 2: World, ylabel={}]
    \ngp{4way}{560_10_Single}{4096}
    
    \end{groupplot}
\end{tikzpicture}
        \caption{Running time breakdown for the synthetic 3-way (top) and 4-way (bottom) tensors}
        \label{fig:scaling_breakdown}
    \end{figure}
    
    \paragraph{3-way.} Observing the single-node scaling (1 to 128) of the 3-way
    experiment, we notice that all HOOI algorithms outperform STHOSVD, with
    HOSI and HOSI-DT being much further ahead of the competition since the large
    $\frac{n}{r}$ ratio of this experiment implies a LLSV bottleneck and these
    two algorithms avoid that. Here, STHOSVD is much slower because it does a
    more expensive LLSVs before the TTMs whereas the HOOI algorithms reduce this
    cost by performing the TTMs beforehand. Starting from 32 nodes, STHOSVD
    begins to outperform HOOI and HOOI-DT due to the cost of the LLSV.
    \Cref{fig:scaling_breakdown} demonstrates that the cost of LLSV is the same
    for HOOI, HOOI-DT and STHOSVD, but because HOOI must perform two
    iterations, it takes twice as long. In fact, we see that the three
    algorithms stagnate at large scaling due to TuckerMPI's limitation of having
    a sequential eigenvalue computation. Though HOSI and HOSI-DT still perform
    two iterations, neither of them need to pay the cost of the sequential
    eigenvalue decomposition. Even with a sequential QR decomposition, they
    still scale best, with the lesser TTM cost of HOSI-DT making it the best out
    of these five algorithms.

    For the 3-way synthetic tensor on 1 core, it can be seen that the cost of
    the Gram computation is the bottleneck for STHOSVD. The TTM cost for the
    dimension tree algorithms are cheaper, as expected. HOSI-DT is faster than
    HOOI-DT simply because of the smaller cost for the LLSV computations. The
    two iterations for all HOOI algorithms are faster than the `single
    iteration' for STHOSVD. However, that is not the case for the 4096 cores
    experiment. Now, the TTM costs for all algorithms are negligible due to the
    parallel scaling. The Gram computation for STHOSVD is also negligible now
    for the same reasons. The bottleneck at this scale is now the Eigenvalue
    computation. The reason why HOOI, HOOI-DT, and STHOSVD stagnate over the
    high number of cores on \Cref{fig:3way_scaling} is because of TuckerMPI's
    limitation of having a sequential eigenvalue computation, and the reason why
    HOOI and HOOI-DT are twice as expensive on multi-node experiments is because
    they must do two iterations, this fact can be visualized on the breakdown
    for the 4096 cores experiment.

    \paragraph{4-way.} In this experiment, HOOI-DT and HOSI-DT are the ones that
    get a comparative headstart since the small $\frac{n}{r}$ ratio of this
    experiment implies a TTM bottleneckm and these two algorithms avoid that.
    They diverge around 128 cores for the same reasons mentioned above as
    HOOI-DT must pay the cost of two expensive eigenvalue computation. For that
    matter, STHOSVD still eventually catches up to HOOI-DT, but now this
    happens at 512 cores as opposed to 32. For this reason, HOSI eventually
    closes the gap to HOSI-DT, with STHOSVD being not too far behind simply
    because of the smaller $\frac{n}{r}$.

    \paragraph{Overall} There are two factors that are pertinent to both 3-way and 4-way
    experiments. The first is the spike noticible at 128 for most (some more
    than others) of the algorithms. These come from a choice of configuration of
    the slurm jobs regarding the binding of NUMA regions to the cores. There
    were two main options available to run these experiments with,
    \textit{cores} and \textit{map\_ldom}. The former assigns all the NUMA
    regions of the node to the running cores, and the latter we most perform a
    manual assignment of the NUMA regions which gives us more control. We are
    not utilizing the full potential of a node from 1 to 64 cores, as there are
    more cores available, but the \textit{cores} configuration still assigns all
    the NUMA regions of the node to the running cores, so these cores have more
    memory than what is `pertinent' to them. Thus, when we first use the full
    potential of a node at 128 cores, there is now a competition for resources.
    Running the \textit{map\_ldom} configuration allows us to bypass that
    limitation, so from 1 to 64 cores we only assign NUMA regions that would be
    `pertinent' to those cores assuring a steady parallel scaling in terms of
    memory. This makes the experiments from 1 to 32 nodes equally slower to all
    five algorithms. In other words it removes the spike, simply because the
    rate of increase is steady now. The other factor regards why HOSI and
    HOSI-DT stop scaling at 8192 cores, the reason is 42. 

\subsection{Performance on Simulation Datasets} \label{sec:performance_comparisons} 

    We turn our focus for the error-specified comparison of our best algorithm,
    HOSI-DT, and the state-of-the-art, STHOSVD. The data sets are decomposed
    using three error tolerances; 0.1 (``high compression''), 0.05 (``mid compression''), and 0.01 (``low compression''). 
    Furthermore, we
    showcase HOSI-DT through three different types of starting ranks for each
    error tolerance. Perfect starting ranks are the same as the final ranks of
    STHOSVD given the maximum relative error threshold. We overshoot and
    undershoot the same starting ranks by 25\% above and below to force our
    algorithm to respectively increase and decrease ranks on the first
    iteration. We cap the number of iterations for HOSI-DT at 3. Though all
    three iterations are shown in the Error vs Time and Error vs Size plots, the
    running time breakdown plots show  the breakdown only for however many
    iterations it took for HOSI-DT to reach the desired error threshold. For
    example, the top right plot of \cref{fig:HCCI_breakdown} it can be seen that
    the HOSI-DT (Over) 0.1 threshold reached the desired at the first iteration,
    so we don't show the breakdown for the second iteration despite its total
    time being shown on \cref{fig:HCCI}.

    \subsubsection{Miranda (3-way)}\label{sec:high_nr}
        The Miranda dataset is a three-dimensional simulation data of the
        density ratios of non-reacting flow of viscous fluids \cite{KD+20}. Each
        of its dimensions is 3072, and it is stored in single
        precision requiring 115 GB.
        Our experiments use 1024 cores (8 nodes) for all algorithms.

        \begin{figure}
            \centering
            \renewcommand{\datapath}{data/Miranda/N8_n1024}
            \renewcommand{\thresh}{1e-01,5e-02,1e-02}
            %!TEX root = ../../main.tex

% need to define \datapath and \thresh (set of error thresholds)

\newcommand{\algorithmname}{HOSI_DT}

\pgfplotsset{HOOIstyle/.style={
	red,
	mark=o,
	point meta = {\thisrow{STHOSVD_Size}/\thisrow{HOSI_DT_Size}},
	every node near coord/.append style={
		above right,
		inner sep=1pt,
		yshift=1ex
	}
}}
\pgfplotsset{STHOSVDstyle/.style={
	blue,
	mark=square*,
	only marks
}}


\begin{tikzpicture}
    \begin{groupplot}[
        group style={
            group size=2 by 1,
            xlabels at=edge bottom,
            ylabels at=edge left,
            horizontal sep=2cm,
            vertical sep=2cm,
        },
        width= 0.45\textwidth,
        height=0.45\textwidth,
        ylabel=Relative Error,
        ylabel near ticks,
        xlabel near ticks,
        ymajorgrids=true,
        legend to name=namedlegend,
        legend style={draw=none, cells={anchor=west}, font=\small,
        legend columns=2},
        title style={yshift=-1.5ex},
      ]
        % First plot: error vs time
        \nextgroupplot[
            title=Error vs Time, 
            xlabel=Time (in seconds),
            ytick = \thresh,
            yticklabel style={/pgf/number format/fixed}
        ]

	\foreach \t in \thresh {
		\addplot [HOOIstyle,mark=triangle] table[x=\algorithmname_Time, y=\algorithmname_Error, col sep=comma] {\datapath_\t_Over.csv};
        		\addplot [HOOIstyle,mark=star] table[x=\algorithmname_Time, y=\algorithmname_Error, col sep=comma] {\datapath_\t_Perfect.csv};
        		\addplot [HOOIstyle,mark=square] table[x=\algorithmname_Time, y=\algorithmname_Error, col sep=comma] {\datapath_\t_Under.csv};
        		\addplot [STHOSVDstyle] table[x=STHOSVD_Time, y=STHOSVD_Error, col sep=comma] {\datapath_\t_Under.csv};	
	}
      
        % Second plot: error vs relative size
        \nextgroupplot[
            title=Error vs Size,
            xlabel=Size (relative to STHOSVD),
            ytick = \thresh,
            yticklabel style={/pgf/number format/fixed}
        ]
        
        \foreach \t in \thresh {
        		\addplot[HOOIstyle,mark=triangle] table [x expr = (\thisrow{HOSI_DT_Size}/\thisrow{STHOSVD_Size}), y=HOSI_DT_Error, col sep=comma] {\datapath_\t_Over.csv};
        		\addplot[HOOIstyle,mark=star] table [x expr = (\thisrow{HOSI_DT_Size}/\thisrow{STHOSVD_Size}), y=HOSI_DT_Error, col sep=comma] {\datapath_\t_Perfect.csv};
        		\addplot[HOOIstyle,mark=square] table [x expr = (\thisrow{HOSI_DT_Size}/\thisrow{STHOSVD_Size}), y=HOSI_DT_Error, col sep=comma] {\datapath_\t_Under.csv};
        		\addplot[STHOSVDstyle] table [x expr = (\thisrow{STHOSVD_Size}/\thisrow{STHOSVD_Size}), y=STHOSVD_Error, col sep=comma] {\datapath_\t_Under.csv};
        }

        \addlegendimage{mark=triangle, color=red}
        \addlegendentry{HOSI-DT Over}
        \addlegendimage{mark=star, color=red}
        \addlegendentry{HOSI-DT Perfect}
        \addlegendimage{mark=square, color=red}
        \addlegendentry{HOSI-DT Under}
        \addlegendimage{mark=square*, color=blue, only marks}
        \addlegendentry{STHOSVD}
    \end{groupplot}
\end{tikzpicture}

% Place the legend below the figure
\begin{center}
    \ref{namedlegend}
\end{center}

 

            \caption{Progression of time, error, and relative size over 3 iterations of rank-adaptive HOSI-DT on the Miranda dataset using 1024 cores.}
            \label{fig:miranda}
        \end{figure}

        \begin{figure}
            \centering
            \renewcommand{\datapath}{data/Miranda/N8_n1024}
            \renewcommand{\thresh}{1e-01,5e-02,1e-02}
            %!TEX root = ../../main.tex


\newcommand{\maxiter}{3}
\pgfmathsetmacro{\n}{\maxiter - 1}

\begin{tikzpicture}
    \begin{axis}[
        hide axis, % Hide the dummy axis
        axis lines=none,
        xmin=0, xmax=1,
        ymin=0, ymax=1,
        width= 0.45\textwidth,
        height=0.45\textwidth,
        legend style={fill=white, at={(2.25, -0.515)}, anchor=north east, font=\small},
        legend columns=1, % Arrange legend in one column
        reverse legend
    ]   
        \addplot [color=white, fill=white, mark=square*,opacity=0] coordinates {(0,0)};
        \addlegendentry{}

        \addlegendimage{mark=square*, color=blue!25, fill=blue!25, only marks, mark options={scale=2.5, fill=blue!25}}
        \addlegendentry{TTM Comp}

        \addlegendimage{mark=square*, color=blue!50, fill=blue!50, only marks, mark options={scale=2.5, fill=blue!50}}
        \addlegendentry{TTM Comm}

        \addlegendimage{mark=square*, color=red!25, fill=red!25, only marks, mark options={scale=2.5, fill=red!25}}
        \addlegendentry{Contraction Comp}

        \addlegendimage{mark=square*, color=red!50, fill=red!50, only marks, mark options={scale=2.5, fill=red!50}}
        \addlegendentry{Contraction Comm}

        \addlegendimage{mark=square*, color=yellow!50, fill=yellow!50, only marks, mark options={scale=2.5, fill=yellow!50}}
        \addlegendentry{Eig/QR}

        \addlegendimage{mark=square*, color=green!25, fill=green!25, only marks, mark options={scale=2.5, fill=green!25}}
        \addlegendentry{Core Comp}
        
        \addlegendimage{mark=square*, color=green!50, fill=green!50, only marks, mark options={scale=2.5, fill=green!50}}
        \addlegendentry{Core Comm}
    \end{axis}

    \begin{groupplot}[
        group style={
            group size=2 by 2,
            horizontal sep=2cm,
            vertical sep=2cm,
        },
        ybar stacked,
        symbolic x coords={HOSI-DT-Under, HOSI-DT-Perfect, HOSI-DT-Over, STHOSVD},
        xticklabels={Under, Perfect, Over, STHOSVD},
        xtick=data,
        ylabel={Time},
        ylabel style = {yshift=-.675cm},
        ymajorgrids=true,
        width=0.45\textwidth,
        height=0.45\textwidth,
        x tick label style={rotate=30, anchor=east},
        title style={yshift=-1.5ex},
    ]

    \xintFor #1 in \thresh \do {
        \nextgroupplot[title = {#1 Error}]
        \foreach \i in {0, ..., \n} {    
            \addplot[ybar,color=blue,fill=blue!25] 
            		table[x=Algorithm, y=TTM_Computation_Iter_\i, col sep=comma] {\datapath_#1_Breakdown.csv};  
		
            \addplot[ybar,color=blue,fill=blue!50] 
            		table[x=Algorithm, y=TTM_Communication_Iter_\i, col sep=comma] {\datapath_#1_Breakdown.csv};
		
            \addplot[ybar,color=red,fill=red!25] 
            		table[x=Algorithm, y=GramLQ_Computation_Iter_\i, col sep=comma] {\datapath_#1_Breakdown.csv};
		
            \addplot[ybar,color=red,fill=red!50] 
    		table[x=Algorithm, y=GramLQ_Communication_Iter_\i, col sep=comma] {\datapath_#1_Breakdown.csv};
		
            \addplot[ybar,color=yellow,fill=yellow!50] 
    		table[x=Algorithm, y=EigSVD_Iter_\i, col sep=comma] {\datapath_#1_Breakdown.csv};
		
            \addplot[ybar,color=green,fill=green!25] 
    		table[x=Algorithm, y=Core_Analysis_Iter_\i, col sep=comma] {\datapath_#1_Breakdown.csv};
		
            \addplot[ybar,color=green,fill=green!50] 
    		table[x=Algorithm, y=Core_Gathering_Iter_\i, col sep=comma] {\datapath_#1_Breakdown.csv};
        }
    }

\end{groupplot}
\end{tikzpicture}

            \caption{Running time breakdown for the Miranda dataset using 1024 cores under different levels of compression.}
            \label{fig:miranda_breakdown}
        \end{figure}

        %% Just need to remove the second half of the paragraph. 
        \Cref{fig:miranda} demonstrates that for all error tolerances, three
        iterations of HOSI-DT combined is faster than STHOSVD. But as mentioned
        earlier, we focus on the least amount of iterations required to reach
        the desired error threshold. It is in high- and
        mid-compression where we find the most speedup. Precisely, perfect ranks
        achieve speedups of $82\times$ for high-compression and $25\times$ for
        mid-compression, undershooting the ranks achieves speedups of $91\times$
        for high-compression and $35\times$ for mid-compression, and
        overshooting the ranks achieves speedups of $156\times$ for
        high-compression and $47\times$ for mid-compression. Low-compression is
        the first scenario where we observe nonnegligible costs of the core
        analysis subroutine. 
        %Regardless, RA-HOSI-DT achieves speedups of
        %$1.3\times$ for perfect ranks, $1.9\times$ for undershooting ranks and
        %$1.5\times$ for overshooting ranks. 
        %These reported speedups are
        %considering all three iterations, but \cref{fig:miranda_breakdown}
        %indicates that no matter the compression, perfect and undershooting
        %ranks require only two iterations to achieve the specified error,
        %overshooting the ranks only requires one iteration. 
        %Thus, considering
        %only our fastest runs to get under the threshold (which are all rank
        %overshots) we have speedups of $156\times$ for high-compression,
        %$47\times$ for mid-compression, and $3\times$ for low compression. 
        For
        high-compression, the best relative compression ratio is $69\%$ which
        occurs at perfect ranks, mid-compression achieves a $10\%$ improvement
        using perfect ranks, and low-compression has better compression at $6\%$
        when underestimating the ranks.

    \subsubsection{HCCI (4-way) and SP (5-way)} \label{sec:low_nr}
    
        We combine the discussion of the HCCI and SP datasets results, as the results are qualitatively similar. The Homogeneous Charge Compression Ignition (HHCI)
        dataset is generated from a numerical simulation of combustion
        \cite{BCL14}. The dimension of the 4-way dataset is $672\times 672\times
        33\times 626$ stored in double precision for a total of 75 GB. Thus, we
        can fit it on a single node and use all 128 cores. The first
        two modes are spatial dimensions, the third mode corresponds to 33
        variable, and the fourth mode corresponds to time steps. 
        The SP dataset is generated from the simulation of a statistically
        stationary planar methane-air flame \cite{KZCS16}. This 5-way dataset has
        dimensions $500\times 500\times 500\times 11\times 400$ stored in double
        precision and requires 4.4 TB in storage.
        For these experiments, we use 2048 cores (16 nodes). 
        The first three modes are spatial dimensions, the fourth mode
        corresponds to 11 variables, and the last mode corresponds to time steps.

        In the case where we are dominated by the TTMs, the comparisons between
        HOSI-DT and STHOSVD are less extreme. \Cref{fig:HCCI} shows that on
        low-compression, STHOSVD is faster than any of the starting ranks of
        HOSI-DT to get to the desired threshold. However, for high- and
        mid-compression HOSI-DT achieves speedups when overshooting the ranks,
        specifically $1.9\times$ for high-compression and $1.4\times$ for
        low-compression, neither of which achieved better compression.
        \Cref{fig:HCCI_breakdown} shows the breakdown times of these speedups.
        However, HODI-DT achieves better compression with perfect and under
        ranks for all error tolerances, but always requiring three iterations to
        do so. 
        
        \Cref{fig:SP} shows that we can typically obtain better compression
        after three iterations. For example, overestimating the ranks for low
        compression yields a speedup of $1.1\times$ after 1 iteration, but we do
        not obtain better compression. Similar to HCCI, three iterations
        produces a smaller Tucker approximation but takes over twice as long.
        However, for high compression, starting from perfect and underestimates
        of the ranks achieve a $27\%$ and $8\%$ improvement on compression over
        STHOSVD after two iterations, respectively. In another example,
        \cref{fig:SP_breakdown} shows that when starting from perfect estimates
        of the ranks for mid compression, HOSI-DT gets the desired error
        tolerance and same compression ratio in less time than STHOSVD, with
        HOSI-DT achieving a $1.4\times$ speedup.
        

        \begin{figure}
            \centering
            \renewcommand{\datapath}{data/HCCI/N1_n128}
            \renewcommand{\thresh}{1e-01,5e-02,1e-02}
            %!TEX root = ../../main.tex

% need to define \datapath and \thresh (set of error thresholds)

\newcommand{\algorithmname}{HOSI_DT}

\pgfplotsset{HOOIstyle/.style={
	red,
	mark=o,
	point meta = {\thisrow{STHOSVD_Size}/\thisrow{HOSI_DT_Size}},
	every node near coord/.append style={
		above right,
		inner sep=1pt,
		yshift=1ex
	}
}}
\pgfplotsset{STHOSVDstyle/.style={
	blue,
	mark=square*,
	only marks
}}


\begin{tikzpicture}
    \begin{groupplot}[
        group style={
            group size=2 by 1,
            xlabels at=edge bottom,
            ylabels at=edge left,
            horizontal sep=2cm,
            vertical sep=2cm,
        },
        width= 0.45\textwidth,
        height=0.45\textwidth,
        ylabel=Relative Error,
        ylabel near ticks,
        xlabel near ticks,
        ymajorgrids=true,
        legend to name=namedlegend,
        legend style={draw=none, cells={anchor=west}, font=\small,
        legend columns=2},
        title style={yshift=-1.5ex},
      ]
        % First plot: error vs time
        \nextgroupplot[
            title=Error vs Time, 
            xlabel=Time (in seconds),
            ytick = \thresh,
            yticklabel style={/pgf/number format/fixed}
        ]

	\foreach \t in \thresh {
		\addplot [HOOIstyle,mark=triangle] table[x=\algorithmname_Time, y=\algorithmname_Error, col sep=comma] {\datapath_\t_Over.csv};
        		\addplot [HOOIstyle,mark=star] table[x=\algorithmname_Time, y=\algorithmname_Error, col sep=comma] {\datapath_\t_Perfect.csv};
        		\addplot [HOOIstyle,mark=square] table[x=\algorithmname_Time, y=\algorithmname_Error, col sep=comma] {\datapath_\t_Under.csv};
        		\addplot [STHOSVDstyle] table[x=STHOSVD_Time, y=STHOSVD_Error, col sep=comma] {\datapath_\t_Under.csv};	
	}
      
        % Second plot: error vs relative size
        \nextgroupplot[
            title=Error vs Size,
            xlabel=Size (relative to STHOSVD),
            ytick = \thresh,
            yticklabel style={/pgf/number format/fixed}
        ]
        
        \foreach \t in \thresh {
        		\addplot[HOOIstyle,mark=triangle] table [x expr = (\thisrow{HOSI_DT_Size}/\thisrow{STHOSVD_Size}), y=HOSI_DT_Error, col sep=comma] {\datapath_\t_Over.csv};
        		\addplot[HOOIstyle,mark=star] table [x expr = (\thisrow{HOSI_DT_Size}/\thisrow{STHOSVD_Size}), y=HOSI_DT_Error, col sep=comma] {\datapath_\t_Perfect.csv};
        		\addplot[HOOIstyle,mark=square] table [x expr = (\thisrow{HOSI_DT_Size}/\thisrow{STHOSVD_Size}), y=HOSI_DT_Error, col sep=comma] {\datapath_\t_Under.csv};
        		\addplot[STHOSVDstyle] table [x expr = (\thisrow{STHOSVD_Size}/\thisrow{STHOSVD_Size}), y=STHOSVD_Error, col sep=comma] {\datapath_\t_Under.csv};
        }

        \addlegendimage{mark=triangle, color=red}
        \addlegendentry{HOSI-DT Over}
        \addlegendimage{mark=star, color=red}
        \addlegendentry{HOSI-DT Perfect}
        \addlegendimage{mark=square, color=red}
        \addlegendentry{HOSI-DT Under}
        \addlegendimage{mark=square*, color=blue, only marks}
        \addlegendentry{STHOSVD}
    \end{groupplot}
\end{tikzpicture}

% Place the legend below the figure
\begin{center}
    \ref{namedlegend}
\end{center}

 

            \caption{Progression of time, error, and relative size over 3 iterations of rank-adaptive HOSI-DT on the HCCI dataset using 128 cores.}
            \label{fig:HCCI}
        \end{figure}
        \begin{figure}
            \centering
            \renewcommand{\datapath}{data/HCCI/N1_n128}
            \renewcommand{\thresh}{1e-01,5e-02,1e-02}
            %!TEX root = ../../main.tex


\newcommand{\maxiter}{3}
\pgfmathsetmacro{\n}{\maxiter - 1}

\begin{tikzpicture}
    \begin{axis}[
        hide axis, % Hide the dummy axis
        axis lines=none,
        xmin=0, xmax=1,
        ymin=0, ymax=1,
        width= 0.45\textwidth,
        height=0.45\textwidth,
        legend style={fill=white, at={(2.25, -0.515)}, anchor=north east, font=\small},
        legend columns=1, % Arrange legend in one column
        reverse legend
    ]   
        \addplot [color=white, fill=white, mark=square*,opacity=0] coordinates {(0,0)};
        \addlegendentry{}

        \addlegendimage{mark=square*, color=blue!25, fill=blue!25, only marks, mark options={scale=2.5, fill=blue!25}}
        \addlegendentry{TTM Comp}

        \addlegendimage{mark=square*, color=blue!50, fill=blue!50, only marks, mark options={scale=2.5, fill=blue!50}}
        \addlegendentry{TTM Comm}

        \addlegendimage{mark=square*, color=red!25, fill=red!25, only marks, mark options={scale=2.5, fill=red!25}}
        \addlegendentry{Contraction Comp}

        \addlegendimage{mark=square*, color=red!50, fill=red!50, only marks, mark options={scale=2.5, fill=red!50}}
        \addlegendentry{Contraction Comm}

        \addlegendimage{mark=square*, color=yellow!50, fill=yellow!50, only marks, mark options={scale=2.5, fill=yellow!50}}
        \addlegendentry{Eig/QR}

        \addlegendimage{mark=square*, color=green!25, fill=green!25, only marks, mark options={scale=2.5, fill=green!25}}
        \addlegendentry{Core Comp}
        
        \addlegendimage{mark=square*, color=green!50, fill=green!50, only marks, mark options={scale=2.5, fill=green!50}}
        \addlegendentry{Core Comm}
    \end{axis}

    \begin{groupplot}[
        group style={
            group size=2 by 2,
            horizontal sep=2cm,
            vertical sep=2cm,
        },
        ybar stacked,
        symbolic x coords={HOSI-DT-Under, HOSI-DT-Perfect, HOSI-DT-Over, STHOSVD},
        xticklabels={Under, Perfect, Over, STHOSVD},
        xtick=data,
        ylabel={Time},
        ylabel style = {yshift=-.675cm},
        ymajorgrids=true,
        width=0.45\textwidth,
        height=0.45\textwidth,
        x tick label style={rotate=30, anchor=east},
        title style={yshift=-1.5ex},
    ]

    \xintFor #1 in \thresh \do {
        \nextgroupplot[title = {#1 Error}]
        \foreach \i in {0, ..., \n} {    
            \addplot[ybar,color=blue,fill=blue!25] 
            		table[x=Algorithm, y=TTM_Computation_Iter_\i, col sep=comma] {\datapath_#1_Breakdown.csv};  
		
            \addplot[ybar,color=blue,fill=blue!50] 
            		table[x=Algorithm, y=TTM_Communication_Iter_\i, col sep=comma] {\datapath_#1_Breakdown.csv};
		
            \addplot[ybar,color=red,fill=red!25] 
            		table[x=Algorithm, y=GramLQ_Computation_Iter_\i, col sep=comma] {\datapath_#1_Breakdown.csv};
		
            \addplot[ybar,color=red,fill=red!50] 
    		table[x=Algorithm, y=GramLQ_Communication_Iter_\i, col sep=comma] {\datapath_#1_Breakdown.csv};
		
            \addplot[ybar,color=yellow,fill=yellow!50] 
    		table[x=Algorithm, y=EigSVD_Iter_\i, col sep=comma] {\datapath_#1_Breakdown.csv};
		
            \addplot[ybar,color=green,fill=green!25] 
    		table[x=Algorithm, y=Core_Analysis_Iter_\i, col sep=comma] {\datapath_#1_Breakdown.csv};
		
            \addplot[ybar,color=green,fill=green!50] 
    		table[x=Algorithm, y=Core_Gathering_Iter_\i, col sep=comma] {\datapath_#1_Breakdown.csv};
        }
    }

\end{groupplot}
\end{tikzpicture}

            \caption{Running time breakdown for the HCCI dataset using 128 cores under different levels of compression.}
            \label{fig:HCCI_breakdown}
        \end{figure}
        

        \begin{figure}
            \centering
            \renewcommand{\datapath}{data/SP/N16_n2048}
            \renewcommand{\thresh}{1e-01,5e-02,1e-02}
            %!TEX root = ../../main.tex

% need to define \datapath and \thresh (set of error thresholds)

\newcommand{\algorithmname}{HOSI_DT}

\pgfplotsset{HOOIstyle/.style={
	red,
	mark=o,
	point meta = {\thisrow{STHOSVD_Size}/\thisrow{HOSI_DT_Size}},
	every node near coord/.append style={
		above right,
		inner sep=1pt,
		yshift=1ex
	}
}}
\pgfplotsset{STHOSVDstyle/.style={
	blue,
	mark=square*,
	only marks
}}


\begin{tikzpicture}
    \begin{groupplot}[
        group style={
            group size=2 by 1,
            xlabels at=edge bottom,
            ylabels at=edge left,
            horizontal sep=2cm,
            vertical sep=2cm,
        },
        width= 0.45\textwidth,
        height=0.45\textwidth,
        ylabel=Relative Error,
        ylabel near ticks,
        xlabel near ticks,
        ymajorgrids=true,
        legend to name=namedlegend,
        legend style={draw=none, cells={anchor=west}, font=\small,
        legend columns=2},
        title style={yshift=-1.5ex},
      ]
        % First plot: error vs time
        \nextgroupplot[
            title=Error vs Time, 
            xlabel=Time (in seconds),
            ytick = \thresh,
            yticklabel style={/pgf/number format/fixed}
        ]

	\foreach \t in \thresh {
		\addplot [HOOIstyle,mark=triangle] table[x=\algorithmname_Time, y=\algorithmname_Error, col sep=comma] {\datapath_\t_Over.csv};
        		\addplot [HOOIstyle,mark=star] table[x=\algorithmname_Time, y=\algorithmname_Error, col sep=comma] {\datapath_\t_Perfect.csv};
        		\addplot [HOOIstyle,mark=square] table[x=\algorithmname_Time, y=\algorithmname_Error, col sep=comma] {\datapath_\t_Under.csv};
        		\addplot [STHOSVDstyle] table[x=STHOSVD_Time, y=STHOSVD_Error, col sep=comma] {\datapath_\t_Under.csv};	
	}
      
        % Second plot: error vs relative size
        \nextgroupplot[
            title=Error vs Size,
            xlabel=Size (relative to STHOSVD),
            ytick = \thresh,
            yticklabel style={/pgf/number format/fixed}
        ]
        
        \foreach \t in \thresh {
        		\addplot[HOOIstyle,mark=triangle] table [x expr = (\thisrow{HOSI_DT_Size}/\thisrow{STHOSVD_Size}), y=HOSI_DT_Error, col sep=comma] {\datapath_\t_Over.csv};
        		\addplot[HOOIstyle,mark=star] table [x expr = (\thisrow{HOSI_DT_Size}/\thisrow{STHOSVD_Size}), y=HOSI_DT_Error, col sep=comma] {\datapath_\t_Perfect.csv};
        		\addplot[HOOIstyle,mark=square] table [x expr = (\thisrow{HOSI_DT_Size}/\thisrow{STHOSVD_Size}), y=HOSI_DT_Error, col sep=comma] {\datapath_\t_Under.csv};
        		\addplot[STHOSVDstyle] table [x expr = (\thisrow{STHOSVD_Size}/\thisrow{STHOSVD_Size}), y=STHOSVD_Error, col sep=comma] {\datapath_\t_Under.csv};
        }

        \addlegendimage{mark=triangle, color=red}
        \addlegendentry{HOSI-DT Over}
        \addlegendimage{mark=star, color=red}
        \addlegendentry{HOSI-DT Perfect}
        \addlegendimage{mark=square, color=red}
        \addlegendentry{HOSI-DT Under}
        \addlegendimage{mark=square*, color=blue, only marks}
        \addlegendentry{STHOSVD}
    \end{groupplot}
\end{tikzpicture}

% Place the legend below the figure
\begin{center}
    \ref{namedlegend}
\end{center}

 

            \caption{Progression of time, error, and relative size over 3 iterations of rank-adaptive HOSI-DT on the SP dataset using 2048 cores.}
            \label{fig:SP}
        \end{figure}

        \begin{figure}
            \centering
            \renewcommand{\datapath}{data/SP/N16_n2048}
            \renewcommand{\thresh}{1e-01,5e-02,1e-02}
            %!TEX root = ../../main.tex


\newcommand{\maxiter}{3}
\pgfmathsetmacro{\n}{\maxiter - 1}

\begin{tikzpicture}
    \begin{axis}[
        hide axis, % Hide the dummy axis
        axis lines=none,
        xmin=0, xmax=1,
        ymin=0, ymax=1,
        width= 0.45\textwidth,
        height=0.45\textwidth,
        legend style={fill=white, at={(2.25, -0.515)}, anchor=north east, font=\small},
        legend columns=1, % Arrange legend in one column
        reverse legend
    ]   
        \addplot [color=white, fill=white, mark=square*,opacity=0] coordinates {(0,0)};
        \addlegendentry{}

        \addlegendimage{mark=square*, color=blue!25, fill=blue!25, only marks, mark options={scale=2.5, fill=blue!25}}
        \addlegendentry{TTM Comp}

        \addlegendimage{mark=square*, color=blue!50, fill=blue!50, only marks, mark options={scale=2.5, fill=blue!50}}
        \addlegendentry{TTM Comm}

        \addlegendimage{mark=square*, color=red!25, fill=red!25, only marks, mark options={scale=2.5, fill=red!25}}
        \addlegendentry{Contraction Comp}

        \addlegendimage{mark=square*, color=red!50, fill=red!50, only marks, mark options={scale=2.5, fill=red!50}}
        \addlegendentry{Contraction Comm}

        \addlegendimage{mark=square*, color=yellow!50, fill=yellow!50, only marks, mark options={scale=2.5, fill=yellow!50}}
        \addlegendentry{Eig/QR}

        \addlegendimage{mark=square*, color=green!25, fill=green!25, only marks, mark options={scale=2.5, fill=green!25}}
        \addlegendentry{Core Comp}
        
        \addlegendimage{mark=square*, color=green!50, fill=green!50, only marks, mark options={scale=2.5, fill=green!50}}
        \addlegendentry{Core Comm}
    \end{axis}

    \begin{groupplot}[
        group style={
            group size=2 by 2,
            horizontal sep=2cm,
            vertical sep=2cm,
        },
        ybar stacked,
        symbolic x coords={HOSI-DT-Under, HOSI-DT-Perfect, HOSI-DT-Over, STHOSVD},
        xticklabels={Under, Perfect, Over, STHOSVD},
        xtick=data,
        ylabel={Time},
        ylabel style = {yshift=-.675cm},
        ymajorgrids=true,
        width=0.45\textwidth,
        height=0.45\textwidth,
        x tick label style={rotate=30, anchor=east},
        title style={yshift=-1.5ex},
    ]

    \xintFor #1 in \thresh \do {
        \nextgroupplot[title = {#1 Error}]
        \foreach \i in {0, ..., \n} {    
            \addplot[ybar,color=blue,fill=blue!25] 
            		table[x=Algorithm, y=TTM_Computation_Iter_\i, col sep=comma] {\datapath_#1_Breakdown.csv};  
		
            \addplot[ybar,color=blue,fill=blue!50] 
            		table[x=Algorithm, y=TTM_Communication_Iter_\i, col sep=comma] {\datapath_#1_Breakdown.csv};
		
            \addplot[ybar,color=red,fill=red!25] 
            		table[x=Algorithm, y=GramLQ_Computation_Iter_\i, col sep=comma] {\datapath_#1_Breakdown.csv};
		
            \addplot[ybar,color=red,fill=red!50] 
    		table[x=Algorithm, y=GramLQ_Communication_Iter_\i, col sep=comma] {\datapath_#1_Breakdown.csv};
		
            \addplot[ybar,color=yellow,fill=yellow!50] 
    		table[x=Algorithm, y=EigSVD_Iter_\i, col sep=comma] {\datapath_#1_Breakdown.csv};
		
            \addplot[ybar,color=green,fill=green!25] 
    		table[x=Algorithm, y=Core_Analysis_Iter_\i, col sep=comma] {\datapath_#1_Breakdown.csv};
		
            \addplot[ybar,color=green,fill=green!50] 
    		table[x=Algorithm, y=Core_Gathering_Iter_\i, col sep=comma] {\datapath_#1_Breakdown.csv};
        }
    }

\end{groupplot}
\end{tikzpicture}

            \caption{Running time breakdown for the SP dataset using 2048 cores under different levels of compression.}
            \label{fig:SP_breakdown}
        \end{figure}


