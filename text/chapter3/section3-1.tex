%!TEX root = ../../main.tex

Recall from \Cref{sec:Tucker Tensors and The Tucker Decomposition} that a Tucker
decomposition of a tensor $\mathcal{X}\in \mathbb{R}^{n_1\times \cdots \times
n_d}$ approximates $\mathcal{X}$ as a product of a core tensor $\mathcal{G} \in
\mathbb{R}^{r_1 \cdots r_d}$ and factor matrices $\mathbf{U}_k \in
\mathbb{R}^{n_k\times r_k} \forall k \in [d]$ where $\mathcal{X} \approx
\mathcal{\hat{X}} = \mathcal{G} \times_1 \mathbf{U}_1 \cdots \times
\mathbf{U}_d$. The optimal rank-$\mathbf{r}$ Tucker decomposition of
$\mathcal{X}$ can be expressed as a solution to the rank-specified optimization problem

\begin{equation}\label{eq:tuckeropt-rank}
    \begin{aligned}
        \min \quad & \| \mathcal{X} - (\mathcal{G} \times_1 \mathbf{U}_1 \cdots \times \mathbf{U}_d)\| \\ 
        \text{subject to } &\mathcal{G} \in \mathbb{R}^{r_1\times \cdots \times r_d}, \mathbf{U}_{k} \in \mathbb{R}^{n_k\times r_k} \forall k\in [d].
    \end{aligned}
\end{equation}

Alternatively, the error-specified formulation of the Tucker approximation
problem is given as

\begin{equation}\label{eq:tuckeropt-err}
    \begin{aligned}
        \min \quad & \prod_{j=1}^d r_j + \sum_{j=1}^d n_jr_j & \\ 
        \text{subject to } &\mathcal{G} \in \mathbf{R}^{r_1\times \cdots r_d}, \mathbf{U}_{k} \in \mathbb{R}^{n_k\times r_k} \forall k\in [d] \\
        \text{and}\quad  & \| \mathcal{X} - (\mathcal{G} \times_1 \mathbf{U}_1 \cdots \times \mathbf{U}_d) \| \leq \epsilon \|\mathcal{X}\|.
    \end{aligned}
\end{equation}

\subsection{STHOSVD} \label{sec:sthosvd} 

    We start with the state-of-the art algorithm that is capable of performing
    both rank-specified and error-specified formulations. \Cref{alg:STHOSVD}
    showcases the $d$-way construction of the STHOSVD algorithm. This method
    approximately solves either \cref{eq:tuckeropt-rank} or
    \cref{eq:tuckeropt-err} by unfolding the $k^\text{th}$ mode of the input
    tensor, computing its leaft leading singular vectors (LLSV), and then
    performing a TTM with the result to truncate the $k^\text{th}$ mode of rank
    $r_k$. Once all factor matrices have been computed, the truncated tensor has
    rank $\mathbf{r}$.

    A relative error error of $\epsilon$ can be achieved by selecting $r_k$ in
    the LLSV computation such that \(\sum_{i=r_k+1}^{n_k}\varSigma_i^2 \leq
    \epsilon^2 \|\mathcal{X}\|^2/d\), where $\varSigma_i$ is the $i^\text{th}$ largest
    singular value of the $k^\text{th}$ unfolding, see \cite{BK25} for more
    details. There are several algorithms one could choose in line
    \ref{line:sthosvd_llsv} to compute $\mathbf{U}_k$. We assume that such
    computation is performed via the eigenvalue decomposition (EVD) of the Gram
    matrix $\mathbf{G}_{(k)}\mathbf{G}_{(k)}$. 

    \begin{algorithm}
        \caption{STHOSVD}
        \label{alg:STHOSVD}
        \begin{algorithmic}
            \State{\textbf{Input:} Tensor $\mathcal{X} \in \mathbb{R}^{n_1 \times \cdots \times n_d}$} % \vspace{-0.75em}}
            \State{\hspace{3.25em} Ranks \textbf{R} $ = r_1, \dots, r_d$ \textbf{OR} relative error tolerance $\epsilon > 0$} % \vspace{-0.75em}}
            
            \State{\textbf{Output:} TTensor $\mathcal{T}$ of ranks \textbf{R} with $\mathcal{T \approx X}$ \textbf{OR} E{\footnotesize \text{RR}} $\equiv ||\mathcal{X - T}|| \leq \epsilon ||\mathcal{X}||$}
            \Function{STHOSVD}{$\mathcal{X}$, \textbf{R} or $\epsilon$}
                \State{\textbf{if} $\epsilon$ is defined \textbf{then} $\bar{\epsilon} \gets (\epsilon / \sqrt{d})\cdot||\mathcal{X}||$}
                \State{$\mathcal{G \gets X}$}
                \For{$k = 1, \dots, d$}
                    \State{$[U_k, \epsilon_k] \gets \textcolor{blue}{LLSV}(G_{(k)}, r_k \text{ or } \bar{\epsilon})$}\label{line:sthosvd_llsv} \Comment{$r_k$ leading left sing. vectors of residual} 
                    \State{$\mathcal{G} \gets \mathcal{G \times_k} U^\intercal_k$}\label{line:sthosvd_ttm} \Comment{compress in mode $k$}
                \EndFor{}
                \vspace{10pt}
                \State{E{\footnotesize \text{RR}} $\gets \displaystyle \sqrt{\sum_{k=1}^d \epsilon_k^2}$} \Comment{equivalent to $||\mathcal{X - T}||$}
                \vspace{10pt}
                \State{\textbf{return} [$\mathcal{G}, U_{1:d}$, E{\footnotesize \text{RR}}]} \Comment{$\mathcal{T \equiv \{G; \textit{$U_{1:d}$}\}}$}
            \EndFunction
        \end{algorithmic}
    \end{algorithm}

\subsection{Classic HOOI} \label{sec:Classic HOOI} 

    Our protagonist is given in \cref{alg:Classic-HOOI} and is an alternative
    method for solving the rank-specified formulation of the Tucker
    approximation problem
    \cite{kroonenberg1980principal,de2000best,kapteyn1986approach}. HOOI is a
    block coordinate descent method and so it requires initial factor matrices.
    Historically, the output factor matrices of STHOSVD have been used as input
    factor matrices for the HOOI algorithm, as the latter has often been as an
    aditional algorithm to just cheaply clean up the error. However, random
    factor matrices can be used and generally no more than two iterations are
    requireed to get a good approximation, often only one iteration is enough to
    get a descent one. 

    HOOI iteratively updates each factor matrix by performing a TTM with all but
    the $k^\text{th}$ factor matrix to obtain an intermediate tensor
    $\mathcal{Y}$ and computing the LLSV of $\mathbf{Y}_{(k)}$. The core tensor
    $\mathcal G$ can be computed oce, at the end, or at the end of every
    iteration in order to computate a per-iteration approximation error. 

    \begin{algorithm}
        \caption{HOOI}
        \label{alg:Classic-HOOI}
        \begin{algorithmic}
            \State{\textbf{Input:} Tensor $\mathcal{X} \in \mathbb{R}^{n_1 \times \cdots \times n_d}$}
            \State{\hspace{3.25em} Either Ranks \textbf{R} $ = r_1, \dots, r_d$}
            \State{\hspace{3.25em} Maximum Number of Iterations}
            
            \State{\textbf{Output:} TTensor $\mathcal{T}$ of ranks \textbf{R} with $\mathcal{T \approx X}$}
            \Function{HOOI}{$\mathcal{X}$, \textbf{R} or $\epsilon$}
                \State{Initialize factor matrices $U_{1:d}$ randomly}
                \State{$\mathcal{G \gets X}$}
                \For{Maximum Number of Iterations}
                    \For{$k = 1, \dots, d$}
                        \State{$\mathcal{Y = X} \times_1 U_1^\intercal \times_2 \cdots \times_{k-1} U_{k-1}^\intercal \times_{k+1} U_{k+1}^\intercal \times_{k+2} \cdots \times_d U_d^\intercal$}
                        \State{$U_k \gets \text{LLSV}(Y_{(k)}, r_k)$}
                    \EndFor{}
                \EndFor{}
                \State{$\mathcal{G \gets Y} \times_d U_d^\intercal$} \Comment{update core}
                \State{\textbf{return} [$\mathcal{G}, U_{1:d}$]} \Comment{$\mathcal{T \equiv \{G; \textit{$U_{1:d}$}\}}$}
            \EndFunction
        \end{algorithmic}
    \end{algorithm}

    HOOI is a good algorithm, but it can be better. As previously mentioned, we
    introduce three optimizations for the HOOI algorithm in an attempt to make
    it more competitive against STHOSVD. 

% \subsection{Motivating HOOI Optimizations}
% Comparing TuckerMPI's STHOSVD and HOOI flops costs (shown in red in
% \cref{tab:flops}), we observe that for most problems, STHOSVD's dominant cost is
% $\nicefrac{n^{d+1}}{P}$ and HOOI's dominant cost is $2d\ell\cdot
% \nicefrac{rn^d}{P}$, where $\ell$ is the number of HOOI iterations to achieve
% the same approximation error as STHOSVD. This implies that  HOOI is
% computationally cheaper if $\nicefrac nr > 2d\ell$, i.e., if the dimension
% reduction in each mode of the tensor is at least twice the number of dimensions
% times the number of HOOI iterations. As claimed in \cref{sec:intro} and
% empirically corroborated in \cref{sec:results}, when initialized randomly, HOOI
% often converges to an error comparable to STHOSVD in as few as 2 iterations.
% Under this assumption, the dimension reduction requirement for HOOI to be
% cheaper becomes $\nicefrac nr > 4d$. Although such a high degree of dimension
% reduction has been observed for some Tucker compression problems, for reasonably
% large $d$ (e.g., 4-way or 5-way tensors), it is rare. In cases of small $d$ and
% large $n$ and number of processors $P$, TuckerMPI's STHOSVD and HOOI can be bottlenecked by the
% sequential EVD computations, which are $\mathcal{O}(dn^3)$ and
% $\mathcal{O}(d\ell n^3)$ respectively.  
% In this case, we never expect HOOI to be cheaper, which we in fact observe and
% discuss in \cref{sec:synthetic_strong_scaling}.

% However, we propose two means of reducing the computational cost of HOOI
% iterations to make the method much more competitive with STHOSVD even in cases
% of smaller dimension reduction. The first optimization is dimension tree
% memoization of the TTMs that usually dominate the cost of HOOI iterations. As
% shown in \cref{sec:dimtrees}, this reduces the TTM computational cost by a
% factor of $\nicefrac d2$, and implies that the dimension reduction requirement
% for HOOI to be cheaper becomes $\nicefrac nr > 8$ (assuming 2 iterations). The
% second optimization is the use of subspace iteration to reduce the dominant cost
% of computing LLSV by a factor of $\nicefrac 14 \cdot \nicefrac nr$ and reduce
% the sequential computation within TuckerMPI by a factor of $\mathcal{O}\big((\nicefrac
% nr)^2\big)$. The details of this optimization are given in \cref{sec:HOSI}.

% Before describing the computational optimizations, we show in \cref{sec:coreanalysis} how we modify the HOOI algorithm to solve the error-specified formulation of the Tucker approximation problem.
% Given this modification, along with the computational optimizations, HOOI becomes much more competitive with STHOSVD and significantly outperforms it in certain cases of high dimension reduction.

% Our proposed algorithm RA-HOSI-DT, with all three modifications/optimizations,
% is given in \cref{alg:hooi_dimtree}. \Cref{tab:flops,tab:comm} summarize the
% cost analysis, showing the benefits of each of the two optimizations and
% comparing the final costs of RA-HOSI-DT with STHOSVD. As argued above,
% RA-HOSI-DT is computationally cheaper roughly when $\nicefrac nr > 8$, and it
% also alleviates a $\mathcal{O}(dn^3)$ sequential bottleneck in STHOSVD. It also requires
% less communication roughly when $\nicefrac nr > 2(P_1+P_d-2)$.

\subsection{HOOI's Dimension Trees Optimization}
    Adapting ranks in each HOOI iteration is a low order cost, however, the cost
    of TTMs is a factor of $d$ more expensive than in STHOSVD. We can reduce the
    cost of TTMs by avoiding redundant computations. Notice that for $k = 1$ in
    \cref{alg:Classic-HOOI} the following multi-TTM is computed $\mathcal{Y} =
    \mathcal{X} \times_2 \mathbf{U}_{2}^\intercal \times_3
    \mathbf{U}_{3}^\intercal \cdots \times_d \mathbf{U}_{d}$. At $k = 2$ the
    multi-TTM is $\mathcal{Y} = \mathcal{X} \times_2 \mathbf{U}_{1}\times
    \times_3 \mathbf{U}_{3}^\intercal \cdots \times_d \mathbf{U}_{d}$. By
    comparing the two multi-TTMs we can see that $d - 2$ TTMs are the same
    (namely 3 to $d$). So we can reuse results from one multi-TTM to the next by
    memoizing intermediate results. This idea, organized using so-called
    ``dimension trees'', was first used in the context of CP decompositions
    \cite{PTC13a} and has been applied to Tucker computations as well
    \cite{kaya2019computing,MLB24}. \Cref{fig:dimtree} shows an example
    dimension tree as we implement them for an order-$6$ tensor where each node
    represents the set of modes in which a TTM has not been performed. At the
    root of the tree, no TTMs have been performed, so the tensor is $\mathcal{X}$.
    Each notch in an edge of the tree represents a TTM in the labeled mode. At
    each leaf node, TTMs in all modes but one have been performed, so we update
    the factor matrix in that mode by performing LLSV. The core tensor $\mathcal{G}$
    is updated at the last leaf node by perform a TTM between the (memoized)
    intermediate tensor and the factor matrix corresponding to the last leaf
    node.

    \begin{figure}
        \centering
        %!TEX root = ../main.tex

\begin{tikzpicture}[scale=2]

    \node (123456) at (4,2) {$\{1,2,3,4,5,6\}$};
    \node (123) at (2,1) {$\{1,2,3\}$};
    \node (1) at (1,0) {$\{1\}$};
    \node (23) at (3,0) {$\{2,3\}$};
    \node (2) at (2,-1) {$\{2\}$};
    \node (3) at (4,-1) {$\{3\}$};

    \node (456) at (6,1) {$\{4,5,6\}$};
    \node (4) at (5,0) {$\{4\}$};
    \node (56) at (7,0) {$\{5,6\}$};
    \node (5) at (6,-1) {$\{5\}$};
    \node (6) at (8,-1) {$\{6\}$};
    
    \draw (123456) -- (123)
    	node[pos=0.25, circle, fill=black, inner sep=1pt,label=above left:{\tiny 6}] {}
	node[pos=0.5, circle, fill=black, inner sep=1pt,label=above left:{\tiny 5}] {}
	node[pos=0.75, circle, fill=black, inner sep=1pt,label=above left:{\tiny 4}] {};
    \draw (123456) -- (456)
    	node[pos=0.25, circle, fill=black, inner sep=1pt,label=above right:{\tiny 1}] {}
	node[pos=0.5, circle, fill=black, inner sep=1pt,label=above right:{\tiny 2}] {}
	node[pos=0.75, circle, fill=black, inner sep=1pt,label=above right:{\tiny 3}] {};
    \draw (123) -- (1)
	node[pos=0.33, circle, fill=black, inner sep=1pt,label=above left:{\tiny 3}] {}
	node[pos=0.66, circle, fill=black, inner sep=1pt,label=above left:{\tiny 2}] {};
    \draw (123) -- (23)
	node[pos=0.5, circle, fill=black, inner sep=1pt,label=above right:{\tiny 1}] {};
    \draw (456) -- (4)
	node[pos=0.33, circle, fill=black, inner sep=1pt,label=above left:{\tiny 6}] {}
	node[pos=0.66, circle, fill=black, inner sep=1pt,label=above left:{\tiny 5}] {};
    \draw (456) -- (56)
	node[pos=0.5, circle, fill=black, inner sep=1pt,label=above right:{\tiny 4}] {};
    \draw (23) -- (2)
	node[pos=0.5, circle, fill=black, inner sep=1pt,label=above left:{\tiny 3}] {};
    \draw (23) -- (3)
	node[pos=0.5, circle, fill=black, inner sep=1pt,label=above right:{\tiny 2}] {};
    \draw (56) -- (5)
	node[pos=0.5, circle, fill=black, inner sep=1pt,label=above left:{\tiny 6}] {};
    \draw (56) -- (6)
	node[pos=0.5, circle, fill=black, inner sep=1pt,label=above right:{\tiny 5}] {};
    

    \end{tikzpicture}
    

        \label{fig:dimtree}
        \caption[A 6-way Dimension Tree]{Illustration of multi-TTM memoization
        for an order-$6$ tensor. Each node in the tree shows the set of modes in
        which multiplication has not been performed. Each notch in an edge is a
        TTM in the labeled mode.  Factor matrices are computed at each leaf node
        in the mode shown. $\mathcal{G}$ is updated in the last leaf node.}
    \end{figure}

    \Cref{alg:dimtree} shows the HOOI iteration using dimension tree memoization
    implemented recursively.


    \begin{algorithm}
        \caption{Recursive HOOI iteration via dimension trees}
        \label{alg:dimtree}
        \begin{algorithmic}[1]
            \Function{[$\mathcal{G},\{\mathbf{U}_{k}\}] = $HOOI-DT}{$\mathcal{X},\{\mathbf{U}_{k}\},\mathbf{m},\mathbf{r}$}
                \If{length($\mathbf{m}) = 1$}
                    \State $\mathbf{U}_{m} = $ \Call{LLSV}{$\mathbf{X}_{(m)}, \mathbf{U}_m, r_m$}
                    \If{$m=d$}
                        \State $\mathcal{G} = \mathcal{X} \times_d \mathbf{U}_{m}^\intercal$
                    \EndIf
                \Else
                    \State Partition $\mathbf{m} = [\mathbf{\mu},\mathbf{\eta}]$ \State $\mathcal{X} = \mathcal{X} \scaleobj{1.7}{\times}_{k \in \mathbf{\mu}} \mathbf{U}_{k}^\intercal$
                    %$\mathcal{X} = \mathcal{X} \times_{\mathbf{\mu}(1)} \mathbf{U}__{\mathbf{\mu}(1)}^\top \times \dots \times_{\mathbf{\mu}(end)} \mathbf{U}__{\mathbf{\mu}(end)}^\top$
                    \State $[\mathcal{G},\{\mathbf{U}_{k}\}] = \Call{HOSI-DT}{\mathcal{X},\{\mathbf{U}_{k}\},\mathbf{\eta}, \mathbf{r}$}
                    \State $\mathcal{X} = \mathcal{X} \scaleobj{1.7}{\times}_{k \in \mathbf{\eta}} \mathbf{U}_{k}^\intercal$
                    %$\mathcal{X} = \mathcal{X} \times_{\mathbf{\eta}(1)} \mathbf{U}__{\mathbf{\eta}(1)}^\top \times \dots \times_{\mathbf{\eta}(end)} \mathbf{U}__{\mathbf{\eta}(end)}^\top$
                    \State $[\mathcal{G},\{\mathbf{U}_{k}\}] =\Call{HOSI-DT}{\mathcal{X},\{\mathbf{U}_{k}\}, \mathbf{\mu},\mathbf{r}$}
                \EndIf
            \EndFunction
        \end{algorithmic}
    \end{algorithm}

\subsection{HOOI's Subspace Iteration Optimization} \label{sec:HOOI's Subspace Iteration Optimization}

    So far, we have assumed that the LLSVs of a matrix $\mathbf{A}$ are obtained
    as the eigenvectors of the Gram matrix, $\mathbf{A}\mathbf{A}^\intercal$.
    % which can be seen in \cref{alg:gram_llsv}
    The next algorithmic improvement we introduce is to compute the leading left
    singular vectors by using subspace iterations. \Cref{alg:subiter} shows a
    single subspace iteration, but in principle, the computations could be
    repeated to improve accuracy.

    \begin{algorithm}
        \caption{LLSV via Subspace Iteration}
        \label{alg:subiter}
        \begin{algorithmic}[1]
        \Function{$\mathbf{Q} = $ \textcolor{red}{LLSV}}{$\mathbf{A}, \mathbf{U}, r$}
    %        \If{$method == Gram+Eig$}
    %            \State $\Mx{Z} = \Tm{B}{j} \Tm{B}{j}'$
    %            % \Comment{Comm: all-to-all and reduce-scatter}
    %            \State $\sqr{\Mx{Q}{j}, \Mx{\Lambda}} = \Call{eig}{\Mx{Z}}$
    %        \ElsIf{$method == Sub.~Iter.$}
    %	   \State $\mathbf{U}_ = \mathbf{U}_(:, 1:r)$
                \State $\mathbf{G} = \mathbf{U}^\intercal \mathbf{A} $ \label{line:SI-TTM}
                % \Comment{Comm: reduce-scatter}
                \State $\mathbf{Z} = \mathbf{A} \mathbf{G}^\intercal$ \label{line:SI-contract}
                % \Comment{Comm: reduce + broadcast}
                \State $[\mathbf{Q}, \sim, \sim] = \Call{QRCP}{\mathbf{Z}}$ \label{line:SI-QRCP}
    %        \EndIf
    %        \State $\Mx{Q}{j} = \Mx{Q}{j}(:, 1:r_j)$
        \EndFunction
        \end{algorithmic}
        %\AD{Not sure if I like branching based on $method$}
        %\GB{I think I will change this to LLSV-SI and make variables more generic}
    \end{algorithm}

    We note that the input matrix $\mathbf{A}$ is $\mathbf{Y}_{(k)}$ from
    \cref{alg:Classic-HOOI} or $\mathbf{X}_{(m)}$ from \cref{alg:dimtree},
    which is the result of an all-but-one multi-TTM, and the input matrix
    $\mathbf{U}$ is the factor matrix from the previous HOOI iteration. This implies
    that the temporary matrix $\mathbf{G}$ in \cref{alg:subiter} is an unfolding of
    the core tensor corresponding to the current set of factor matrices. That
    is, the matrix multiplication in line \ref{line:SI-TTM} is a TTM, which we
    implement using existing TuckerMPI subroutines. The multiplication in
    line \ref{line:SI-contract} is a tensor contraction in all modes but one between
    the core tensor and the result of an all-but-one multi-TTM, which is not
    implemented in TuckerMPI. Our parallel algorithm mimics the computation of
    the Gram matrix of a tensor unfolding, but it is a nonsymmetric operation
    and has different costs. Finally, we perform QR with column pivoting in
    line \ref{line:SI-QRCP} to orthonormalize the subspace iteration result and also
    order the columns to aid in core analysis, which is discussed in
    \cref{sec:HOOI's Adaptive Rank Optimization}. We choose to do only a single subspace iteration
    because we use an accurate initialization (from the previous HOOI iteration)
    and because high accuracy of a HOOI subiteration is less of a priority than
    high accuracy of the full HOOI iteration.

\subsection{HOOI's Adaptive Rank Optimization} \label{sec:HOOI's Adaptive Rank Optimization}

    \begin{figure}
        \centering
        \begin{tikzpicture}[scale=1,namenode/.style={scale=1}]
    % These distances can be meddled with to get you what you want
    % Within reason...
    \def\ix{2.5} %
    \def\iy{2.5} %
    \def\iz{1.75} %
    \def\corescale{1.75} % Has to be more than 1
    \def\rot{90}
    \def\r{0.25}
    \def\rx{\ix/\corescale}
    \def\ry{\iy/\corescale}
    \def\rz{\iz/\corescale}
    \def\redux{0.15}
    \def\reduy{0.15}
    \def\reduz{0.35}
    
    % Give figure's starting point and draw a rectange
    \coordinate (TFrontLowerLeft) at (0,0);
    \draw (TFrontLowerLeft) rectangle ++ (\ix,\iy);
 
    % This scope draws the top face of the cube
    \begin{scope}[shift={(TFrontLowerLeft)}, canvas is zx plane at y=\iy,rotate=\rot]
       \draw (0,0) rectangle ++ (\ix,\iz);
    \end{scope}
 
    % This scope draws the left side of the cube
    \begin{scope}[shift={(TFrontLowerLeft)},canvas is zy plane at x=\ix,rotate=90]
       \draw (0,0) rectangle ++ (\iy,\iz); %
    \end{scope}
 
    % This writes the tensor name in the front face of the cube
    \node[namenode] at ($(TFrontLowerLeft) + (0.5*\ix, 0.5*\iy)$)  {$\mathbf{\mathcal{T}}$};
 
    % This writes the equal sign to the side of the cube
    \coordinate (ApproxCtr) at ($(TFrontLowerLeft) + (0.75+\ix+0.4*\iz,0.7*\iy)$);
    \node[namenode] at (ApproxCtr) {$\approx$};
 
    % First Factor Matrix
    \coordinate (AFrontLowerLeft) at ($(ApproxCtr) + (0.75,-\ry)$);
    \draw (AFrontLowerLeft) rectangle ++ (\rx,\iy);
    \node[namenode] at ($(AFrontLowerLeft) + (0.5*\rx, 0.5*\iy)$)  {$A$};
    \draw[red, fill=red, fill opacity = 0.5] (AFrontLowerLeft) rectangle ++ (\rx-\redux,\iy);
 
    % This scope draws the top face of the cube
    \coordinate (CFrontLowerLeft) at ($(AFrontLowerLeft) + (0.5 + \rx, 0.25*\ix)$);
    \draw (CFrontLowerLeft) rectangle ++ (\rx,\ry);
    \coordinate (SubCore) at ($(CFrontLowerLeft) + (0, \reduy)$);
    \draw[red, fill=red, fill opacity = 0.5] (SubCore) rectangle ++ (\rx - \redux, \ry - \reduy);
 
    \begin{scope}[shift={(CFrontLowerLeft)}, canvas is zx plane at y=\ry,rotate=\rot]
       \draw (0,0) rectangle ++ (\rx,\rz);
    \end{scope}
    \begin{scope}[shift={(SubCore)}, canvas is zx plane at y=\ry-\reduy,rotate=\rot]
       \draw[red, fill=red, fill opacity = 0.5] (0,0) rectangle ++ (\rx-\redux,\rz-\reduz);
    \end{scope}
    
    % This scope draws the left side of the cube
    \begin{scope}[shift={(CFrontLowerLeft)},canvas is zy plane at x=\rx,rotate=90]
       \draw (0,0) rectangle ++ (\ry,\rz); %
    \end{scope}
    \begin{scope}[shift={(SubCore)},canvas is zy plane at x=\rx-\redux,rotate=90]
       \draw[red, fill=red, fill opacity = 0.5] (0,0) rectangle ++ (\ry-\reduy,\rz-\reduz); %
    \end{scope}
 
    % This writes the core name in the front face of the cube
    \node[namenode] at ($(CFrontLowerLeft) + (0.5*\rx, 0.5*\ry)$)  {$\mathbf{\mathcal{G}}$};
 
    % Second Factor Matrix
    \coordinate (BFrontLowerLeft) at ($(CFrontLowerLeft) + (\rx + 0.75*\rz, 0.15*\rz)$);
    \draw (BFrontLowerLeft) rectangle ++ (\ix,\ry);
    \node[namenode] at ($(BFrontLowerLeft) + (0.5*\ix, 0.5*\ry)$)  {$B$};
    \draw[red, fill=red, fill opacity = 0.5] ($(BFrontLowerLeft) + (0, \reduy)$) rectangle ++ (\ix,\ry-\reduy);
 
    % Third Factor Matrix
    \begin{scope}[shift={($(CFrontLowerLeft) + (0.25*\rz,-0.25*\rz)$)},canvas is zx plane at y=\iy,rotate=90]
       \draw (0,\r) rectangle ++ (\rz*1.3,\iz*1.15);
    \end{scope}
    \begin{scope}[shift={($(CFrontLowerLeft) + (0.25*\rz,-0.25*\rz)$)},canvas is zx plane at y=\iy,rotate=90]
       \draw[red, fill=red, fill opacity = 0.5] (0,\r) rectangle ++ (\rz*1.3 - \reduz,\iz*1.15);
    \end{scope}
    \node[namenode] at ($(CFrontLowerLeft) + (1*\rx,2.75*\rz)$) {$C$};
 
         
 \end{tikzpicture}
        \caption{Adaptive HOOI}
        \label{fig:adaptive_hooi}
    \end{figure}

    A significant disadvantage of HOOI is that it solves only the rank-specified
    formulation of the Tucker approximation problem, whereas STHOSVD can
    adaptively select ranks based on a relative error tolerance. We propose a
    technique that allows HOOI to automatically adapt ranks to meet a
    user-specified relative error tolerance. 

    Recall that for the error-specified formulation, given an error tolerance
    $\varepsilon$ and an initial rank estimate $\mathbf{r}$, our method adaptively
    finds a Tucker decomposition $\mathbf{\hat{X}} = [\mathcal{G}; \mathbf{U}_{1}, \dots,
    \mathbf{U}_{d}]$ for a tensor $\mathcal{X} \in \mathbb{R}^{n_1 \times \dots \times
    n_d}$ such that $\| \mathbf{\hat{X}} - \mathcal{X} \| \leq \varepsilon \| \mathcal{X} \|$.
    We start with a typical HOOI iteration using our initial rank estimate
    $\mathbf{r}$, partially compressing our tensor in all modes except mode $k$ and
    updating factor matrix $\mathbf{U}_{k}$ to be the first $r_k$ left singular
    vectors of the partially compressed tensor. Once all modes have been
    processed in this manner, we check the error of the approximation at that
    point. 
    Whereas in clasical HOOI the core is only updated after the iterations, here
    we compute the core tensor at the end of everu iteration and perform error
    analysis on it. To check the error, we use the identity that for orthonormal
    matrices $\mathbf{U}_{1},\dots,\mathbf{U}_{d}$ and $\mathcal{G} = \mathcal{X} \times_1
    \mathbf{U}_{1}\intercal \times \dots \times_d \mathbf{U}_{d}^\intercal$, the approximation error can be
    written as $\| \mathcal{X} - \mathbf{\hat{X}} \|^2 = \| \mathcal{X} - \mathcal{G} \times_1
    \mathbf{U}_{1} \times \dots \times_d \mathbf{U}_{d} \|^2 = \| \mathcal{X} \|^2 - \| \mathcal{G}
    \|^2$ (\cite[Proposition 6.3]{BK25}). If the current Tucker approximation is
    not sufficiently accurate, we increase all ranks by a factor $\alpha$ and
    perform the next HOOI iteration. If the current approximation satisfies the
    error threshold, then we can optimize over all rank truncations by analyzing
    the core tensor's entries.
    We can thus estimate the relative error in the approximation by computing
    $\| \mathcal{G} \|$, and choosing the next rank $\mathbf{r}$ so that $\|
    \mathcal{G}(\mathbf{1}:\mathbf{r})\|^2 \approx (1 - \varepsilon^2) \| \mathcal{X} \|^2.$ 
    Specifically, we solve the optimization problem
    \begin{equation}\label{eq:rankcond}
    \begin{aligned}
    \min_{\mathbf{r}}  \quad & \Pi_{j=1}^d r_j + \Sigma_{j=1}^{d} n_j r_j,\\ 
    \text{ subject to } &\|\mathcal{G}(\mathbf{1}:\mathbf{r}) \|^2 \geq (1 - \varepsilon^2) \| \mathcal{X} \|^2.
    \end{aligned}
    \end{equation}
    This computes the leading subtensor of $\mathcal{G}$ that minimizes the size of
    the Tucker approximation and also satisfies the error threshold. Note that
    any subtensor of the core, along with the corresponding columns of the
    factor matrices, is a valid Tucker approximation with error determined by
    the norm of the core subtensor. The optimal subtensor need not be a leading
    one, but we order factor matrix columns to concentrate the weight of
    $\mathcal{G}$ towards the entry of smallest index value so that the heuristic of
    searching over only leading subtensors is reasonable.

    If such a rank $\mathbf{r}$ exists, we set our next rank as the closest
    index satisfying \eqref{eq:rankcond} and truncate to that rank before
    iterating. If no $\mathbf{r}$ exists, our current rank is too small, so we
    increase it by a small factor $\alpha$ before trying again. Typically,
    $\alpha \approx 2$ is sufficient. The details of this algorithm, are
    described in \ref{alg:Adaptive-HOOI}. In practice, we do the optimization
    problem above in a way that minimizes the memory footprint. We do so by
    exploiting HOOI's form of the immediate core $\mathcal{G}$. We use the
    immediate form to compute its cumulative sum of the squared core
    $\mathcal{G}^2$ (square all entries in the core) and then consider all
    values of this cumulative sum squared tensor to find the first instance that
    satisfies the error tolerance to get the new ranks.

    \begin{algorithm}
        \caption{Adaptive HOOI}
        \label{alg:Adaptive-HOOI}
        \begin{algorithmic}
            \Function{performCoreAnalysis}{$\mathcal{G}, \epsilon, \mathbf{r}$}
                \If{$||\mathcal{G} ||^2 \geq (1 - \epsilon^2)||\mathcal{X}||^2$}
                    \State{Find \textbf{r} = arg min $||\mathcal{G}(1:\textbf{r}) ||^2$}
                    \State{$\hspace{0.1375\linewidth}$subject to $||\mathcal{G}(1:\textbf{r}) ||^2 \geq (1 - \epsilon^2)||\mathcal{X}||^2$}
                    \State{}
                    \State{Truncate $\mathcal{G}, A, B, C$ according to $\mathbf{r}$}
                \Else{}
                    \State{\textbf{r} = $\alpha$ \textbf{r}}
                    \State{Increase columns of $ A, B, C$ according to $\mathbf{r}$}
                \EndIf{}
                \State{\textbf{return} $\mathbf{r}$}
            \EndFunction{}
        \end{algorithmic}
    \end{algorithm}

    % \begin{eqnarray*}
    %     \text{min} ||\mathcal{X - G} \times U_1 \times_2 \cdots \times_d U_d || \\
    %     \text{subject to } \mathcal{G} \in \mathbb{R}^{n_1 \times \cdots \times n_d}, U_k \in \mathbb{R}^{n_k \times r_k} \text{ } \forall k \in [d]
    % \end{eqnarray*}
    
    % Now suppose we have a relative error tolerance of how accurate we want
    % our approximation to be. Then the approximation must satisfy:
    % \begin{eqnarray*}
    %     \frac{||\mathcal{X - T}||}{||\mathcal{X}||} \leq \epsilon \\
    %     \frac{||\mathcal{X - T}||^2}{||\mathcal{X}||^2} \leq \epsilon^2 \\
    %     \therefore ||\mathcal{X - T}||^2 \leq \epsilon^2 \cdot ||\mathcal{X}||^2 \\
    %     \therefore ||\mathcal{X}||^2 - ||\mathcal{G}||^2 \leq \epsilon^2 \cdot ||\mathcal{X}||^2 \\
    %     \therefore ||\mathcal{X}||^2 - \epsilon^2 \cdot ||\mathcal{X}||^2 \leq ||\mathcal{G}||^2 \\
    %     \therefore (1 - \epsilon^2) \cdot ||\mathcal{X}||^2 \leq ||\mathcal{G}|| \\
    % \end{eqnarray*}
    
    % We can thus estimate the relative error in the approximation by
    % computing $||\mathcal{G}||^2$ and choosing the next rank-\textbf{R} so
    % that $||\mathcal{G}(1:\textbf{R}) ||^2 \approx (1 -
    % \epsilon^2)||\mathcal{X}||^2$. Specifically, we solve the optimization
    % problem
    % \begin{eqnarray*}
    %     \underbrace{\text{arg min}}_\textbf{R} ||\mathcal{G}(1:\textbf{R})||^2 \\
    %     \text{subject to } ||\mathcal{G}(1:\textbf{R}) ||^2 \geq (1 - \epsilon^2)||\mathcal{X}||^2
    % \end{eqnarray*}
    
    
    % \begin{algorithm}
    %     \caption{Adaptive HOOI}
    %     \label{alg:Adaptive-HOOI}
    %     \begin{algorithmic}
    %         \State{\textbf{Input:} Tensor $\mathcal{X} \in \mathbb{R}^{n_1 \times \cdots \times n_d}$}
    %         \State{\hspace{3.25em} Either Ranks \textbf{R} $ = r_1, \dots, r_d$}
    %         \State{\hspace{3.25em} Maximum Number of Iterations}
    %         \State{\hspace{3.25em} Rank Size Increase Rate $\alpha$}
            
    %         \State{\textbf{Output:} TTensor $\mathcal{T}$ of ranks \textbf{R} with $\mathcal{T \approx X}$}
    %         \Function{AdaptiveHOOI}{$\mathcal{X}$, \textbf{R} or $\epsilon$}
    %             \State{Initialize factor matrices $U_{1:d}$ randomly}
    %             \State{$\mathcal{G \gets X}$}
    %             \For{Maximum Number of Iterations}
    %                 \For{$k = 1, \dots, d$}
    %                     \State{$\mathcal{Y = X} \times_1 U_1^\intercal \times_2 \cdots \times_{k-1} U_{k-1}^\intercal \times_{k+1} U_{k+1}^\intercal \times_{k+2} \cdots \times_d U_d^\intercal$}
    %                     \State{$U_k \gets \text{LLSV}(Y_{(k)}, r_k)$}
    %                 \EndFor{}
    %                 \State{$\mathcal{G \gets Y} \times_d U_d^\intercal$} \Comment{update core}
    %                 \If{$||\mathcal{G} ||^2 \geq (1 - \epsilon^2)||\mathcal{X}||^2$}
    %                     \State{Find \textbf{R} = arg min $||\mathcal{G}(1:\textbf{R}) ||^2$,subject to $||\mathcal{G}(1:\textbf{R}) ||^2 \geq (1 - \epsilon^2)||\mathcal{X}||^2$}
    %                     \State{$\mathcal{G = G}(1:\textbf{R}, U_k = U_k(1:r_k) \text{  } \forall k \in [d]$}
    %                 \Else{}
    %                     \State{\textbf{R} = $\alpha$ \textbf{R}}
    %                 \EndIf{}
    %             \EndFor{}
    %             \State{\textbf{return} [$\mathcal{G}, U_{1:d}$]} \Comment{$\mathcal{T \equiv \{G; \textit{$U_{1:d}$}\}}$}
    %         \EndFunction
    %     \end{algorithmic}
    % \end{algorithm}
    